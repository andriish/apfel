%% LyX 2.0.2 created this file.  For more info, see http://www.lyx.org/.
%% Do not edit unless you really know what you are doing.
\documentclass[twoside,english]{article}
\usepackage{lmodern}
\usepackage[T1]{fontenc}
\usepackage[latin9]{inputenc}
\usepackage[a4paper]{geometry}
\geometry{verbose,tmargin=3cm,bmargin=2.5cm,lmargin=2cm,rmargin=2cm}
\usepackage{babel}
\usepackage{float}
\usepackage{bm}
\usepackage{amsmath}
\usepackage{esint}
\usepackage[unicode=true,pdfusetitle,
 bookmarks=true,bookmarksnumbered=false,bookmarksopen=false,
 breaklinks=false,pdfborder={0 0 0},backref=false,colorlinks=false]
 {hyperref}

\makeatletter

%%%%%%%%%%%%%%%%%%%%%%%%%%%%%% LyX specific LaTeX commands.
%% Because html converters don't know tabularnewline
\providecommand{\tabularnewline}{\\}

%%%%%%%%%%%%%%%%%%%%%%%%%%%%%% Textclass specific LaTeX commands.
%\numberwithin{equation}{section}

\makeatother

\begin{document}

\title{DGLAP Evolution Equation on the Complex Plane}

\author{Valerio Bertone}
\maketitle
\begin{abstract}
In this document I describe the extension of the DGLAP equation for
complex values of the factorization scale $\mu$ and the relative
implementation in {\tt APFEL}.
\end{abstract}

\section{DGLAP on the Complex Plane}

Let us start from the usual DGLAP evolution equation for the
distribution $f$(\footnote{At this stage it is not necessary to
  distinguish between singlet or non-singlet distributions. We will
  consider these cases separately later once the formalism has been
  settled.}):
\begin{equation}\label{DGLAPStandard}
\mu^2\frac{\partial f}{\partial \mu^2} =
P\left(x,\alpha_s(\mu)\right)\otimes f(x,\mu)\,
\end{equation}
where $\otimes$ represents the usual Mellin convolution such that:
\begin{equation}
A(x)\otimes B(x) \equiv \int_0^1dz \int_0^1dy A(y)B(z)\delta(x-yz)=
\int^1_x\frac{dy}{y}
A(x)B\left(\frac{y}{x}\right) = \int^1_x\frac{dy}{y}
A\left(\frac{y}{x}\right)B(x)\,.
\end{equation}
Now we consider the RGE for the strong coupling $\alpha_s$, that reads:
\begin{equation}\label{RGEalphas}
\mu^2\frac{\partial \alpha_s}{\partial \mu^2} = \beta(\alpha_s)\,,
\end{equation}
and combine it with the DGLAP equation in eq.~(\ref{DGLAPStandard}),
obtaining the DGLAP equation differential in $\alpha_s$:
\begin{equation}\label{DGLAPalphas}
\frac{\partial f}{\partial \alpha_s} =
R\left(x,\alpha_s\right)\otimes f(x,\alpha_s)\,
\end{equation}
where:
\begin{equation}
R\left(x,\alpha_s\right) = \frac{P\left(x,\alpha_s\right)}{\beta(\alpha_s)}\,.
\end{equation}

The next fundamental step is the promotion of the factorization scale
$\mu$ from a real to a complex variable:
\begin{equation}
\mu \rightarrow \eta = \mu + i\nu\,.
\end{equation}
As a consequence, we need to promote also the strong the PDF $f$ and
the strong coupling $\alpha_s$ to being complex functions, that is:
\begin{equation}
\begin{array}{rcl}
f &\rightarrow& F = f + ig\,,\\
\alpha_s &\rightarrow& \zeta_s = \alpha_s + i\xi_s\,.
\end{array}
\end{equation}
This has as a further consequence that the DGLAP and the $\alpha_s$
evolution equations in eqs.~(\ref{DGLAPStandard})
and~(\ref{RGEalphas}) become complex differential equations:
\begin{equation}
\eta^2\frac{\partial F}{\partial \eta^2} =
P\left(x,\zeta_s(\eta)\right)\otimes F(x,\eta)\,,
\end{equation}
and:
\begin{equation}
\eta^2\frac{\partial \zeta_s}{\partial \eta^2} = \beta(\zeta_s)\,,
\end{equation}
that can be again combined in:
\begin{equation}\label{DGLAPalphasComplex}
\frac{\partial F}{\partial \zeta_s} =
R\left(x,\zeta_s\right)\otimes F(x,\zeta_s)\,.
\end{equation}

The main goal is the solution of eq.~(\ref{DGLAPalphasComplex}). The
starting observation is the fact that the complex function $F$ must be
an analytical function of the complex variable $\zeta_s$. This implies
that the real and the complex parts of $F$ must obey the
Cauchy-Riemann equations, that is:
\begin{equation}
\begin{array}{rcl}
\displaystyle \frac{\partial f}{\partial \alpha_s} & = & \displaystyle
                                                         \frac{\partial
                                                         g}{\partial
                                                         \xi_s} \\
\\
\displaystyle \frac{\partial f}{\partial \xi_s} & = & \displaystyle
                                                         -\frac{\partial
                                                         g}{\partial
                                                         \alpha_s}
\end{array}\,,
\end{equation}
so that the derivative of $F$ with respect to $\eta_s$ can be expanded
as:
\begin{equation}\label{ComplexDerivative}
\frac{\partial F}{\partial \zeta_s} = \frac{\partial f}{\partial
  \alpha_s} + i \frac{\partial g}{\partial \alpha_s} =\frac{\partial g}{\partial
  \xi_s} - i \frac{\partial f}{\partial \xi_s}\,.
\end{equation}
Now let us consider the function $R$. Being it a complex function, it
can be split into a real and a complex part:
\begin{equation}
R = S + i T\,,
\end{equation}
and thus:
\begin{equation}\label{Rexp}
R\otimes F = ( S + i T )\otimes ( f+i g ) = ( S \otimes f - T \otimes g )
+ i ( T \otimes f + S \otimes g )\,.
\end{equation}

We can now combine eqs.~(\ref{ComplexDerivative}) and~(\ref{Rexp})
into eq.~(\ref{DGLAPalphasComplex}). This allows us to obtain two sets
of coupled real differential equations that can be written in the following
matricial form:
\begin{equation}\label{DGLAPcomplex1}
\frac{\partial }{\partial \alpha_s}{f \choose g} = 
\begin{pmatrix}
S & -T \\
T & S
\end{pmatrix}
\otimes {f \choose g}\,,
\end{equation}
and:
\begin{equation}\label{DGLAPcomplex2}
\frac{\partial }{\partial \xi_s}{f \choose g} = 
\begin{pmatrix}
-T & -S \\
S & -T
\end{pmatrix}
\otimes {f \choose g}\,.
\end{equation}
The solution of eqs.~(\ref{DGLAPcomplex1}) and~(\ref{DGLAPcomplex2})
allows one to obtain the dependence of the real functions $f$ and $g$
($i.e.$ the real and the complex part of the ``complex'' PDF $F$) on
the real variables $\alpha_s$ and $\xi_s$ ($i.e.$ the real and the
complex part of the ``complex'' strong coupling $\zeta_s$) which in
turn are functions of the complex factorization scale $\eta$. It
should be noticed that while solving eq.~(\ref{DGLAPcomplex1}) the
value of $\xi_s$ should be kept constant and conversely while solving
eq.~(\ref{DGLAPcomplex2}) the value of $\alpha_s$ should be kept
constant. Geometrically, this means that eq.~(\ref{DGLAPcomplex1})
allows one to compute the PDF evolution along the real axis in the
complex plane of $\zeta_s$ while eq.~(\ref{DGLAPcomplex2}) allows one
to compute the PDF evolution along the immaginary axis.  Of course, a
suitable combination of these evolution allows to reach any point of
the complex plane of $\zeta_s$ starting from any other point. This is
strictly true only if no branch cut is crossed during the
evolution. Finally, it is interesting to notice that the splitting
function matices in the r.h.s. of eqs.~(\ref{DGLAPcomplex1})
and~(\ref{DGLAPcomplex2}) commute. This has the consequence that the
order in which the derivarives with respect of $\alpha_s$ and $\xi_s$
does not affect the result. Of course, this feature must be reflected
in the solutions of eqs.~(\ref{DGLAPcomplex1})
and~(\ref{DGLAPcomplex2}). In oder words, this means that, aslo on the
complex plane, the evolution factor to be applied to the initial state
PDF only depends on the initial and the final point and not on the
path followed to connect the two points.

Now we need to extract the functions $S$ and $T$ from $R$ and in the
next section we will show their form at leading order (LO) in QCD.

\subsection{Solution at LO}

At LO in QCD we have that:
\begin{equation}
R\left(x,\zeta_s\right) =
\frac{P\left(x,\zeta_s\right)}{\beta(\zeta_s)} =
-\frac{P^{(0)}(x)}{\beta_0} \frac{1}{\zeta_s} = -\frac{P^{(0)}(x)}{\beta_0} \frac{\alpha_s-i\xi_s}{\alpha^2_s+\xi_s^2}
\end{equation}
and thus:
\begin{equation}\label{LOKernels}
S(x,\alpha_s,\xi_s) = -\frac{P^{(0)}(x)}{\beta_0}
\frac{\alpha_s}{\alpha^2_s+\xi_s^2}\quad\mbox{and}\quad T(x,\alpha_s,\xi_s) = -\frac{P^{(0)}(x)}{\beta_0} \frac{-\xi_s}{\alpha^2_s+\xi_s^2}\,.
\end{equation}
Using eq.~(\ref{LOKernels}), eqs.~(\ref{DGLAPcomplex1})
and~(\ref{DGLAPcomplex2}) become:
\begin{equation}\label{DGLAPcomplex1LO}
\frac{\partial }{\partial \alpha_s}{f \choose g} = 
-\frac{P^{(0)}(x)}{\beta_0}
\frac{1}{\alpha^2_s+\xi_s^2} \begin{pmatrix}
\alpha_s & \xi_s \\
-\xi_s & \alpha_s
\end{pmatrix}
\otimes {f \choose g}\,,
\end{equation}
and:
\begin{equation}\label{DGLAPcomplex2LO}
\frac{\partial }{\partial \xi_s}{f \choose g} = 
-\frac{P^{(0)}(x)}{\beta_0}
\frac{1}{\alpha^2_s+\xi_s^2} \begin{pmatrix}
\xi_s & -\alpha_s \\
\alpha_s & \xi_s
\end{pmatrix}
\otimes {f \choose g}\,.
\end{equation}
In Mellin space, the Mellin convolution of the equations above becomes
a simple product and can then be solved more easily. In fact, the
equations above in Mellin space become:
\begin{equation}\label{DGLAPcomplex1LOMellin}
\frac{\partial }{\partial \alpha_s}{f \choose g} = 
-\frac{\gamma^{(0)}(N)}{\beta_0}
\frac{1}{\alpha^2_s+\xi_s^2} \begin{pmatrix}
\alpha_s & \xi_s \\
-\xi_s & \alpha_s
\end{pmatrix}
{f \choose g}\,,
\end{equation}
and:
\begin{equation}\label{DGLAPcomplex2LOMellin}
\frac{\partial }{\partial \xi_s}{f \choose g} = 
-\frac{\gamma^{(0)}(N)}{\beta_0}
\frac{1}{\alpha^2_s+\xi_s^2} \begin{pmatrix}
\xi_s & -\alpha_s \\
\alpha_s & \xi_s
\end{pmatrix}
{f \choose g}\,.
\end{equation}
Defining:
\begin{equation}
\mathbf{F} \equiv {f \choose g}\quad\mbox{and}\quad R_0 = \frac{\gamma^{(0)}(N)}{\beta_0}\,,
\end{equation}
and considering that:
\begin{equation}
\int dx\frac{x}{x^2+y^2} = \frac12\ln(x^2+y^2) \quad\mbox{and}\quad
\int dx\frac{y}{x^2+y^2} = \mbox{atan}\left(\frac{x}{y}\right)=\frac{\pi}2-\mbox{atan}\left(\frac{y}{x}\right)\,,
\end{equation}
the solution of eqs.~(\ref{DGLAPcomplex1LOMellin})
and~(\ref{DGLAPcomplex2LOMellin}) is:
\begin{equation}\label{SolutionLO1}
\mathbf{F}(N,\alpha_s,\xi_s) = \mathbf{F}(N,\alpha_{s,0},\xi_s)\exp\left[-R_0\Gamma_\alpha\right]\,,
\end{equation}
and:
\begin{equation}\label{SolutionLO2}
\mathbf{F}(N,\alpha_s,\xi_s) = \mathbf{F}(N,\alpha_{s},\xi_{s,0})\exp\left[-R_0\Gamma_\xi\right]\,,
\end{equation}
with:
\begin{equation}
\Gamma_\alpha=
\begin{pmatrix}
\frac12\ln\left(\frac{\alpha_s^2+\xi_s^2}{\alpha_{s,0}^2+\xi_s^2}\right) &
-\mbox{atan}\left(\frac{\xi_s}{\alpha_s}\right) +
\mbox{atan}\left(\frac{\xi_s}{\alpha_{s,0}}\right) \\
\mbox{atan}\left(\frac{\xi_s}{\alpha_s}\right) -
\mbox{atan}\left(\frac{\xi_s}{\alpha_{s,0}}\right)
& \frac12\ln\left(\frac{\alpha_s^2+\xi_s^2}{\alpha_{s,0}^2+\xi_s^2}\right)
\end{pmatrix}\,,
\end{equation}
and:
\begin{equation}
\Gamma_\xi=
\begin{pmatrix}
\frac12\ln\left(\frac{\alpha_s^2+\xi_s^2}{\alpha_{s}^2+\xi_{s,0}^2}\right) &
-\mbox{atan}\left(\frac {\xi_s}{\alpha_s}\right) +
\mbox{atan}\left(\frac {\xi_{s,0}}{\alpha_{s}}\right) \\
\mbox{atan}\left(\frac{\xi_s}{\alpha_s}\right) -
\mbox{atan}\left(\frac{\xi_{s,0}}{\alpha_{s}}\right) & \frac12\ln\left(\frac{\alpha_s^2+\xi_s^2}{\alpha_{s}^2+\xi_{s,0}^2}\right)
\end{pmatrix}\,.
\end{equation}
It is interesting to observe that, if the complex strong coupling
$\zeta_s$ becomes real, $i.e.$ $\xi_s\rightarrow0$, the matrices above become:
\begin{equation}
\lim_{\xi_s\rightarrow0}\Gamma_\alpha= \ln\left(\frac{\alpha_s}{\alpha_{s,0}}\right)
\begin{pmatrix}
1 & 0 \\
0 & 1
\end{pmatrix}\,,
\end{equation}
and:
\begin{equation}
\lim_{\xi_s\rightarrow0} \Gamma_\xi= 0\,,
\end{equation}
and thus the evolution in $\alpha_s$ of the real part of the PDF $f$
reduces to the expected one while no evolution in $\xi_s$ is left. As
for the evolution in $\alpha_s$ of the imaginary part of the PDF $g$,
there is an evolution factor but it is decoupled from the real part
and thus it has an effect only if the imaginary part of the initial
scale PDF is different from zero.

The solutions in eqs.~(\ref{SolutionLO1}) and~(\ref{SolutionLO2}) are
written in a pretty formal way because they imply the exponential of
matrices. However, since $\Gamma_\alpha$ and $\Gamma_\xi$ are 2 by 2
matrices, their exponential in known a simple closed form. In
particular:
\begin{equation}\label{ExpMatrix}
\exp\begin{pmatrix}
a & b \\
c & d
\end{pmatrix} = \frac{\exp\left[(a+d)/2\right]}{\Delta}
\begin{pmatrix}
m_{11} & m_{12} \\
m_{21} & m_{22}
\end{pmatrix}
\end{equation}
where:
\begin{equation}
\Delta = \sqrt{(a-d)^2+4bc}
\end{equation}
and:
\begin{equation}\label{ExpMatrixEntries}
\begin{array}{rcl}
m_{11} & = & \displaystyle \Delta\mbox{cosh}\left(\frac{\Delta}2\right)+(a-d)\mbox{sinh}\left(\frac{\Delta}2\right)\\
\\
m_{12} & = & \displaystyle 2 b\,\mbox{sinh}\left(\frac{\Delta}2\right)\\
\\
m_{21} & = & \displaystyle 2 c\,\mbox{sinh}\left(\frac{\Delta}2\right)\\
\\
m_{22} & = & \displaystyle \Delta\mbox{cosh}\left(\frac{\Delta}2\right)-(a-d)\mbox{sinh}\left(\frac{\Delta}2\right)
\end{array}
\end{equation}
Given the structure of $\Gamma_\alpha$ and $\Gamma_\xi$, we can
simplify the formulas above. In the case of $\Gamma_\alpha$ the
structure is:
\begin{equation}
\Gamma_\alpha = \begin{pmatrix}
a & -b \\
b & a
\end{pmatrix}
\end{equation}
with:
\begin{equation}
\begin{array}{rcl}
a &=&\displaystyle
\frac12\ln\left(\frac{\alpha_s^2+\xi_s^2}{\alpha_{s,0}^2+\xi_s^2}\right)
      = \ln\left(\frac{|\zeta_s|}{|\zeta_{s,0}|}\right)\\
\\
b &=&\displaystyle \mbox{atan}\left(\frac {\xi_s}{\alpha_s}\right) -
\mbox{atan}\left(\frac {\xi_s}{\alpha_{s,0}}\right) = \theta -
      \theta_0 = \Delta\theta\,,
\end{array}
\end{equation}
and thus, after some simplifications, we find:(\footnote{Notice that:
$$
a+ib=\ln\left(\frac{\zeta_s}{\zeta_{s,0}}\right)\,.
$$
}):
\begin{equation}\label{AlphaEvolutionLO}
\exp\left[-R_0\Gamma_\alpha\right] = \exp[-R_0a]
\begin{pmatrix}
\cos\left(-R_0b\right) & -\sin\left(-R_0b\right) \\
\sin\left(-R_0b\right) & \cos\left(-R_0b\right)
\end{pmatrix}
=\left(\frac{|\zeta_s|}{|\zeta_{s,0}|}\right)^{-R_0}
\begin{pmatrix}
\cos\left(-R_0\Delta\theta\right) & -\sin\left(-R_0\Delta\theta\right) \\
\sin\left(-R_0\Delta\theta\right) & \cos\left(-R_0\Delta\theta\right)
\end{pmatrix}\,.
\end{equation}

For $\Gamma_\xi$,the structure is exactly the same:
\begin{equation}
\Gamma_\alpha = \begin{pmatrix}
c & -d \\
d & c
\end{pmatrix}
\end{equation}
with:
\begin{equation}
\begin{array}{rcl}
c &=&\displaystyle
      \frac12\ln\left(\frac{\alpha_s^2+\xi_s^2}{\alpha_{s}^2+\xi_{s,0}^2}\right)
      = \ln\left(\frac{|\zeta_s|}{|\zeta_{s,0}|}\right)\\
\\
d &=&\displaystyle \mbox{atan}\left(\frac{\xi_s}{\alpha_s}\right) -
\mbox{atan}\left(\frac{\xi_{s,0}}{\alpha_{s}}\right) = \theta -
      \theta_0 = \Delta\theta\,,
\end{array}
\end{equation}
and thus also here we find:
\begin{equation}\label{XiEvolutionLO}
\exp\left[-R_0\Gamma_\xi\right] = \exp[-R_0c]
\begin{pmatrix}
\cos\left(-R_0d\right) & -\sin\left(-R_0d\right) \\
\sin\left(-R_0d\right) & \cos\left(-R_0d\right)
\end{pmatrix}
=\left(\frac{|\zeta_s|}{|\zeta_{s,0}|}\right)^{-R_0}
\begin{pmatrix}
\cos\left(-R_0\Delta\theta\right) & -\sin\left(-R_0\Delta\theta\right) \\
\sin\left(-R_0\Delta\theta\right) & \cos\left(-R_0\Delta\theta\right)
\end{pmatrix}\,.
\end{equation}
Eqs.~(\ref{AlphaEvolutionLO}) and~(\ref{XiEvolutionLO}) are the main
result of this section because they are the evolution factors in the
real and imaginary direction to be applied to the initial scale PDF.
Considering that eqs.~(\ref{AlphaEvolutionLO})
and~(\ref{XiEvolutionLO}) have exactly the same form they can combined
in one single evolution factor with the following compact result:
\begin{equation}\label{SolutionLOGlobal}
\mathbf{F}(N,\zeta_s) = \mathbf{F}(N,\zeta_{s,0})
\left(\frac{|\zeta_s|}{|\zeta_{s,0}|}\right)^{-R_0}
\begin{pmatrix}
\cos\left[-R_0(\theta-\theta_0)\right] & -\sin\left[-R_0(\theta-\theta_0)\right] \\
\sin\left[-R_0(\theta-\theta_0)\right] & \cos\left[-R_0(\theta-\theta_0)\right]
\end{pmatrix}\,,
\end{equation}
where $|\zeta_s|$ and $|\zeta_{s,0}|$ are the absolute value of the
initial and final values of the complex coupling while $\theta$ and
$\theta_0$ are the respective phases such that:
\begin{equation}
\zeta_s=|\zeta_s|\exp(i\theta)\quad\mbox{and}\quad \zeta_{s,0}=|\zeta_{s,0}|\exp(i\theta_0)\,,
\end{equation}
which is equivalent to:
\begin{equation}\label{SolutionLOGlobal}
{F}(N,\zeta_s) = {F}(N,\zeta_{s,0})
\left(\frac{|\zeta_s|}{|\zeta_{s,0}|}\right)^{-R_0}e^{-iR_0(\theta-\theta_0)}={F}(N,\zeta_{s,0})\left[-R_0\ln\left(\frac{\zeta_s}{\zeta_{s,0}}\right)\right]
= \left(\frac{\zeta_s}{\zeta_{s,0}}\right)^{-R_0}\,,
\end{equation}
which is the straight LO solution of the Mellin version of
eq.~(\ref{DGLAPalphasComplex}). We can thus deduce that, also beyond
LO, eq.~(\ref{DGLAPalphasComplex}) can be solved using the standard
techniques.

Since the best way to solve eq.~(\ref{DGLAPalphasComplex}) is in $N$
(Mellin) space, under the condition that the $x$-space PDF is also
complex the numerical inversion algorithm from $N$ to $x$ space needs
to be adapted. 

The inverse Mellin transformationis defined as:
\begin{equation}\label{MellinInverse}
F(x) = \frac{1}{2\pi i}\int_{c-i\infty}^{c+i\infty} dN\,x^{-N}F(N)\,,
\end{equation}
where the real number $c$ has to be such that the integral
$\int_0^1dx\,x^{c-1}F(x)$ is absolutely convergent. Hence $c$ has to
lie to the right of the rightmost singularity of $F(N)$ in the complex
plane.

Under the assumption that $F(N)$ is an analytical (or holomorphic)
function, the Cauchy theorem states that one can deform the
integration path in a continuous way without changing the result of
the integral, provided that no pole of the function $F(N)$ is crossed
during the deformation. This allows us to cleverly choose a different
path that makes the solution of the integral in
eq.~(\ref{MellinInverse}) easy to implement in a numerical code. A
possible choice is the so-called Talbot path $\mathcal{C}_T$, such
that eq.~(\ref{MellinInverse}) is equivalent to:
\begin{equation}\label{MellinInverseTalbot}
F(x) = \frac{1}{2\pi i}\int_{\mathcal{C}_T} dN\,x^{-N}F(N)\,,
\end{equation}
where:
\begin{equation}\label{TalbotPathPar}
\mathcal{C}_T:\left\lbrace N(\theta)=r\theta(\cot\theta+i);\,\theta\in(-\pi,+\pi) \right\rbrace
\end{equation}
being $r$ a parameter possibly depending only on $x$. Performing a
variable substitution in eq.~(\ref{MellinInverseTalbot}) using
eq.~(\ref{TalbotPathPar}) and defining $t\equiv-\ln x$, one gets:
\begin{equation}\label{TalbotPathParDer1}
\begin{array}{rcl}
 F(x) &=& \displaystyle \frac{1}{2\pi i}\int_{-\pi}^{+\pi}d\theta\,\frac{dN(\theta)}{d\theta}\,e^{t\,N(\theta)}{F}(N(\theta))\\
\\
      &=& \displaystyle \frac{r}{2\pi}\int_{-\pi}^{+\pi}d\theta\left[1+i\sigma(\theta)\right]e^{t\,N(\theta)}{F}(N(\theta))\,,
\end{array}
\end{equation}
with:
\begin{equation}\label{TalbotPathParDer}
\sigma(\theta) = \theta + \cot\theta(\theta\cot\theta-1)\,.
\end{equation}
Usually the computation of the integral in the r.h.s. of
eq.~(\ref{TalbotPathParDer1}) is usually performed numerically using
the fact that $F(N)$ is the Mellin tranform of a real function and
thus, under this assumption, $F(N^*)=F^*(N)$. In the case we are
considering here, $i.e.$ also the $x$-space PDF $F(x)$ is a complex
function, this assumption can no longer be used. Defining:
\begin{equation}
G(\theta) = \frac{r}{2\pi}\left[1+i\sigma(\theta)\right]e^{t\,N(\theta)}{F}(N(\theta))\,,
\end{equation}
and using the trapezoidal method to solve the integral in
eq.~(\ref{TalbotPathParDer1}) one gets:
\begin{equation}\label{trapez}
 F(x) = \delta \left[\sum_{k=1}^{M-1}G(-\pi+k\delta)+\frac{G(-\pi)+G(\pi)}2\right]\,,
\end{equation}
where $M$ is the number of equal intervals in which the integration
range is divided and:
\begin{equation}
\delta =\frac{2\pi}{M}
\end{equation}
is essentially the width of the single interval. One can show that the
boundary terms $G(-\pi$) and $G(\pi)$ in eq.~(\ref{trapez}) vanish and
thus, defining $\theta_k=-\pi+k\delta$, it becomes:
\begin{equation}\label{trapez1}
 F(x) = \frac{r}{M}\sum_{k=1}^{M-1}\left[1+i\sigma(\theta_k)\right]e^{t\,N(\theta_k)}{F}(N(\theta_k))\,.
\end{equation}


From eq.~(\ref{TalbotPathPar}) one can easily see that
$N(-\theta) = N^*(\theta)$, while from eq.~(\ref{TalbotPathParDer}) it
is evident that $\sigma(\theta)$ is an odd function of
$\theta$. Moreover, $F(N)$ being the Mellin transform of a real
function, one automatically has that $F(N^*)=F^*(N)$. Using this
information, one gets:
\begin{equation}\label{TalbotFinalIntegral}
\begin{array}{rcl}
 F(x) &=& \displaystyle \frac{r}{\pi}\int_{0}^{+\pi}d\theta\,\mbox{Re}\left\{\left[1+i\sigma(\theta)\right]e^{t\,N(\theta)}{F}(N(\theta))\right\}\,,
\end{array}
\end{equation}
where $\mbox{Re}\{\,\dots\}$ is the real part of its
argument. Finally, eq.~(\ref{TalbotFinalIntegral}) can be solved
numerically using, for instance, the trapezoidal approximation.




At this point, in order to be able to implement in numerical code, it
is necessary to distinguish between non-slinglet and singlet
distributions. In the case of the non-singlet distributions
eqs.~(\ref{AlphaEvolutionLO}) and~(\ref{XiEvolutionLO}) can be
implemented exactly as they are written because the anomalous
dimension $\gamma^{(0)}$ appearing in
eqs.~(\ref{DGLAPcomplex1LOMellin}) and~(\ref{DGLAPcomplex2LOMellin})
is a singled-valued function. In the singlet case instead
$\gamma^{(0)}$ (and thus $R_0$) is actually a 2 by 2 matrix of
functions and thus, since it appears in the exponential and in the
trigonometric functions, eqs.~(\ref{AlphaEvolutionLO})
and~(\ref{XiEvolutionLO}) need to be treated in a suitable way.







\end{document}
