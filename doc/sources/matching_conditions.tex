\documentclass[10pt,a4paper]{article}
\usepackage{amsmath,amssymb,bm,graphicx,makeidx,subfigure}
\usepackage[italian,english]{babel}
\usepackage[center,small]{caption}[2007/01/07]
\usepackage{fancyhdr}

\oddsidemargin = 12pt
\topmargin = 0pt
\textwidth = 440pt
\textheight = 650pt

\makeindex

\begin{document}

\tableofcontents
\newpage

\section{PDFs Matching Conditions}

If the (Zero Mass) Variable Flavour Number Scheme (ZM$-$VFNS) at NNLO is considered, matching conditions for the PDFs and the coupling constant at the heavy quarks thresholds ($m_c^2$, $m_b^2$ and $m_t^2$) must be implemented. This is due to the fact that we are working with an ``effective theory'' where before a certain threshold, say $m_h^2$, the heavy quark flavour $h$ is treated as infinitely massive, while after the crossing of the same threshold the same flavour is treated as massless. This results in discontinuities of PDFs and coupling constant in correspondence of the thresholds. Such discontinuities can be evaluated in perturbation theory, and in particular one can see that they start at NNLO, so that PDFs and $a_s$ are continous at LO and NLO \cite{Buza:1996wv}.

\vspace{5pt}
The discontinuity of the PDF $l$ (in the Mellin space) of a light quark(\footnote{Note that the light quarks run from $1$ to $N_f$.}) just beyond of the threshold $m_h^2$($=m_c^2,m_b^2,m_t^2$), where the effective flavour number passes from $N_f$ to $N_f+1$, is given as a function of the same PDF just before the threshold by the following relation \cite{Vogt:2004ns}:
\begin{equation}
l^{(N_f+1)}(N,m_h^2)=[1+a_s^2(m_h^2)A_{qq,h}^{N\!S,(2)}(N)]l^{(N_f)}(N,m_h^2)\,.
\end{equation}
with $l=u,\overline{u},d,\overline{d},\dots$, while the gluon distribution function is given by:
\begin{equation}\label{gluon}
\begin{array}{rcl}
\displaystyle g^{(N_f+1)}(N,m_h^2)&=&[1+a_s^2(m_h^2)A_{gg,h}^{S,(2)}(N)]g^{(N_f)}(N,m_h^2)+\\
\\
\displaystyle & &a_s^2(m_h^2)A^{S,(2)}_{gq,h}(N)\Sigma^{(N_f)}(N,m^2_h)
\end{array}
\end{equation}
and, in the end, the sum of heavy quark $h$ and its anti-quark $\overline{h}$, which are going to be produced after the threshold $m_h^2$, is:
\begin{equation}
(h^{(N_f+1)}+\overline{h}^{(N_f+1)})(N,m_h^2)=a_s^2(m_h^2)[\tilde{A}^{S,(2)}_{hq}(N)\Sigma^{(N_f)}(N,m_h^2)+\tilde{A}^{S,(2)}_{hg}(N)g^{(N_f)}(N,m_h^2)]\,.
\end{equation}
Of course, we have $h=\overline{h}$. 

Now, since:
\begin{equation}
\Sigma^{(N_f+1)}=\sum_{l=1}^{N_f}(l^{(N_f+1)}+\overline{l}^{(N_f+1)})+(h^{(N_f+1)}+\overline{h}^{(N_f+1)})
\end{equation}
we find that the matching condition for the singlet is:
\begin{equation}\label{singlet}
\begin{array}{c}
\displaystyle\Sigma^{(N_f+1)}(N,m_h^2)=[1+a_s^2(m_h^2)A_{qq,h}^{N\!S,(2)}(N)]\Sigma^{(N_f)}(N,m_h^2)+\\
\\
\displaystyle a_s^2(m_h^2)[\tilde{A}^{S,(2)}_{hq}(N)\Sigma^{(N_f)}(N,m_h^2)+\tilde{A}^{S,(2)}_{hg}(N)g^{(N_f)}(N,m_h^2)]
\end{array}
\end{equation}
so that, from eqs. (\ref{gluon}) and (\ref{singlet}):
\begin{equation}\label{couple1}
\begin{array}{c}
\displaystyle {\Sigma^{(N_f+1)} \choose g^{(N_f+1)}}=\begin{pmatrix}1+a_s^2[A_{qq,h}^{N\!S,(2)}+\tilde{A}^{S,(2)}_{hq}] & a_s^2\tilde{A}^{S,(2)}_{hg}\\
a_s^2A^{S,(2)}_{gq,h} & 1+a_s^2A_{gg,h}^{S,(2)}\end{pmatrix}{\Sigma^{(N_f)} \choose g^{(N_f)}}\\
\\
\displaystyle =\left[\begin{pmatrix} 1 & 0 \\ 0 & 1\end{pmatrix}+a_s^2A_{qq,h}^{N\!S,(2)}\begin{pmatrix} 1 & 0 \\ 0 & 0\end{pmatrix}+a_s^2\begin{pmatrix} \tilde{A}^{S,(2)}_{hq} & \tilde{A}^{S,(2)}_{hg} \\A^{S,(2)}_{gq,h} & A_{gg,h}^{S,(2)}\end{pmatrix}\right]{\Sigma^{(N_f)} \choose g^{(N_f)}}
\end{array}
\end{equation}
where we have obmitted all the dependencies. So we have abtained the matching conditions for the singlet and the gluon distribution functions.

Now we consider the other distributions. The valence distribution $V$ for $N_f+1$ active (light) flavours just beyond the threshold $m_h^2$ is defined as:
\begin{equation}
V^{(N_f+1)}=\sum_{l=1}^{N_f}(l^{(N_f+1)}-\overline{l}^{(N_f+1)})+(h^{(N_f+1)}-\overline{h}^{(N_f+1)})
\end{equation}
but, since $h=\overline{h}$, the last term vanish and we are left with:
\begin{equation}\label{valence}
\begin{array}{c}
\displaystyle V^{(N_f+1)}=\sum_{l=1}^{N_f}(l^{(N_f+1)}-\overline{l}^{(N_f+1)})=\\
\\
\displaystyle [1+a_s^2A_{qq,h}^{N\!S,(2)}]\sum_{l=1}^{N_f}(l^{(N_f)}-\overline{l}^{(N_f)})=[1+a_s^2A_{qq,h}^{N\!S,(2)}]V^{(N_f)}\,.
\end{array}
\end{equation}
So, this is the matching condition for the valence distribution $V$.

Now we consider the valence distribution $V_3$ and $V_8$ which are both composed only by light quarks, namely:
\begin{equation}
V_3=(u-\overline{u})-(d-\overline{d})\quad\mbox{and}\quad V_8=(u-\overline{u})+(d-\overline{d})-2(s-\overline{s})
\end{equation}
so that, in these cases, the matching condistions work as in the case of $V$, i.e.:
\begin{equation}\label{v38}
V_{3,8}^{(N_f+1)}=[1+a_s^2A_{qq,h}^{N\!S,(2)}]V^{(N_f)}_{3,8}\,.
\end{equation}

The same holds for $T_{3}$ and $T_8$, which are defined as:
\begin{equation}
T_3=(u+\overline{u})-(d+\overline{d})\quad\mbox{and}\quad V_8=(u+\overline{u})+(d+\overline{d})-2(s+\overline{s})
\end{equation}
so:
\begin{equation}\label{t38}
T_{3,8}^{(N_f+1)}=[1+a_s^2A_{qq,h}^{N\!S,(2)}]T^{(N_f)}_{3,8}\,.
\end{equation}

The remanining valence distribution $V_{15}$, $V_{24}$ and $V_{35}$ are defined as:
\begin{equation}
\begin{array}{l}
V_{15}=(u-\overline{u})+(d-\overline{d})+(s-\overline{s})-3(c-\overline{c})\\
V_{24}=(u-\overline{u})+(d-\overline{d})+(s-\overline{s})+(c-\overline{c})-4(b-\overline{b})\\
V_{35}=(u-\overline{u})+(d-\overline{d})+(s-\overline{s})+(c-\overline{c})+(b-\overline{b})-5(t-\overline{t})
\end{array}
\end{equation}
and since in each one of them the heavy quarks appear always as difference between quark and anti-quark, they cancel excatly. For example, at the $m_b^2$ threshold, $V_{15}$ is entirely composed by light quarks so there is no problem, while $V_{24}$ and $V_{35}$ have also a $b$-quark contribution, given by $-4(b-\overline{b})$ and $(b-\overline{b})$ respectively (of course, the $t(\overline{t})$ distribubution is zero). Anyway, this terms give no matching condition since the $b$ contribution is exactly equal to the $\overline{b}$ contribution, so that they cancel. So:
\begin{equation}\label{v152435}
V_{15,24,35}^{(N_f+1)}=[1+a_s^2A_{qq,h}^{N\!S,(2)}]V^{(N_f)}_{15,24,35}\,.
\end{equation}

In the end, to deal with $T_{15}$, $T_{24}$ and $T_{35}$, we have to specify the threshold. Indeed, in these cases the heavy quark contribution does not cancel. Their definition is:
\begin{equation}
\begin{array}{l}
T_{15}=(u+\overline{u})+(d+\overline{d})+(s+\overline{s})-3(c+\overline{c})\\
T_{24}=(u+\overline{u})+(d+\overline{d})+(s+\overline{s})+(c+\overline{c})-4(b+\overline{b})\\
T_{35}=(u+\overline{u})+(d+\overline{d})+(s+\overline{s})+(c+\overline{c})+(b+\overline{b})-5(t+\overline{t})\,.
\end{array}
\end{equation}
Just before the threshold $m_c^2$ we have only 3 active light flavours ($u$, $d$ and $s$), while just beyond $m_c^2$ we have 4 active flavours and among them the flavour $c$ is considered to be heavy. Of course, we have no $b$ and $t$ contribution (so $T_{24}$ and $T_{35}$ are equal). So the matching conditions are:
\begin{equation}
\begin{array}{c}
\displaystyle T_{15}^{(4)}=[1+a_s^2A_{qq,c}^{N\!S,(2)}]\underbrace{\sum_{l=u,d,s}(l^{(3)}+\overline{l}^{(3)})}_{\Sigma^{(3)}}-3a_s^2[\tilde{A}^{S,(2)}_{cq}\Sigma^{(3)}+\tilde{A}^{S,(2)}_{cg}g^{(3)}]=\\
\\
\displaystyle \begin{pmatrix} 1+a_s^2[A_{qq,c}^{N\!S,(2)}-3\tilde{A}^{S,(2)}_{cq}] & -3a_s^2\tilde{A}^{S,(2)}_{cg}\end{pmatrix}{\Sigma^{(3)} \choose g^{(3)}}
\end{array}
\end{equation}
while:
\begin{equation}
T_{24,35}^{(4)}=\begin{pmatrix} 1+a_s^2[A_{qq,c}^{N\!S,(2)}+\tilde{A}^{S,(2)}_{cq}] & a_s^2\tilde{A}^{S,(2)}_{cg}\end{pmatrix}{\Sigma^{(3)} \choose g^{(3)}}\,.
\end{equation}
We can put the above relation in a matricial form:
\begin{equation}\label{pippo1}
\begin{pmatrix} T_{15}^{(4)} \\ T_{24}^{(4)} \\ T_{35}^{(4)} \end{pmatrix} = \begin{pmatrix}  1+a_s^2[A_{qq,c}^{N\!S,(2)}-3\tilde{A}^{S,(2)}_{cq}] & -3a_s^2\tilde{A}^{S,(2)}_{cg}\\  
1+a_s^2[A_{qq,c}^{N\!S,(2)}+\tilde{A}^{S,(2)}_{cq}] & a_s^2\tilde{A}^{S,(2)}_{cg} \\
1+a_s^2[A_{qq,c}^{N\!S,(2)}+\tilde{A}^{S,(2)}_{cq}] & a_s^2\tilde{A}^{S,(2)}_{cg} \end{pmatrix}{\Sigma^{(3)} \choose g^{(3)}}\,.
\end{equation}
But now it is easy to generalize. At $m_b^2$, $T_{15}$ does not contain, so:
\begin{equation}\label{pippo2}
T_{15}^{(5)}=[1+a_s^2A_{qq,b}^{N\!S,(2)}]T_{15}^{(4)}
\end{equation}
while:
\begin{equation}\label{pippo3}
\begin{pmatrix} T_{24}^{(5)} \\ T_{35}^{(5)} \end{pmatrix} = \begin{pmatrix}  1+a_s^2[A_{qq,b}^{N\!S,(2)}-4\tilde{A}^{S,(2)}_{bq}] & -4a_s^2\tilde{A}^{S,(2)}_{bg}\\  
1+a_s^2[A_{qq,b}^{N\!S,(2)}+\tilde{A}^{S,(2)}_{bq}] & a_s^2\tilde{A}^{S,(2)}_{bg} \end{pmatrix}{\Sigma^{(4)} \choose g^{(4)}}\,.
\end{equation}
Finally, at $m_t^2$ we have:
\begin{equation}\label{pippo4}
\begin{array}{l}
\displaystyle T_{15}^{(6)}=[1+a_s^2A_{qq,t}^{N\!S,(2)}]T_{15}^{(5)}\\
\\
\displaystyle T_{24}^{(6)}=[1+a_s^2A_{qq,t}^{N\!S,(2)}]T_{25}^{(5)}
\end{array}
\end{equation}
and:
\begin{equation}\label{pippo5}
T_{35}^{(6)} = \begin{pmatrix}  1+a_s^2[A_{qq,t}^{N\!S,(2)}-5\tilde{A}^{S,(2)}_{tq}] & -5a_s^2\tilde{A}^{S,(2)}_{tg}\end{pmatrix}{\Sigma^{(5)} \choose g^{(5)}}\,.
\end{equation}

An explicit calculation for the coefficients $A^{(2)}$ in the $x$-space can be found in [hep-ph/9612398]. Anyhow, that calculation is performed more generally in the case $m_h^2\neq\mu_F^2$. This results in extra-terms proportional to $\ln(m_h^2/\mu_F^2)$, which vanish if, as we do, one takes the factorization scale $\mu^2$ coinciding with the scale of the process $Q^2$. Moreover, in that case also NLO ($\propto a_s$) appear in the maching conditions.

To summarize the PDF matching conditions at the threshold $m_h^2$, we have that:
\begin{itemize}
\item singlet and gluon couple as follows:
\begin{equation}\label{couple}
{\Sigma^{(N_f+1)} \choose g^{(N_f+1)}}=\left[\begin{pmatrix} 1 & 0 \\ 0 & 1\end{pmatrix}+a_s^2A_{qq,h}^{N\!S,(2)}\begin{pmatrix} 1 & 0 \\ 0 & 0\end{pmatrix}+a_s^2\begin{pmatrix} \tilde{A}^{S,(2)}_{hq} & \tilde{A}^{S,(2)}_{hg} \\A^{S,(2)}_{gq,h} & A_{gg,h}^{S,(2)}\end{pmatrix}\right]{\Sigma^{(N_f)} \choose g^{(N_f)}}
\end{equation}
\item from eqs. (\ref{valence}), (\ref{v38}), (\ref{v152435}) and (\ref{t38}), one can see that $V$, $V_{3,8,\dots,35}$ and $T_{3,8}$ behave in the same way, i.e.:
\begin{equation}\label{ciao}
P^{(N_f+1)}=[1+a_s^2A_{qq,h}^{N\!S,(2)}]P^{(N_f)}\quad\mbox{with}\quad P=V,V_3,\dots,V_{35},T_3,T_8
\end{equation}
\item $T_{15}$, $T_{24}$ and $T_{35}$ have different matching conditions depending on the threshold. In particular: for $m_h^2=m_c^2$ they are given by eq. (\ref{pippo1}), for $m_h^2=m_b^2$ they are given by eqs. (\ref{pippo2}) and (\ref{pippo3}) and for $m_h^2=m_t^2$ they are given by eqs. (\ref{pippo4}) and (\ref{pippo5})
\end{itemize}

In the following Sections we will discuss how to write the evolution kernels in the presence of the matching conditions. We will explicitly consider only the forward evolution, i.e. the final scale $Q^2$ greater than the initial one $Q_0^2$. Anyway the backward evolution ($Q_0^2>Q^2$) can be easly obtained from the forward one. In fact, given the evolution kernel $\Gamma$, the following relation holds:
\begin{equation}
\Gamma(Q^2,Q_0^2)\Gamma(Q_0^2,Q^2)=1\quad\Longrightarrow\quad\Gamma(Q^2,Q_0^2)=\Gamma^{-1}(Q_0^2,Q^2).
\end{equation}
so, if $Q^2>Q_0^2$ the code computes directly $\Gamma(Q^2,Q_0^2)$, else if $Q^2_0>Q^2$ the code evaluates first the forward evolution $\Gamma(Q_0^2,Q^2)$ and then, to get $\Gamma(Q^2,Q_0^2)$, it calculates $\Gamma^{-1}(Q_0^2,Q^2)$.

\subsection{Matching Conditions on the Evolution Kernels: 0 Thresholds Crossing}

Before to discuss the crossing of the thresholds, it would be useful to write down the evolution kernels in the ``trivial'' situation of no threshold crossing. There are 4 particular cases: 1) $Q_0^2<Q^2<m_c^2$ with $N_f=3$ active flavours, 2) $m_c^2<Q_0^2<Q^2<m_b^2$ with $N_f=4$ active flavours, 3) $m_b^2<Q_0^2<Q^2<m_t^2$ with $N_f=5$ active flavours and 4) $m_t^2<Q_0^2<Q^2$ with $N_f=6$ active flavours, wihich do not need the introduction of the matching conditions. So, in the following Subsections we will write down the evolution of the whole PDF set the these cases.

\subsubsection{$Q_0^2<Q^2<m_c^2$}
\begin{itemize}
\item \textbf{Singlet} and \textbf{gluon}:
\begin{equation}
{\Sigma^{(3)}(Q^2) \choose g^{(3)}(Q^2)} =\underbrace{\begin{pmatrix} \Gamma_{qq}& \Gamma_{qg} \\ \Gamma_{gq}& \Gamma_{gg}\end{pmatrix}}_{(Q^2,Q_0^2)}{\Sigma^{(3)}(Q_0^2) \choose g^{(3)}(Q_0^2)}
\end{equation}
\item $\mathbf{V}$:
\begin{equation}
V^{(3)}(Q^2)=\Gamma^{v}(Q^2,Q_0^2)V^{(3)}(Q^2_0)
\end{equation}
\item $\mathbf{V_3}$ and $\mathbf{V_8}$:
\begin{equation}
V^{(3)}_{3,8}(Q^2)=\Gamma^{-}(Q^2,Q_0^2)V^{(3)}_{3,8}(Q^2_0)
\end{equation}
\item $\mathbf{V_{15}}$, $\mathbf{V_{24}}$ and $\mathbf{V_{35}}$:
\begin{equation}
V_{15,24,35}^{(3)}(Q^2)=\Gamma^{v}(Q^2,Q^2_0)V^{(3)}(Q_0^2)
\end{equation}
\item $\mathbf{T_3}$ and $\mathbf{T_8}$:
\begin{equation}
T^{(3)}_{3,8}(Q^2)=\Gamma^{+}(Q^2,Q_0^2)T^{(3)}_{3,8}(Q^2_0)
\end{equation}
\item $\mathbf{T_{15}}$, $\mathbf{T_{24}}$ and $\mathbf{T_{35}}$:
\begin{equation}
T_{15,24,35}^{(3)}(Q^2) = \underbrace{\begin{pmatrix} \Gamma_{qq} & \Gamma_{qg}\end{pmatrix}}_{(Q^2,Q_0^2)}{\Sigma^{(3)}(Q_0^2) \choose g^{(3)}(Q_0^2)}
\end{equation}
\end{itemize}

\subsubsection{$m_c^2<Q_0^2<Q^2<m_b^2$}
\begin{itemize}
\item \textbf{Singlet} and \textbf{gluon}:
\begin{equation}
{\Sigma^{(4)}(Q^2) \choose g^{(4)}(Q^2)} =\underbrace{\begin{pmatrix} \Gamma_{qq}& \Gamma_{qg} \\ \Gamma_{gq}& \Gamma_{gg}\end{pmatrix}}_{(Q^2,Q_0^2)}{\Sigma^{(4)}(Q_0^2) \choose g^{(4)}(Q_0^2)}
\end{equation}
\item $\mathbf{V}$:
\begin{equation}
V^{(4)}(Q^2)=\Gamma^{v}(Q^2,Q_0^2)V^{(4)}(Q^2_0)
\end{equation}
\item $\mathbf{V_3}$, $\mathbf{V_8}$ and $\mathbf{V_{15}}$:
\begin{equation}
V^{(4)}_{3,8,15}(Q^2)=\Gamma^{-}(Q^2,Q_0^2)V^{(4)}_{3,8,15}(Q^2_0)
\end{equation}
\item $\mathbf{V_{24}}$ and $\mathbf{V_{35}}$:
\begin{equation}
V_{24,35}^{(4)}(Q^2)=\Gamma^{v}(Q^2,Q^2_0)V^{(4)}(Q_0^2)
\end{equation}
\item $\mathbf{T_3}$, $\mathbf{T_8}$ and $\mathbf{T_{15}}$:
\begin{equation}
T^{(4)}_{3,8,15}(Q^2)=\Gamma^{+}(Q^2,Q_0^2)T^{(4)}_{3,8,15}(Q^2_0)
\end{equation}
\item $\mathbf{T_{24}}$ and $\mathbf{T_{35}}$:
\begin{equation}
T_{24,35}^{(4)}(Q^2) = \underbrace{\begin{pmatrix} \Gamma_{qq} & \Gamma_{qg}\end{pmatrix}}_{(Q^2,Q_0^2)}{\Sigma^{(4)}(Q_0^2) \choose g^{(4)}(Q_0^2)}
\end{equation}
\end{itemize}

\subsubsection{$m_b^2<Q_0^2<Q^2<m_t^2$}
\begin{itemize}
\item \textbf{Singlet} and \textbf{gluon}:
\begin{equation}
{\Sigma^{(5)}(Q^2) \choose g^{(5)}(Q^2)} =\underbrace{\begin{pmatrix} \Gamma_{qq}& \Gamma_{qg} \\ \Gamma_{gq}& \Gamma_{gg}\end{pmatrix}}_{(Q^2,Q_0^2)}{\Sigma^{(5)}(Q_0^2) \choose g^{(5)}(Q_0^2)}
\end{equation}
\item $\mathbf{V}$:
\begin{equation}
V^{(5)}(Q^2)=\Gamma^{v}(Q^2,Q_0^2)V^{(5)}(Q^2_0)
\end{equation}
\item $\mathbf{V_3}$, $\mathbf{V_8}$, $\mathbf{V_{15}}$ and $\mathbf{V_{24}}$:
\begin{equation}
V^{(5)}_{3,8,15,24}(Q^2)=\Gamma^{-}(Q^2,Q_0^2)V^{(5)}_{3,8,15,24}(Q^2_0)
\end{equation}
\item $\mathbf{V_{35}}$:
\begin{equation}
V_{35}^{(5)}(Q^2)=\Gamma^{v}(Q^2,Q^2_0)V^{(5)}(Q_0^2)
\end{equation}
\item $\mathbf{T_3}$, $\mathbf{T_8}$, $\mathbf{T_{15}}$ and $\mathbf{T_{24}}$:
\begin{equation}
T^{(5)}_{3,8,15,24}(Q^2)=\Gamma^{+}(Q^2,Q_0^2)T^{(5)}_{3,8,15,24}(Q^2_0)
\end{equation}
\item $\mathbf{T_{35}}$:
\begin{equation}
T_{35}^{(5)}(Q^2) = \underbrace{\begin{pmatrix} \Gamma_{qq} & \Gamma_{qg}\end{pmatrix}}_{(Q^2,Q_0^2)}{\Sigma^{(5)}(Q_0^2) \choose g^{(5)}(Q_0^2)}
\end{equation}
\end{itemize}

\subsubsection{$m_t^2<Q_0^2<Q^2$}
\begin{itemize}
\item \textbf{Singlet} and \textbf{gluon}:
\begin{equation}
{\Sigma^{(6)}(Q^2) \choose g^{(6)}(Q^2)} =\underbrace{\begin{pmatrix} \Gamma_{qq}& \Gamma_{qg} \\ \Gamma_{gq}& \Gamma_{gg}\end{pmatrix}}_{(Q^2,Q_0^2)}{\Sigma^{(6)}(Q_0^2) \choose g^{(6)}(Q_0^2)}
\end{equation}
\item $\mathbf{V}$:
\begin{equation}
V^{(6)}(Q^2)=\Gamma^{v}(Q^2,Q_0^2)V^{(6)}(Q^2_0)
\end{equation}
\item $\mathbf{V_3}$, $\mathbf{V_8}$, $\mathbf{V_{15}}$, $\mathbf{V_{24}}$ and $\mathbf{V_{35}}$:
\begin{equation}
V^{(6)}_{3,8,15,24,35}(Q^2)=\Gamma^{-}(Q^2,Q_0^2)V^{(6)}_{3,8,15,24,35}(Q^2_0)
\end{equation}
\item $\mathbf{T_3}$, $\mathbf{T_8}$, $\mathbf{T_{15}}$, $\mathbf{T_{24}}$ and $\mathbf{T_{35}}$:
\begin{equation}
T^{(6)}_{3,8,15,24,35}(Q^2)=\Gamma^{+}(Q^2,Q_0^2)T^{(6)}_{3,8,15,24,35}(Q^2_0)
\end{equation}
\end{itemize}


\subsection{Matching Conditions on the Evolution Kernels: 1 Threshold Crossing}

In order to implemet the matching conditions in our code, we will show how to transfer them from the PDFs to the evolution kernerls.

In this Section we suppose that the evolution crosses only one threshold. We will show how the matching conditions on the PDFs modify the form of the evolution kernels in the cases: 1) $Q_0^2<m_c^2\leq Q^2$, 2) $Q_0^2<m_b^2\leq Q^2$ and $Q_0^2<m_c^2\leq Q^2$. 

\subsubsection{$Q_0^2<m_c^2\leq Q^2$}

In order to evolve PDFs from the scale $Q_0^2$ to $Q^2$ passing through the threshold $m_c^2$, we have to: first evolve them from $Q^2_0$ to $m_c^2$, where there are 3 active flavours, then increase the number of active flavour from 3 to 4 by imposing the matching conditions, and in the end evolve the PDFs, now having 4 active flavours, from $m_c^2$ to the scale $Q^2$.

In what follows we will work only in the Mellin space, so we will drop any dependence on $N$.

Let's start with singlet and gluon. We have:
\begin{equation}\label{couple4Qmc}
{\Sigma^{(4)} \choose g^{(4)}}(Q^2) = \begin{pmatrix} \Gamma_{qq} & \Gamma_{qg} \\ \Gamma_{gq}& \Gamma_{gg}\end{pmatrix}(Q^2,m_c^2){\Sigma^{(4)} \choose g^{(4)}}(m_c^2)\,.
\end{equation}
From eq. (\ref{couple}):
\begin{equation}
\displaystyle {\Sigma^{(4)} \choose g^{(4)}}(m_c^2)=\left[\begin{pmatrix} 1 & 0 \\ 0 & 1\end{pmatrix}+a_s^2(m_c^2)A_{qq,c}^{N\!S,(2)}\begin{pmatrix} 1 & 0 \\ 0 & 0\end{pmatrix}+a_s^2(m_c^2)\begin{pmatrix} \tilde{A}^{S,(2)}_{cq} & \tilde{A}^{S,(2)}_{cg} \\A^{S,(2)}_{gq,c} & A_{gg,c}^{S,(2)}\end{pmatrix}\right]{\Sigma^{(3)} \choose g^{(3)}}(m_c^2)
\end{equation}
now, substituting the above relation into the eq. (\ref{couple4Qmc}), we get:
\begin{equation}\label{couple43Qmc}
\begin{array}{l}
\displaystyle {\Sigma^{(4)} \choose g^{(4)}}(Q^2) = \begin{pmatrix} \Gamma_{qq} & \Gamma_{qg} \\ \Gamma_{gq}& \Gamma_{gg}\end{pmatrix}(Q^2,m_c^2)\Bigg[\begin{pmatrix} 1 & 0 \\ 0 & 1\end{pmatrix}+a_s^2(m_c^2)A_{qq,c}^{N\!S,(2)}\begin{pmatrix} 1 & 0 \\ 0 & 0\end{pmatrix}+\\
\\
\hspace{180pt}\displaystyle a_s^2(m_c^2)\begin{pmatrix} \tilde{A}^{S,(2)}_{cq} & \tilde{A}^{S,(2)}_{cg} \\A^{S,(2)}_{gq,c} & A_{gg,c}^{S,(2)}\end{pmatrix}\Bigg]{\Sigma^{(3)} \choose g^{(3)}}(m_c^2)\,.
\end{array}
\end{equation}
But:
\begin{equation}\label{couple3mcQ0}
{\Sigma^{(3)} \choose g^{(3)}}(m_c^2) = \begin{pmatrix} \Gamma_{qq} & \Gamma_{qg} \\ \Gamma_{gq}& \Gamma_{gg}\end{pmatrix}(m_c^2,Q_0^2){\Sigma^{(3)} \choose g^{(3)}}(Q_0^2)\,.
\end{equation}
So, in the end:
\begin{equation}\label{couple43QQ0}
\begin{array}{l}
\displaystyle {\Sigma^{(4)} \choose g^{(4)}}(Q^2) = \Bigg\{\begin{pmatrix} \Gamma_{qq} & \Gamma_{qg} \\ \Gamma_{gq}& \Gamma_{gg}\end{pmatrix}(Q^2,m_c^2)\Bigg[\begin{pmatrix} 1 & 0 \\ 0 & 1\end{pmatrix}+a_s^2(m_c^2)A_{qq,c}^{N\!S,(2)}\begin{pmatrix} 1 & 0 \\ 0 & 0\end{pmatrix}+\\
\\
\hspace{70pt}\displaystyle a_s^2(m_c^2)\begin{pmatrix} \tilde{A}^{S,(2)}_{cq} & \tilde{A}^{S,(2)}_{cg} \\A^{S,(2)}_{gq,c} & A_{gg,c}^{S,(2)}\end{pmatrix}\Bigg]\begin{pmatrix} \Gamma_{qq} & \Gamma_{qg} \\ \Gamma_{gq}& \Gamma_{gg}\end{pmatrix}(m_c^2,Q_0^2)\Bigg\}{\Sigma^{(3)} \choose g^{(3)}}(Q_0^2)
\end{array}\,.
\end{equation}

Now we consider the distrubutions $V$, $V_{3,8}$ and $T_{3,8}$ which evolve re\-spe\-cti\-ve\-ly through $\Gamma^v$, $\Gamma^-$ and $\Gamma^+$, but which obey the same matching conditions. So:
\begin{equation}
P^{(4)}(Q^2)=\Gamma^{(P)}(Q^2,m_c^2)P^{(4)}(m_c^2)
\end{equation}
so that:
\begin{equation}
P=\left\{
\begin{array}{ll}
\displaystyle V &\rightarrow \Gamma^{(P)}=\Gamma^v\\
\displaystyle V_{3,8} &\rightarrow \Gamma^{(P)}=\Gamma^-\\
\displaystyle T_{3,8} &\rightarrow \Gamma^{(P)}=\Gamma^+
\end{array}\right.
\end{equation}
but:
\begin{equation}
P^{(4)}(m_c^2)=[1+a_s^2(m_c^2)A_{qq,c}^{N\!S,(2)}]P^{(3)}(m_c^2)
\end{equation}
and:
\begin{equation}
P^{(3)}(m_c^2)=\Gamma^{(P)}(m_c^2,Q_0^2)P^{(3)}(Q^2_0)
\end{equation}
so that:
\begin{equation}
P^{(4)}(Q^2)=\left\{\Gamma^{(P)}(Q^2,m_c^2)[1+a_s^2(m_c^2)A_{qq,c}^{N\!S,(2)}]\Gamma^{(P)}(m_c^2,Q_0^2)\right\}P^{(3)}(Q^2_0)
\end{equation}

Now we consider $V_{15}$, which before $m_c^2$ evolves as:
\begin{equation}
V_{15}^{(3)}(m_c^2)=\Gamma^{v}(m_c^2,Q_0^2)V^{(3)}(Q_0^2)
\end{equation}
while after $m_c^2$ it evolves as:
\begin{equation}
V_{15}^{(4)}(Q^2)=\Gamma^{-}(Q_0^2,m_c^2)V^{(4)}_{15}(m_c^2)\,.
\end{equation}
From eq. (\ref{ciao}), we find that the matching condition at $m_c^2$ is:
\begin{equation}
V^{(4)}_{15}(m_c^2)=[1+a_s^2(m_c^2)A_{qq,c}^{N\!S,(2)}] V^{(3)}_{15}(m_c^2)
\end{equation}
so:
\begin{equation}
V_{15}^{(4)}(Q^2)=\left\{\Gamma^{-}(Q^2,m_c^2)[1+a_s^2(m_c^2)A_{qq,c}^{N\!S,(2)}]\Gamma^{v}(m_c^2,Q_0^2)\right\}V^{(3)}(Q_0^2)
\end{equation}

Instead, for $Q_0^2<m_c^2\leq Q^2$, both $V_{24}$ and $V_{35}$ evolve as: 
\begin{equation}
V_{24,35}^{(4)}(Q^2)=\left\{\Gamma^{v}(Q^2,m_c^2)[1+a_s^2(m_c^2)A_{qq,c}^{N\!S,(2)}]\Gamma^{v}(m_c^2,Q_0^2)\right\}V^{(3)}(Q_0^2)
\end{equation}

Now, we consider $T_{15}$. After $m_c^2$, it evolves as:
\begin{equation}
T_{15}^{(4)}(Q^2)=\Gamma^{+}(Q^2,m_c^2)T^{(4)}_{15}(m_c^2)\,.
\end{equation}
This time the matching condition is given by the first line of eq. (\ref{pippo1}):
\begin{equation}
T_{15}^{(4)}(m_c^2)= \begin{pmatrix}  1+a_s^2(m_c^2)[A_{qq,c}^{N\!S,(2)}-3\tilde{A}^{S,(2)}_{cq}] & -3a_s^2(m_c^2)\tilde{A}^{S,(2)}_{cg} \end{pmatrix}{\Sigma^{(3)}(m_c^2) \choose g^{(3)}(m_c^2)}\,.
\end{equation}
But:
\begin{equation}
{\Sigma^{(3)}(m_c^2) \choose g^{(3)}(m_c^2) }=\begin{pmatrix} \Gamma_{qq}(m_c^2,Q_0^2) & \Gamma_{qg}(m_c^2,Q_0^2)\\ \Gamma_{gq}(m_c^2,Q_0^2) & \Gamma_{gg}(m_c^2,Q_0^2)\end{pmatrix}{\Sigma^{(3)}(Q_0^2) \choose g^{(3)}(Q_0^2)}
\end{equation}
In the end, one finds that:
\begin{equation}
\begin{array}{rcl}
\displaystyle T_{15}^{(4)}(Q^2)&=&\Bigg\{\Gamma^{+}(Q^2,m_c^2)\begin{pmatrix} 1+a_s^2(m_c^2)[A_{qq,c}^{N\!S,(2)}-3\tilde{A}^{S,(2)}_{cq}] & -3a_s^2(m_c^2)\tilde{A}^{S,(2)}_{cg}\end{pmatrix}\times\\
\\
& &\displaystyle \begin{pmatrix}\Gamma_{qq}(m_c^2,Q_0^2) & \Gamma_{qg}(m_c^2,Q_0^2)\\ \Gamma_{gq}(m_c^2,Q_0^2) & \Gamma_{gg}(m_c^2,Q_0^2)\end{pmatrix}\Bigg\}{\Sigma^{(3)}(Q_0^2) \choose g^{(3)}(Q_0^2)}
\end{array}
\end{equation} 

Now we are left only with $T_{24}$ and $T_{35}$, which evolve as the single before and after the threshol, so they evolve exactly as the first line of eq. (\ref{couple43QQ0}), i.e:
\begin{equation}
\begin{array}{l}
\displaystyle T_{24,35}^{(4)}(Q^2)= \Bigg\{\begin{pmatrix} \Gamma_{qq} & \Gamma_{qg} \end{pmatrix}(Q^2,m_c^2)\Bigg[\begin{pmatrix} 1 & 0 \\ 0 & 1\end{pmatrix}+a_s^2(m_c^2)A_{qq,c}^{N\!S,(2)}\begin{pmatrix} 1 & 0 \\ 0 & 0\end{pmatrix}+\\
\\
\hspace{70pt}\displaystyle a_s^2(m_c^2)\begin{pmatrix} \tilde{A}^{S,(2)}_{cq} & \tilde{A}^{S,(2)}_{cg} \\A^{S,(2)}_{gq,c} & A_{gg,c}^{S,(2)}\end{pmatrix}\Bigg]\begin{pmatrix} \Gamma_{qq} & \Gamma_{qg} \\ \Gamma_{gq}& \Gamma_{gg}\end{pmatrix}(m_c^2,Q_0^2)\Bigg\}{\Sigma^{(3)} \choose g^{(3)}}(Q_0^2)
\end{array}\,.
\end{equation}

Now, let us summarize what happens to the evolution kernels, by introducing the matching conditions, if one crosses the $m_c^2$ threshold. We remind that the matching conditions appear only from the NNLO.
\begin{itemize}
\item \textbf{Singlet} and \textbf{gluon}:
\begin{equation}
{\Sigma^{(4)}(Q^2) \choose g^{(4)}(Q^2)} = \Bigg\{\underbrace{\begin{pmatrix} \Gamma_{qq} & \Gamma_{qg} \\ \Gamma_{gq}& \Gamma_{gg}\end{pmatrix}}_{(Q^2,m_c^2)}\begin{pmatrix} M_{11}^c & M_{12}^c \\ M_{21}^c & M_{22}^c\end{pmatrix}\underbrace{\begin{pmatrix} \Gamma_{qq}& \Gamma_{qg} \\ \Gamma_{gq}& \Gamma_{gg}\end{pmatrix}}_{(m_c^2,Q_0^2)}\Bigg\}{\Sigma^{(3)}(Q_0^2) \choose g^{(3)}(Q_0^2)}
\end{equation}
where:
\begin{equation}
\begin{pmatrix} M_{11}^c & M_{12}^c \\ M_{21}^c & M_{22}^c\end{pmatrix}=\begin{pmatrix} 1 & 0 \\ 0 & 1\end{pmatrix}+a_s^2(m_c^2)A_{qq,c}^{N\!S,(2)}\begin{pmatrix} 1 & 0 \\ 0 & 0\end{pmatrix}+a_s^2(m_c^2)\begin{pmatrix} \tilde{A}^{S,(2)}_{cq} & \tilde{A}^{S,(2)}_{cg} \\A^{S,(2)}_{gq,c} & A_{gg,c}^{S,(2)}\end{pmatrix}
\end{equation}
\item $\mathbf{V}$:
\begin{equation}
V^{(4)}(Q^2)=\left\{\Gamma^{v}(Q^2,m_c^2)[1+a_s^2(m_c^2)A_{qq,c}^{N\!S,(2)}]\Gamma^{v}(m_c^2,Q_0^2)\right\}V^{(3)}(Q^2_0)
\end{equation}
\item $\mathbf{V_3}$ and $\mathbf{V_8}$:
\begin{equation}
V^{(4)}_{3,8}(Q^2)=\left\{\Gamma^{-}(Q^2,m_c^2)[1+a_s^2(m_c^2)A_{qq,c}^{N\!S,(2)}]\Gamma^{-}(m_c^2,Q_0^2)\right\}V^{(3)}_{3,8}(Q^2_0)
\end{equation}
\item $\mathbf{V_{15}}$:
\begin{equation}
V_{15}^{(4)}(Q^2)=\left\{\Gamma^{-}(Q^2,m_c^2)[1+a_s^2(m_c^2)A_{qq,c}^{N\!S,(2)}]\Gamma^{v}(m_c^2,Q_0^2)\right\}V^{(3)}(Q_0^2)
\end{equation}
\item $\mathbf{V_{24}}$ and $\mathbf{V_{35}}$:
\begin{equation}
V_{24,35}^{(4)}(Q^2)=\left\{\Gamma^{v}(Q^2,m_c^2)[1+a_s^2(m_c^2)A_{qq,c}^{N\!S,(2)}]\Gamma^{v}(m_c^2,Q_0^2)\right\}V^{(3)}(Q_0^2)
\end{equation}
\item $\mathbf{T_3}$ and $\mathbf{T_8}$:
\begin{equation}
T^{(4)}_{3,8}(Q^2)=\left\{\Gamma^{+}(Q^2,m_c^2)[1+a_s^2(m_c^2)A_{qq,c}^{N\!S,(2)}]\Gamma^{+}(m_c^2,Q_0^2)\right\}T^{(3)}_{3,8}(Q^2_0)
\end{equation}
\item $\mathbf{T_{15}}$:
\begin{equation}
\begin{array}{rcl}
\displaystyle T_{15}^{(4)}(Q^2)&=&\Bigg\{\Gamma^{+}(Q^2,m_c^2)\begin{pmatrix} 1+a_s^2(m_c^2)[A_{qq,c}^{N\!S,(2)}-3\tilde{A}^{S,(2)}_{cq}] & -3a_s^2(m_c^2)\tilde{A}^{S,(2)}_{cg}\end{pmatrix}\times\\
\\
 & & \displaystyle \underbrace{\begin{pmatrix}\Gamma_{qq}& \Gamma_{qg} \\ \Gamma_{gq} & \Gamma_{gg}\end{pmatrix}}_{(m_c^2,Q_0^2)}\Bigg\}{\Sigma^{(3)}(Q_0^2) \choose g^{(3)}(Q_0^2)}
\end{array}
\end{equation} 
\item $\mathbf{T_{24}}$ and $\mathbf{T_{35}}$:
\begin{equation}
T_{24,35}^{(4)}(Q^2) = \Bigg\{\underbrace{\begin{pmatrix} \Gamma_{qq} & \Gamma_{qg}\end{pmatrix}}_{(Q^2,m_c^2)}\begin{pmatrix} M_{11}^c & M_{12}^c \\ M_{21}^c & M_{22}^c\end{pmatrix}\underbrace{\begin{pmatrix} \Gamma_{qq}& \Gamma_{qg} \\ \Gamma_{gq}& \Gamma_{gg}\end{pmatrix}}_{(m_c^2,Q_0^2)}\Bigg\}{\Sigma^{(3)}(Q_0^2) \choose g^{(3)}(Q_0^2)}
\end{equation}
\end{itemize}

Now, it is very easy to rewrite the above summary for the crossing of the remaining thresholds $m_b^2$ ($Q_0^2<m_b^2\leq Q^2$) and $m_t^2$ ($Q_0^2<m_t^2\leq Q^2$).

\subsubsection{$Q_0^2<m_b^2\leq Q^2$}
\begin{itemize}
\item \textbf{Singlet} and \textbf{gluon}:
\begin{equation}
{\Sigma^{(5)}(Q^2) \choose g^{(5)}(Q^2)} = \Bigg\{\underbrace{\begin{pmatrix} \Gamma_{qq} & \Gamma_{qg} \\ \Gamma_{gq}& \Gamma_{gg}\end{pmatrix}}_{(Q^2,m_b^2)}\begin{pmatrix} M_{11}^b & M_{12}^b \\ M_{21}^b & M_{22}^b\end{pmatrix}\underbrace{\begin{pmatrix} \Gamma_{qq}& \Gamma_{qg} \\ \Gamma_{gq}& \Gamma_{gg}\end{pmatrix}}_{(m_b^2,Q_0^2)}\Bigg\}{\Sigma^{(4)}(Q_0^2) \choose g^{(4)}(Q_0^2)}
\end{equation}
where:
\begin{equation}
\begin{pmatrix} M_{11}^b & M_{12}^b \\ M_{21}^b & M_{22}^b\end{pmatrix}=\begin{pmatrix} 1 & 0 \\ 0 & 1\end{pmatrix}+a_s^2(m_b^2)A_{qq,b}^{N\!S,(2)}\begin{pmatrix} 1 & 0 \\ 0 & 0\end{pmatrix}+a_s^2(m_b^2)\begin{pmatrix} \tilde{A}^{S,(2)}_{bq} & \tilde{A}^{S,(2)}_{bg} \\A^{S,(2)}_{gq,b} & A_{gg,b}^{S,(2)}\end{pmatrix}
\end{equation}
\item $\mathbf{V}$:
\begin{equation}
V^{(5)}(Q^2)=\left\{\Gamma^{v}(Q^2,m_b^2)[1+a_s^2(m_b^2)A_{qq,b}^{N\!S,(2)}]\Gamma^{v}(m_b^2,Q_0^2)\right\}V^{(4)}(Q^2_0)
\end{equation}
\item $\mathbf{V_3}$, $\mathbf{V_8}$ and $\mathbf{V_{15}}$:
\begin{equation}
V^{(5)}_{3,8,15}(Q^2)=\left\{\Gamma^{-}(Q^2,m_b^2)[1+a_s^2(m_b^2)A_{qq,b}^{N\!S,(2)}]\Gamma^{-}(m_b^2,Q_0^2)\right\}V^{(4)}_{3,8,15}(Q^2_0)
\end{equation}
\item $\mathbf{V_{24}}$:
\begin{equation}
V_{24}^{(5)}(Q^2)=\left\{\Gamma^{-}(Q_0^2,m_b^2)[1+a_s^2(m_b^2)A_{qq,b}^{N\!S,(2)}]\Gamma^{v}(m_b^2,Q_0^2)\right\}V^{(4)}(Q_0^2)
\end{equation}
\item $\mathbf{V_{35}}$:
\begin{equation}
V_{35}^{(5)}(Q^2)=\left\{\Gamma^{v}(Q_0^2,m_b^2)[1+a_s^2(m_b^2)A_{qq,b}^{N\!S,(2)}]\Gamma^{v}(m_b^2,Q_0^2)\right\}V^{(4)}(Q_0^2)
\end{equation}
\item $\mathbf{T_3}$, $\mathbf{T_8}$ and $\mathbf{T_{15}}$:
\begin{equation}
T^{(5)}_{3,8,15}(Q^2)=\left\{\Gamma^{+}(Q^2,m_b^2)[1+a_s^2(m_b^2)A_{qq,b}^{N\!S,(2)}]\Gamma^{+}(m_b^2,Q_0^2)\right\}T^{(4)}_{3,8,15}(Q^2_0)
\end{equation}
\item $\mathbf{T_{24}}$:
\begin{equation}
\begin{array}{rcl}
\displaystyle T_{24}^{(5)}(Q^2)&=&\Bigg\{\Gamma^{+}(Q^2,m_b^2)\begin{pmatrix} 1+a_s^2(m_b^2)[A_{qq,b}^{N\!S,(2)}-4\tilde{A}^{S,(2)}_{bq}] & -4a_s^2(m_b^2)\tilde{A}^{S,(2)}_{bg}\end{pmatrix}\times\\
\\
 & & \displaystyle \underbrace{\begin{pmatrix}\Gamma_{qq}& \Gamma_{qg} \\ \Gamma_{gq} & \Gamma_{gg}\end{pmatrix}}_{(m_b^2,Q_0^2)}\Bigg\}{\Sigma^{(4)}(Q_0^2) \choose g^{(4)}(Q_0^2)}
\end{array}
\end{equation} 
\item $\mathbf{T_{35}}$:
\begin{equation}
T_{35}^{(5)}(Q^2) = \Bigg\{\underbrace{\begin{pmatrix} \Gamma_{qq} & \Gamma_{qg}\end{pmatrix}}_{(Q^2,m_b^2)}\begin{pmatrix} M_{11}^b & M_{12}^b \\ M_{21}^b & M_{22}^b\end{pmatrix}\underbrace{\begin{pmatrix} \Gamma_{qq}& \Gamma_{qg} \\ \Gamma_{gq}& \Gamma_{gg}\end{pmatrix}}_{(m_b^2,Q_0^2)}\Bigg\}{\Sigma^{(4)}(Q_0^2) \choose g^{(4)}(Q_0^2)}
\end{equation}
\end{itemize}

\subsubsection{$Q_0^2<m_t^2\leq Q^2$}
\begin{itemize}
\item \textbf{Singlet} and \textbf{gluon}:
\begin{equation}
{\Sigma^{(6)}(Q^2) \choose g^{(6)}(Q^2)} = \Bigg\{\underbrace{\begin{pmatrix} \Gamma_{qq} & \Gamma_{qg} \\ \Gamma_{gq}& \Gamma_{gg}\end{pmatrix}}_{(Q^2,m_t^2)}\begin{pmatrix} M_{11}^t & M_{12}^t \\ M_{21}^t & M_{22}^t\end{pmatrix}\underbrace{\begin{pmatrix} \Gamma_{qq}& \Gamma_{qg} \\ \Gamma_{gq}& \Gamma_{gg}\end{pmatrix}}_{(m_t^2,Q_0^2)}\Bigg\}{\Sigma^{(5)}(Q_0^2) \choose g^{(5)}(Q_0^2)}
\end{equation}
where:
\begin{equation}
\begin{pmatrix} M_{11}^t & M_{12}^t \\ M_{21}^t & M_{22}^t\end{pmatrix}=\begin{pmatrix} 1 & 0 \\ 0 & 1\end{pmatrix}+a_s^2(m_t^2)A_{qq,t}^{N\!S,(2)}\begin{pmatrix} 1 & 0 \\ 0 & 0\end{pmatrix}+a_s^2(m_t^2)\begin{pmatrix} \tilde{A}^{S,(2)}_{tq} & \tilde{A}^{S,(2)}_{tg} \\A^{S,(2)}_{gq,t} & A_{gg,t}^{S,(2)}\end{pmatrix}
\end{equation}
\item $\mathbf{V}$:
\begin{equation}
V^{(6)}(Q^2)=\left\{\Gamma^{v}(Q^2,m_t^2)[1+a_s^2(m_t^2)A_{qq,t}^{N\!S,(2)}]\Gamma^{v}(m_t^2,Q_0^2)\right\}V^{(5)}(Q^2_0)
\end{equation}
\item $\mathbf{V_3}$, $\mathbf{V_8}$, $\mathbf{V_{15}}$ and $\mathbf{V_{24}}$:
\begin{equation}
V^{(6)}_{3,8,15,24}(Q^2)=\left\{\Gamma^{-}(Q^2,m_t^2)[1+a_s^2(m_t^2)A_{qq,t}^{N\!S,(2)}]\Gamma^{-}(m_t^2,Q_0^2)\right\}V^{(5)}_{3,8,15,24}(Q^2_0)
\end{equation}
\item $\mathbf{V_{35}}$:
\begin{equation}
V_{35}^{(6)}(Q^2)=\left\{\Gamma^{-}(Q_0^2,m_t^2)[1+a_s^2(m_t^2)A_{qq,t}^{N\!S,(2)}]\Gamma^{v}(m_t^2,Q_0^2)\right\}V^{(5)}(Q_0^2)
\end{equation}
\item $\mathbf{T_3}$, $\mathbf{T_8}$, $\mathbf{T_{15}}$ and $\mathbf{T_{24}}$:
\begin{equation}
T^{(6)}_{3,8,15,24}(Q^2)=\left\{\Gamma^{+}(Q^2,m_t^2)[1+a_s^2(m_t^2)A_{qq,t}^{N\!S,(2)}]\Gamma^{+}(m_t^2,Q_0^2)\right\}T^{(5)}_{3,8,15,24}(Q^2_0)
\end{equation}
\item $\mathbf{T_{35}}$:
\begin{equation}
\begin{array}{rcl}
\displaystyle T_{35}^{(6)}(Q^2)&=&\Bigg\{\Gamma^{+}(Q^2,m_t^2)\begin{pmatrix} 1+a_s^2(m_t^2)[A_{qq,t}^{N\!S,(2)}-5\tilde{A}^{S,(2)}_{tq}] & -5a_s^2(m_t^2)\tilde{A}^{S,(2)}_{tg}\end{pmatrix}\times\\
\\
 & & \displaystyle \underbrace{\begin{pmatrix}\Gamma_{qq}& \Gamma_{qg} \\ \Gamma_{gq} & \Gamma_{gg}\end{pmatrix}}_{(m_t^2,Q_0^2)}\Bigg\}{\Sigma^{(5)}(Q_0^2) \choose g^{(5)}(Q_0^2)}
\end{array}
\end{equation} 
\end{itemize}

\subsection{Matching Conditions on the Evolution Kernels: 2 Thresholds Crossing}

In this Section we will discuss the case in which the evolution crosses two thrersholds. Therefore there are only two situations: 1) $Q_0^2<m_c^2<m_b^2\leq Q^2$ and 2) $Q_0^2<m_b^2<m_t^2\leq Q^2$. Anyway, there is nothing new, indeed to obtain such evolution kernerls we have just to ``merge'' togheter what we have already done in the previous Section.

\subsubsection{$Q_0^2<m_c^2<m_b^2\leq Q^2$}

\begin{itemize}
\item \textbf{Singlet} and \textbf{gluon}:
\begin{equation}
\begin{array}{rcl}
\displaystyle {\Sigma^{(5)}(Q^2) \choose g^{(5)}(Q^2)} &=& \displaystyle \Bigg\{\underbrace{\begin{pmatrix} \Gamma_{qq} & \Gamma_{qg} \\ \Gamma_{gq}& \Gamma_{gg}\end{pmatrix}}_{(Q^2,m_b^2)}\begin{pmatrix} M_{11}^b & M_{12}^b \\ M_{21}^b & M_{22}^b\end{pmatrix}\underbrace{\begin{pmatrix} \Gamma_{qq}& \Gamma_{qg} \\ \Gamma_{gq}& \Gamma_{gg}\end{pmatrix}}_{(m_b^2,m_c^2)}\times\\
\\
 & &\displaystyle\begin{pmatrix} M_{11}^c & M_{12}^c \\ M_{21}^c & M_{22}^c\end{pmatrix}\underbrace{\begin{pmatrix} \Gamma_{qq}& \Gamma_{qg} \\ \Gamma_{gq}& \Gamma_{gg}\end{pmatrix}}_{(m_c^2,Q_0^2)}\Bigg\}{\Sigma^{(3)}(Q_0^2) \choose g^{(3)}(Q_0^2)}
\end{array}
\end{equation}
\item $\mathbf{V}$:
\begin{equation}
\begin{array}{rcl}
V^{(5)}(Q^2)&=&\displaystyle \Big\{\Gamma^{v}(Q^2,m_b^2)[1+a_s^2(m_b^2)A_{qq,b}^{N\!S,(2)}]\times\\
\\
 & &\displaystyle \Gamma^{v}(m_b^2,m_c^2)[1+a_s^2(m_c^2)A_{qq,c}^{N\!S,(2)}]\Gamma^{v}(m_c^2,Q_0^2)\Big\}V^{(3)}(Q^2_0)
\end{array}
\end{equation}
\item $\mathbf{V_3}$ and $\mathbf{V_8}$:
\begin{equation}
\begin{array}{rcl}
V^{(5)}_{3,8}(Q^2)&=&\displaystyle \Big\{\Gamma^{-}(Q^2,m_b^2)[1+a_s^2(m_b^2)A_{qq,b}^{N\!S,(2)}]\times\\
\\
& & \displaystyle \Gamma^{-}(m_b^2,m_c^2)[1+a_s^2(m_c^2)A_{qq,c}^{N\!S,(2)}]\Gamma^{-}(m_c^2,Q_0^2)\Big\}V^{(3)}_{3,8}(Q^2_0)
\end{array}
\end{equation}
\item $\mathbf{V_{15}}$:
\begin{equation}
\begin{array}{rcl}
V^{(5)}_{15}(Q^2)&=&\displaystyle \Big\{\Gamma^{-}(Q^2,m_b^2)[1+a_s^2(m_b^2)A_{qq,b}^{N\!S,(2)}]\times\\
\\
& & \displaystyle \Gamma^{-}(m_b^2,m_c^2)[1+a_s^2(m_c^2)A_{qq,c}^{N\!S,(2)}]\Gamma^{v}(m_c^2,Q_0^2)\Big\}V^{(3)}(Q^2_0)
\end{array}
\end{equation}
\item $\mathbf{V_{24}}$:
\begin{equation}
\begin{array}{rcl}
V^{(5)}_{24}(Q^2)&=&\displaystyle \Big\{\Gamma^{-}(Q^2,m_b^2)[1+a_s^2(m_b^2)A_{qq,b}^{N\!S,(2)}]\times\\
\\
& & \displaystyle \Gamma^{v}(m_b^2,m_c^2)[1+a_s^2(m_c^2)A_{qq,c}^{N\!S,(2)}]\Gamma^{v}(m_c^2,Q_0^2)\Big\}V^{(3)}(Q^2_0)
\end{array}
\end{equation}
\item $\mathbf{V_{35}}$:
\begin{equation}
\begin{array}{rcl}
V^{(5)}_{35}(Q^2)&=&\displaystyle \Big\{\Gamma^{v}(Q^2,m_b^2)[1+a_s^2(m_b^2)A_{qq,b}^{N\!S,(2)}]\times\\
\\
& & \displaystyle \Gamma^{v}(m_b^2,m_c^2)[1+a_s^2(m_c^2)A_{qq,c}^{N\!S,(2)}]\Gamma^{v}(m_c^2,Q_0^2)\Big\}V^{(3)}(Q^2_0)
\end{array}
\end{equation}

\item $\mathbf{T_3}$ and $\mathbf{T_8}$:
\begin{equation}
\begin{array}{rcl}
T^{(5)}_{3,8}(Q^2)&=&\displaystyle \Big\{\Gamma^{+}(Q^2,m_b^2)[1+a_s^2(m_b^2)A_{qq,b}^{N\!S,(2)}]\times\\
\\
& & \displaystyle \Gamma^{+}(m_b^2,m_c^2)[1+a_s^2(m_c^2)A_{qq,c}^{N\!S,(2)}]\Gamma^{+}(m_c^2,Q_0^2)\Big\}T^{(3)}_{3,8}(Q^2_0)
\end{array}
\end{equation}

\item $\mathbf{T_{15}}$:
\begin{equation}
\begin{array}{rcl}
T^{(5)}_{15}(Q^2)&=&\displaystyle \Bigg\{\Gamma^{+}(Q^2,m_b^2)[1+a_s^2(m_b^2)A_{qq,b}^{N\!S,(2)}]\Gamma^{+}(m_b^2,m_c^2)\times\\
\\
& & \displaystyle \begin{pmatrix} 1+a_s^2(m_c^2)[A_{qq,c}^{N\!S,(2)}-3\tilde{A}^{S,(2)}_{cq}] & -3a_s^2(m_c^2)\tilde{A}^{S,(2)}_{cg}\end{pmatrix}\underbrace{\begin{pmatrix}\Gamma_{qq}& \Gamma_{qg} \\ \Gamma_{gq} & \Gamma_{gg}\end{pmatrix}}_{(m_c^2,Q_0^2)}\Bigg\}{\Sigma^{(3)}(Q_0^2) \choose g^{(3)}(Q_0^2)}
\end{array}
\end{equation}


\item $\mathbf{T_{24}}$:
\begin{equation}
\begin{array}{rcl}
\displaystyle T_{24}^{(5)}(Q^2)&=&\Bigg\{\Gamma^{+}(Q^2,m_b^2)\begin{pmatrix} 1+a_s^2(m_b^2)[A_{qq,b}^{N\!S,(2)}-4\tilde{A}^{S,(2)}_{bq}] & -4a_s^2(m_b^2)\tilde{A}^{S,(2)}_{bg}\end{pmatrix}\times\\
\\
 & & \displaystyle \underbrace{\begin{pmatrix}\Gamma_{qq}& \Gamma_{qg} \\ \Gamma_{gq} & \Gamma_{gg}\end{pmatrix}}_{(m_b^2,m_c^2)}\begin{pmatrix} M^c_{11} & M^c_{12} \\ M_{21}^c & M_{22}^c \end{pmatrix}\underbrace{\begin{pmatrix}\Gamma_{qq}& \Gamma_{qg} \\ \Gamma_{gq} & \Gamma_{gg}\end{pmatrix}}_{(m_c^2,Q_0^2)}\Bigg\}{\Sigma^{(3)}(Q_0^2) \choose g^{(3)}(Q_0^2)}
\end{array}
\end{equation}
\item $\mathbf{T_{35}}$:
\begin{equation}
\begin{array}{rcl}
T_{35}^{(5)}(Q^2) &=&\displaystyle \Bigg\{\underbrace{\begin{pmatrix} \Gamma_{qq} & \Gamma_{qg}\end{pmatrix}}_{(Q^2,m_b^2)}\begin{pmatrix} M_{11}^b & M_{12}^b \\ M_{21}^b & M_{22}^b\end{pmatrix}\underbrace{\begin{pmatrix} \Gamma_{qq} & \Gamma_{qg} \\ \Gamma_{gq}& \Gamma_{gg}\end{pmatrix}}_{(m_b^2,m_c^2)}\times\\
\\
& & \displaystyle \begin{pmatrix} M^c_{11} & M^c_{12} \\ M_{21}^c & M_{22}^c \end{pmatrix}\underbrace{\begin{pmatrix}\Gamma_{qq}& \Gamma_{qg} \\ \Gamma_{gq} & \Gamma_{gg}\end{pmatrix}}_{(m_c^2,Q_0^2)}\Bigg\} {\Sigma^{(3)}(Q_0^2) \choose g^{(3)}(Q_0^2)}
\end{array}
\end{equation}
\end{itemize}

\subsubsection{$Q_0^2<m_b^2<m_t^2\leq Q^2$}

\begin{itemize}
\item \textbf{Singlet} and \textbf{gluon}:
\begin{equation}
\begin{array}{rcl}
\displaystyle {\Sigma^{(6)}(Q^2) \choose g^{(6)}(Q^2)} &=& \displaystyle \Bigg\{\underbrace{\begin{pmatrix} \Gamma_{qq} & \Gamma_{qg} \\ \Gamma_{gq}& \Gamma_{gg}\end{pmatrix}}_{(Q^2,m_t^2)}\begin{pmatrix} M_{11}^t & M_{12}^t \\ M_{21}^t & M_{22}^t\end{pmatrix}\underbrace{\begin{pmatrix} \Gamma_{qq}& \Gamma_{qg} \\ \Gamma_{gq}& \Gamma_{gg}\end{pmatrix}}_{(m_t^2,m_b^2)}\times\\
\\
 & &\displaystyle\begin{pmatrix} M_{11}^b & M_{12}^b \\ M_{21}^b & M_{22}^b\end{pmatrix}\underbrace{\begin{pmatrix} \Gamma_{qq}& \Gamma_{qg} \\ \Gamma_{gq}& \Gamma_{gg}\end{pmatrix}}_{(m_b^2,Q_0^2)}\Bigg\}{\Sigma^{(4)}(Q_0^2) \choose g^{(4)}(Q_0^2)}
\end{array}
\end{equation}
\item $\mathbf{V}$:
\begin{equation}
\begin{array}{rcl}
V^{(6)}(Q^2)&=&\displaystyle \Big\{\Gamma^{v}(Q^2,m_t^2)[1+a_s^2(m_t^2)A_{qq,t}^{N\!S,(2)}]\times\\
\\
 & &\displaystyle \Gamma^{v}(m_t^2,m_b^2)[1+a_s^2(m_b^2)A_{qq,b}^{N\!S,(2)}]\Gamma^{v}(m_b^2,Q_0^2)\Big\}V^{(4)}(Q^2_0)
\end{array}
\end{equation}
\item $\mathbf{V_3}$, $\mathbf{V_8}$ and $\mathbf{V_{15}}$:
\begin{equation}
\begin{array}{rcl}
V^{(6)}_{3,8,15}(Q^2)&=&\displaystyle \Big\{\Gamma^{-}(Q^2,m_t^2)[1+a_s^2(m_t^2)A_{qq,t}^{N\!S,(2)}]\times\\
\\
& & \displaystyle \Gamma^{-}(m_t^2,m_b^2)[1+a_s^2(m_b^2)A_{qq,b}^{N\!S,(2)}]\Gamma^{-}(m_b^2,Q_0^2)\Big\}V^{(4)}_{3,8,15}(Q^2_0)
\end{array}
\end{equation}
\item $\mathbf{V_{24}}$:
\begin{equation}
\begin{array}{rcl}
V^{(6)}_{24}(Q^2)&=&\displaystyle \Big\{\Gamma^{-}(Q^2,m_t^2)[1+a_s^2(m_t^2)A_{qq,t}^{N\!S,(2)}]\times\\
\\
& & \displaystyle \Gamma^{-}(m_t^2,m_b^2)[1+a_s^2(m_b^2)A_{qq,b}^{N\!S,(2)}]\Gamma^{v}(m_b^2,Q_0^2)\Big\}V^{(4)}(Q^2_0)
\end{array}
\end{equation}
\item $\mathbf{V_{35}}$:
\begin{equation}
\begin{array}{rcl}
V^{(6)}_{35}(Q^2)&=&\displaystyle \Big\{\Gamma^{-}(Q^2,m_t^2)[1+a_s^2(m_t^2)A_{qq,t}^{N\!S,(2)}]\times\\
\\
& & \displaystyle \Gamma^{v}(m_t^2,m_b^2)[1+a_s^2(m_b^2)A_{qq,b}^{N\!S,(2)}]\Gamma^{v}(m_b^2,Q_0^2)\Big\}V^{(4)}(Q^2_0)
\end{array}
\end{equation}

\item $\mathbf{T_3}$, $\mathbf{T_8}$ and $\mathbf{T_{15}}$:
\begin{equation}
\begin{array}{rcl}
T^{(6)}_{3,8,15}(Q^2)&=&\displaystyle \Big\{\Gamma^{+}(Q^2,m_t^2)[1+a_s^2(m_t^2)A_{qq,t}^{N\!S,(2)}]\times\\
\\
& & \displaystyle \Gamma^{+}(m_t^2,m_b^2)[1+a_s^2(m_b^2)A_{qq,b}^{N\!S,(2)}]\Gamma^{+}(m_b^2,Q_0^2)\Big\}T^{(4)}_{3,8,15}(Q^2_0)
\end{array}
\end{equation}

\item $\mathbf{T_{24}}$:
\begin{equation}
\begin{array}{rcl}
T^{(6)}_{24}(Q^2)&=&\displaystyle \Bigg\{\Gamma^{+}(Q^2,m_t^2)[1+a_s^2(m_t^2)A_{qq,t}^{N\!S,(2)}]\Gamma^{+}(m_t^2,m_b^2)\times\\
\\
& & \displaystyle \begin{pmatrix} 1+a_s^2(m_b^2)[A_{qq,b}^{N\!S,(2)}-4\tilde{A}^{S,(2)}_{bq}] & -4a_s^2(m_b^2)\tilde{A}^{S,(2)}_{bg}\end{pmatrix}\underbrace{\begin{pmatrix}\Gamma_{qq}& \Gamma_{qg} \\ \Gamma_{gq} & \Gamma_{gg}\end{pmatrix}}_{(m_b^2,Q_0^2)}\Bigg\}{\Sigma^{(4)}(Q_0^2) \choose g^{(4)}(Q_0^2)}
\end{array}
\end{equation}


\item $\mathbf{T_{35}}$:
\begin{equation}
\begin{array}{rcl}
\displaystyle T_{35}^{(6)}(Q^2)&=&\Bigg\{\Gamma^{+}(Q^2,m_t^2)\begin{pmatrix} 1+a_s^2(m_t^2)[A_{qq,t}^{N\!S,(2)}-5\tilde{A}^{S,(2)}_{tq}] & -5a_s^2(m_t^2)\tilde{A}^{S,(2)}_{tg}\end{pmatrix}\times\\
\\
 & & \displaystyle \underbrace{\begin{pmatrix}\Gamma_{qq}& \Gamma_{qg} \\ \Gamma_{gq} & \Gamma_{gg}\end{pmatrix}}_{(m_t^2,m_b^2)}\begin{pmatrix} M^b_{11} & M^b_{12} \\ M_{21}^b & M_{22}^b \end{pmatrix}\underbrace{\begin{pmatrix}\Gamma_{qq}& \Gamma_{qg} \\ \Gamma_{gq} & \Gamma_{gg}\end{pmatrix}}_{(m_b^2,Q_0^2)}\Bigg\}{\Sigma^{(4)}(Q_0^2) \choose g^{(4)}(Q_0^2)}
\end{array}
\end{equation}
\end{itemize}

\subsection{Matching Conditions on the Evolution Kernels: 3 Thresholds Crossing}

In this Section we will discuss the case in which the evolution crosses three thrersholds. Therefore there ais only one situation: $Q_0^2<m_c^2<m_b^2,m_t^2\leq Q^2$

\subsubsection{$Q_0^2<m_c^2<m_b^2<m_t^2\leq Q^2$}
\begin{itemize}
\item \textbf{Singlet} and \textbf{gluon}:
\begin{equation}
\begin{array}{rcl}
\displaystyle {\Sigma^{(6)}(Q^2) \choose g^{(6)}(Q^2)} &=& \displaystyle \Bigg\{\underbrace{\begin{pmatrix} \Gamma_{qq} & \Gamma_{qg} \\ \Gamma_{gq}& \Gamma_{gg}\end{pmatrix}}_{(Q^2,m_t^2)}\begin{pmatrix} M_{11}^t & M_{12}^t \\ M_{21}^t & M_{22}^t\end{pmatrix}\underbrace{\begin{pmatrix} \Gamma_{qq}& \Gamma_{qg} \\ \Gamma_{gq}& \Gamma_{gg}\end{pmatrix}}_{(m_t^2,m_b^2)}\times\\
\\
 & &\displaystyle\begin{pmatrix} M_{11}^b & M_{12}^b \\ M_{21}^b & M_{22}^b\end{pmatrix}\underbrace{\begin{pmatrix} \Gamma_{qq}& \Gamma_{qg} \\ \Gamma_{gq}& \Gamma_{gg}\end{pmatrix}}_{(m_b^2,m_c^2)}\begin{pmatrix} M_{11}^c & M_{12}^c \\ M_{21}^c & M_{22}^c\end{pmatrix}\\
\\
& &\displaystyle \underbrace{\begin{pmatrix} \Gamma_{qq}& \Gamma_{qg} \\ \Gamma_{gq}& \Gamma_{gg}\end{pmatrix}}_{(m_c^2,Q_0^2)}\Bigg\}{\Sigma^{(3)}(Q_0^2) \choose g^{(3)}(Q_0^2)}
\end{array}
\end{equation}
\item $\mathbf{V}$:
\begin{equation}
\begin{array}{rcl}
V^{(6)}(Q^2)&=&\displaystyle \Big\{\Gamma^{v}(Q^2,m_t^2)[1+a_s^2(m_t^2)A_{qq,t}^{N\!S,(2)}]\times\\
\\
 & &\displaystyle \Gamma^{v}(m_t^2,m_b^2)[1+a_s^2(m_b^2)A_{qq,b}^{N\!S,(2)}]\Gamma^{v}(m_b^2,m_c^2)\\
\\
& & \displaystyle[1+a_s^2(m_c^2)A_{qq,c}^{N\!S,(2)}]\Gamma^{v}(m_c^2,Q_0^2)\Big\}V^{(3)}(Q^2_0)
\end{array}
\end{equation}
\item $\mathbf{V_3}$ and $\mathbf{V_8}$:
\begin{equation}
\begin{array}{rcl}
V^{(6)}_{3,8}(Q^2)&=&\displaystyle \Big\{\Gamma^{-}(Q^2,m_t^2)[1+a_s^2(m_t^2)A_{qq,t}^{N\!S,(2)}]\times\\
\\
& & \displaystyle \Gamma^{-}(m_t^2,m_b^2)[1+a_s^2(m_b^2)A_{qq,b}^{N\!S,(2)}]\Gamma^{-}(m_b^2,m_c^2)\\
\\
& & \displaystyle [1+a_s^2(m_c^2)A_{qq,c}^{N\!S,(2)}]\Gamma^{-}(m_c^2,Q_0^2)\Big\}V^{(3)}_{3,8}(Q^2_0)
\end{array}
\end{equation}
\item $\mathbf{V_{15}}$:
\begin{equation}
\begin{array}{rcl}
V^{(6)}_{15}(Q^2)&=&\displaystyle \Big\{\Gamma^{-}(Q^2,m_t^2)[1+a_s^2(m_t^2)A_{qq,t}^{N\!S,(2)}]\times\\
\\
& & \displaystyle \Gamma^{-}(m_t^2,m_b^2)[1+a_s^2(m_b^2)A_{qq,b}^{N\!S,(2)}]\Gamma^{-}(m_b^2,m_c^2)\\ 
\\
& & \displaystyle [1+a_s^2(m_c^2)A_{qq,c}^{N\!S,(2)}]\Gamma^{v}(m_c^2,Q_0^2)\Big\}V^{(3)}(Q^2_0)
\end{array}
\end{equation}
\item $\mathbf{V_{24}}$:
\begin{equation}
\begin{array}{rcl}
V^{(6)}_{15}(Q^2)&=&\displaystyle \Big\{\Gamma^{-}(Q^2,m_t^2)[1+a_s^2(m_t^2)A_{qq,t}^{N\!S,(2)}]\times\\
\\
& & \displaystyle \Gamma^{-}(m_t^2,m_b^2)[1+a_s^2(m_b^2)A_{qq,b}^{N\!S,(2)}]\Gamma^{v}(m_b^2,m_c^2)\\ 
\\
& & \displaystyle [1+a_s^2(m_c^2)A_{qq,c}^{N\!S,(2)}]\Gamma^{v}(m_c^2,Q_0^2)\Big\}V^{(3)}(Q^2_0)
\end{array}
\end{equation}
\item $\mathbf{V_{35}}$:
\begin{equation}
\begin{array}{rcl}
V^{(6)}_{15}(Q^2)&=&\displaystyle \Big\{\Gamma^{-}(Q^2,m_t^2)[1+a_s^2(m_t^2)A_{qq,t}^{N\!S,(2)}]\times\\
\\
& & \displaystyle \Gamma^{v}(m_t^2,m_b^2)[1+a_s^2(m_b^2)A_{qq,b}^{N\!S,(2)}]\Gamma^{v}(m_b^2,m_c^2)\\ 
\\
& & \displaystyle [1+a_s^2(m_c^2)A_{qq,c}^{N\!S,(2)}]\Gamma^{v}(m_c^2,Q_0^2)\Big\}V^{(3)}(Q^2_0)
\end{array}
\end{equation}
\item $\mathbf{T_3}$ and $\mathbf{T_8}$:
\begin{equation}
\begin{array}{rcl}
T^{(6)}_{3,8}(Q^2)&=&\displaystyle \Big\{\Gamma^{+}(Q^2,m_t^2)[1+a_s^2(m_t^2)A_{qq,t}^{N\!S,(2)}]\times\\
\\
& & \displaystyle \Gamma^{+}(m_t^2,m_b^2)[1+a_s^2(m_b^2)A_{qq,b}^{N\!S,(2)}]\Gamma^{+}(m_b^2,m_c^2)\\
\\
& & \displaystyle [1+a_s^2(m_c^2)A_{qq,c}^{N\!S,(2)}]\Gamma^{+}(m_c^2,Q_0^2)\Big\}T^{(3)}_{3,8}(Q^2_0)
\end{array}
\end{equation}

\item $\mathbf{T_{15}}$:
\begin{equation}
\begin{array}{rcl}
T^{(6)}_{15}(Q^2)&=&\displaystyle \Big\{\Gamma^{+}(Q^2,m_t^2)[1+a_s^2(m_t^2)A_{qq,t}^{N\!S,(2)}]\times\\
\\
& & \displaystyle \Gamma^{+}(m_t^2,m_b^2)[1+a_s^2(m_b^2)A_{qq,b}^{N\!S,(2)}]\Gamma^{+}(m_b^2,m_c^2)\\
\\
& & \displaystyle \begin{pmatrix} 1+a_s^2(m_c^2)[A_{qq,c}^{N\!S,(2)}-3\tilde{A}^{S,(2)}_{cq}] & -3a_s^2(m_c^2)\tilde{A}^{S,(2)}_{cg}\end{pmatrix}\underbrace{\begin{pmatrix}\Gamma_{qq}& \Gamma_{qg} \\ \Gamma_{gq} & \Gamma_{gg}\end{pmatrix}}_{(m_c^2,Q_0^2)}\Bigg\}{\Sigma^{(3)}(Q_0^2) \choose g^{(3)}(Q_0^2)}
\end{array}
\end{equation}
\item $\mathbf{T_{24}}$:
\begin{equation}
\begin{array}{rcl}
T^{(6)}_{24}(Q^2)&=&\displaystyle \Big\{\Gamma^{+}(Q^2,m_t^2)[1+a_s^2(m_t^2)A_{qq,t}^{N\!S,(2)}]\Gamma^{+}(m_t^2,m_b^2)\times\\
\\
& & \displaystyle \begin{pmatrix} 1+a_s^2(m_b^2)[A_{qq,b}^{N\!S,(2)}-4\tilde{A}^{S,(2)}_{bq}] & -4a_s^2(m_b^2)\tilde{A}^{S,(2)}_{bg}\end{pmatrix}\underbrace{\begin{pmatrix}\Gamma_{qq}& \Gamma_{qg} \\ \Gamma_{gq} & \Gamma_{gg}\end{pmatrix}}_{(m_b^2,m_c^2)}\times\\
\\
 & &\displaystyle\begin{pmatrix} M_{11}^c & M_{12}^c \\ M_{21}^c & M_{22}^c\end{pmatrix}\underbrace{\begin{pmatrix} \Gamma_{qq}& \Gamma_{qg} \\ \Gamma_{gq}& \Gamma_{gg}\end{pmatrix}}_{(m_c^2,Q_0^2)}\Bigg\}{\Sigma^{(3)}(Q_0^2) \choose g^{(3)}(Q_0^2)}
\end{array}
\end{equation}
\item $\mathbf{T_{35}}$:
\begin{equation}
\begin{array}{rcl}
T^{(6)}_{35}(Q^2)&=&\displaystyle \Big\{\Gamma^{+}(Q^2,m_t^2)\begin{pmatrix} 1+a_s^2(m_b^2)[A_{qq,t}^{N\!S,(2)}-5\tilde{A}^{S,(2)}_{tq}] & -5a_s^2(m_t^2)\tilde{A}^{S,(2)}_{tg}\end{pmatrix}\times\\
\\
& &\displaystyle\begin{pmatrix} M_{11}^b & M_{12}^b \\ M_{21}^b & M_{22}^b\end{pmatrix}\underbrace{\begin{pmatrix} \Gamma_{qq}& \Gamma_{qg} \\ \Gamma_{gq}& \Gamma_{gg}\end{pmatrix}}_{(m_b^2,m_c^2)}\\
\\
& &\displaystyle\begin{pmatrix} M_{11}^c & M_{12}^c \\ M_{21}^c & M_{22}^c\end{pmatrix}\underbrace{\begin{pmatrix} \Gamma_{qq}& \Gamma_{qg} \\ \Gamma_{gq}& \Gamma_{gg}\end{pmatrix}}_{(m_c^2,Q_0^2)}\Bigg\}{\Sigma^{(3)}(Q_0^2) \choose g^{(3)}(Q_0^2)}
\end{array}
\end{equation}

\end{itemize}


\newpage
\begin{thebibliography}{alp}

\bibitem{Buza:1996wv}
  M.~Buza, Y.~Matiounine, J.~Smith and W.~L.~van Neerven,
  %``Charm electroproduction viewed in the variable-flavour number scheme
  %versus fixed-order perturbation theory,''
  Eur.\ Phys.\ J.\  C {\bf 1}, 301 (1998)
  [arXiv:hep-ph/9612398].
  %%CITATION = EPHJA,C1,301;%%

\bibitem{Vogt:2004ns}
  A.~Vogt,
  %``Efficient evolution of unpolarized and polarized parton distributions  with
  %QCD-PEGASUS,''
  Comput.\ Phys.\ Commun.\  {\bf 170} (2005) 65
  [arXiv:hep-ph/0408244].
  %%CITATION = CPHCB,170,65;%%

\end{thebibliography}





\end{document}
