%% LyX 2.0.3 created this file.  For more info, see http://www.lyx.org/.
%% Do not edit unless you really know what you are doing.
\documentclass[twoside,english]{paper}
\usepackage{lmodern}
\renewcommand{\ttdefault}{lmodern}
\usepackage[T1]{fontenc}
\usepackage[latin9]{inputenc}
\usepackage[a4paper]{geometry}
\geometry{verbose,tmargin=3cm,bmargin=2.5cm,lmargin=2cm,rmargin=2cm}
\usepackage{color}
\usepackage{babel}
\usepackage{float}
\usepackage{bm}
\usepackage{amsthm}
\usepackage{amsmath}
\usepackage{amssymb}
\usepackage{graphicx}
\usepackage{esint}
\usepackage[unicode=true,pdfusetitle,
 bookmarks=true,bookmarksnumbered=false,bookmarksopen=false,
 breaklinks=false,pdfborder={0 0 0},backref=false,colorlinks=false]
 {hyperref}
\usepackage{breakurl}

\makeatletter

%%%%%%%%%%%%%%%%%%%%%%%%%%%%%% LyX specific LaTeX commands.
%% Because html converters don't know tabularnewline
\providecommand{\tabularnewline}{\\}

%%%%%%%%%%%%%%%%%%%%%%%%%%%%%% Textclass specific LaTeX commands.
\numberwithin{equation}{section}
\numberwithin{figure}{section}

%%%%%%%%%%%%%%%%%%%%%%%%%%%%%% User specified LaTeX commands.
\usepackage{babel}

\@ifundefined{showcaptionsetup}{}{%
 \PassOptionsToPackage{caption=false}{subfig}}
\usepackage{subfig}
\makeatother

\usepackage{listings}

\makeindex

\begin{document}

\title{Intrinsic Charm Implementation}

\author{Valerio Bertone$^{a}$}

\institution{$^{a}$PH Department, TH Unit, CERN, CH-1211 Geneva 23, Switzerland}
\maketitle

\begin{abstract}
In these set of notes I will describe the strategy to include the
intrinsic charm (IC) contribution to the FONLL structure functions as
implemented in {\tt APFEL}. I will first consider the massive sector
(and its massless limit), where the IC implies the presence of the
charm in the initial state with the consequence of additional diagrams
to be include in the computation. I will the consider the massless
sector where the presence of an IC implies a retratment of the PDF
matching conditions at the charm threshold.
\end{abstract}

\tableofcontents{}

\section{Intrinsic Charm Contribution to the Massive Structure Functions}

Assuming the presence of IC in the proton, the massive structure 
functions with $N_f=3$ light flavours acquire a further contribution 
coming from the presence of a massive charm in the initial state. As a 
consequence, the massive structure functions get a term that is 
proportional to a \textit{static} charm PDF, $i.e.$ a PDF that, being 
massive, does not evolve according to the DGLAP equantion. Such a 
contribution starts already at order $\alpha_s^0$ and has the novel 
effect to ``allign'' the massive scheme to the massless scheme in 
terms of power counting because, contrary to what happens without IC,
the two sectors start at $\alpha_s^0$. 

\subsection{Order $\alpha_s^0$ Contributions}

In order to write explicitly the form of such LO contributions to the
DIS structure functions, I consider eq.~(2) of \cite{Kretzer:1998ju}
where the function $Q_1$ schould be indentified with the charm PDF. It
should be noticed that in the $N_f=3$ scheme, such PDF does not obey
the DGLAP equation because, due to the presence of the mass of the
charm $m_c$, no large collinear logarithms appear in the calculation
and thus there is no need to resum them.

From eq.~(2) of \cite{Kretzer:1998ju} one reads that the
$\mathcal{O}(\alpha_s^0)$ IC contributions to the massive structrure
functions are given by:
\begin{subequations}\label{GeneralSF}
\begin{equation}
F_1^{\rm FF,IC}(x,Q^2) = \frac{S_+\Sigma_{++}-2m_1m_2S_-}{2\Delta}c(\chi)
\end{equation}
\begin{equation}
F_2^{\rm FF,IC}(x,Q^2) = \frac{S_+\Delta}{2Q^2}2xc(\chi)
\end{equation}
\begin{equation}
xF_3^{\rm FF,IC}(x,Q^2) = 2Rxc(\chi)
\end{equation}
\end{subequations}
where $m_1$ and $m_2$ are the masses of the incoming and outcoming
quarks, respectively, while $\Delta\equiv\Delta(m_1^2,m_2^2,-Q^2)$
with the function $\Delta$ defined as:
\begin{equation}
  \Delta(a,b,c)=\sqrt{a^2+b^2+c^2 -2(ab+ac+bc)}
\end{equation}
and:
\begin{equation}
\Sigma_{\pm\pm}=Q^2\pm m_2^2\pm m_1^2
\end{equation}
\begin{equation}\label{Rescaledx}
\chi=\frac{x}{2Q^2}(\Sigma_{+-}+\Delta)
\end{equation}
The quantities $S_\pm$ and $R_\pm$, instead, are linked to the EW
couplings and depend on the vector boson that strikes the heavy quark
with mass $m_1$ in the initial state. Notice that in
eq.~(\ref{GeneralSF}) the PDF $c$ does not depend on any factorization
scale and, as mentioned before, the reason is that it is a static
distribution of non-perturbative origin that does not evolve according
to the DGLAP equation.

In practice, assuming the presence of IC in the proton, the massive
(FF) structure functions become:
\begin{equation}
F_i^{\rm FF}(x,Q^2)\longrightarrow F_i^{\rm FF}(x,Q^2)  + F_i^{\rm
  FF,IC}(x,Q^2) \quad\mbox{with}\quad i =1,2,3
\end{equation}

Now, for a purely \textit{electromagnetic} process, where only a
$\gamma$ strikes the charm, one has:
\begin{equation}
S_+=S_-=e_c^2\quad\mbox{and}\quad R=0
\end{equation}
Moreover, in this case both the incoming and the outcoming quarks are
of the same flavour (charm) therefore we have $m_1=m_2=m_c$. Under
this conditions one finds:
\begin{subequations}
\begin{equation}
F_1^{\rm FF,IC}(x,Q^2)= \frac{1}{2\sqrt{1+4\lambda}}e_c^2c(\chi)
\end{equation}
\begin{equation}
F_2^{\rm FF,IC}(x,Q^2)=\left(\sqrt{1+4\lambda}\right)e_c^2xc(\chi)
\end{equation}
\begin{equation}
xF_3^{\rm FF,IC}(x,Q^2)= 0
\end{equation}
\end{subequations}
with:
\begin{equation}\label{RescaledX}
\chi = \frac{x}{2}\left(1+\sqrt{1+4\lambda}\right) = \frac{x}{\eta}\,,
\end{equation}
where I have defined:
\begin{equation}\label{Rescaledx}
\eta=\frac{2Q^2}{\Sigma_{+-}+\Delta} = 2\left(1+\sqrt{1+4\lambda}\right)^{-1}=2\left(1+\sqrt{1+4\lambda}\right)^{-1}\,,
\end{equation}
with $\lambda=\frac{m_c^2}{Q^2}$.

For a \textit{neutral current} process, where all the $\gamma$, the
$Z$ and the interference $\gamma Z$ contributions are considered, one
has:
\begin{equation}
S_{\pm}=B_c(\tilde{B}_c)=e_c^2-2e_cV_eV_cP_Z+(V_e^2+A_e^2)(V_c^2\pm A_c^2)P_Z^2\quad\mbox{and}\quad R=D_c=-2e_c A_cA_eP_Z+4V_cA_cV_eA_eP_Z^2
\end{equation}
with:
\begin{equation}
V_c = \frac12-\frac43\sin^2\theta_W\quad\mbox{and}\quad A_c=\frac12
\end{equation}
and
\begin{equation}
V_e = -\frac12+2\sin^2\theta_W\quad\mbox{and}\quad A_e=-\frac12
\end{equation}
the vector and the axial coupling of charm and electron to the $Z$ and
where:
\begin{equation}
P_Z=\frac1{4\sin^2\theta_W(1-\sin^2\theta_W)}\frac{Q^2}{Q^2+M_Z^2}
\end{equation}
Here, exactly as in the electromagnetic case, $m_1=m_2=m_c$ so that
one ends up with:
\begin{subequations}
\begin{equation}
F_1^{\rm FF,IC}(x,Q^2)= \frac{B_c+2\lambda(B_c-\tilde{B}_c)}{2\sqrt{1+4\lambda}} c(\chi)\,,
\end{equation}
\begin{equation}
F_2^{\rm FF,IC}(x,Q^2)=\left(\sqrt{1+4\lambda}\right)B_cxc(\chi)\,,
\end{equation}
\begin{equation}
xF_3^{\rm FF,IC}(x,Q^2)= 2D_cxc(\chi)\,.
\end{equation}
\end{subequations}

Finally, for a \textit{charged current} process, where a charged boson
$W^\pm$ strikes the charm, one has:
\begin{equation}
S_+=S_-=2|V_{cs}|^2\quad\mbox{and}\quad R=|V_{cs}|^2
\end{equation}
if the outcoming quark is a strange or an anti-strange, and:
\begin{equation}
S_+=S_-=2|V_{cd}|^2\quad\mbox{and}\quad R=|V_{cd}|^2
\end{equation}
if the outcoming quark is a down or an anti-down. 

In this case $m_1=m_c$ but $m_2=0$ with the consequence that:
\begin{subequations}
\begin{equation}
F_1^{\rm FF,IC}(x,Q^2)= |V_{cj}|^2c(x)
\end{equation}
\begin{equation}
F_2^{\rm FF,IC}(x,Q^2)= 2\left(1+\lambda\right)|V_{cj}|^2xc(x)
\end{equation}
\begin{equation}
xF_3^{\rm FF,IC}(x,Q^2)= 2|V_{cj}|^2xc(x)
\end{equation}
\end{subequations}
with $j=d,s$. Note that in this case $\eta=1$ and thus $\chi=x$.

In order to take into account the possible contributions due to
intrinsic charm, one has to consider all diagrams contributing to a
given process. As far as the neutral current (electromagnetic) case is
concerned, one has to consider also the presence of $\overline{c}$ in
the proton which, summed to the contribution of the $c$, gives:
\begin{subequations}\label{NCEM}
\begin{equation}
F_1^{\rm FF,IC}(x,Q^2)= \frac{1}{2\sqrt{1+4\lambda}}B_c c^+(\chi)=\frac{1}{2x}B_c\frac{\eta^2}{2-\eta}\left[1 +\frac{2(1-\eta)}{\eta^2}\left(1-\frac{\tilde{B}_c}{B_c}\right)\right]\chi c^+(\chi)
\end{equation}
\begin{equation}
F_2^{\rm FF,IC}(x,Q^2)=\frac{2\sqrt{1+4\lambda}}{1+\sqrt{1+4\lambda}}B_c\chi c^+(\chi) = (2-\eta)B_c\chi c^+(\chi)
\end{equation}
\begin{equation}
xF_3^{\rm FF,IC}(x,Q^2)=\frac{4}{1+\sqrt{1+4\lambda}}D_c\chi c^-(\chi) = 2\eta D_c \chi c^-(\chi)
\end{equation}
\end{subequations}
where:
\begin{equation}
c^{\pm}=c\pm\overline{c}
\end{equation}
therefore:
\begin{equation}
F_L^{\rm FF,IC}(x,Q^2) = F_2^{\rm FF,IC}(x,Q^2) -2x F_1^{\rm
FF,IC}(x,Q^2)  = 4\frac{1-\eta}{2-\eta}
B_c\left[1-\frac{1}{2}\left(1-\frac{\tilde{B}_c}{B_c}\right)\right] \chi c^+(\chi)
\end{equation}

In the charged current case, instead, one has to distinguish between
neutrino and anti-neutrino scattering. The neutrino scattering gives
as a result the following structure functions:
\begin{subequations}\label{CCnu}
\begin{equation}
F_1^{\nu,\rm FF,IC}(x,Q^2)= (|V_{cd}|^2+|V_{cs}|^2)\overline{c}(x)
\end{equation}
\begin{equation}
F_2^{\nu,\rm FF,IC}(x,Q^2)= 2\left(1+\lambda\right)(|V_{cd}|^2+|V_{cs}|^2)x\overline{c}(x)
\end{equation}
\begin{equation}
xF_3^{\nu,\rm FF,IC}(x,Q^2)= 2(|V_{cd}|^2+|V_{cs}|^2)x\overline{c}(x)
\end{equation}
\begin{equation}
F_L^{\nu,\rm FF,IC}(x,Q^2) = 2\lambda(|V_{cd}|^2+|V_{cs}|^2)x\overline{c}(x)
\end{equation}
\end{subequations}
The anti-neutrino scattering instead gives as a result the following
structure functions:
\begin{subequations}\label{CCnb}
\begin{equation}
F_1^{\overline{\nu},\rm FF,IC}(x,Q^2)= (|V_{cd}|^2+|V_{cs}|^2)c(x)
\end{equation}
\begin{equation}
F_2^{\overline{\nu},\rm FF,IC}(x,Q^2)= 2\left(1+\lambda\right)(|V_{cd}|^2+|V_{cs}|^2)xc(x)
\end{equation}
\begin{equation}
xF_3^{\overline{\nu},\rm FF,IC}(x,Q^2)= 2(|V_{cd}|^2+|V_{cs}|^2)xc(x)
\end{equation}
\begin{equation}
F_L^{\overline{\nu},\rm FF,IC}(x,Q^2)= 2\lambda(|V_{cd}|^2+|V_{cs}|^2)xc(x)
\end{equation}
\end{subequations}

It should be pointed out that since the charm quark belongs to the sea
it is symmetric under isospin symmetry and thus all the above
structure functions are the same for proton and neutron.

\subsection{Massless Limit}

When implementing the FONLL scheme, one also needs to consider the
massless limit of the massive structuture functions (FF0). To this
end, we just need to take the limit for $m_c\rightarrow 0$ of
eqs.~(\ref{NCEM}), (\ref{CCnu}) and~(\ref{CCnb}). Considering that:
\begin{equation}
\eta \mathop{\longrightarrow}_{m_c\rightarrow0} 1 \quad\Rightarrow\quad \chi \mathop{\longrightarrow}_{m_c\rightarrow0} x\,,
\end{equation}
one finds :
\begin{subequations}\label{NCEM}
\begin{equation}
F_1^{\rm FF0,IC}(x,Q^2)= \frac{1}{2}B_c c^+(x)
\end{equation}
\begin{equation}
F_2^{\rm FF0,IC}(x,Q^2)=B_cxc^+(x)
\end{equation}
\begin{equation}
xF_3^{\rm FF0,IC}(x,Q^2)= 2D_cxc^-(x)
\end{equation}
\begin{equation}
F_L^{\rm FF0,IC}(x,Q^2)= 0
\end{equation}
\end{subequations}
and:
\begin{subequations}\label{CCnu}
\begin{equation}
F_1^{\nu,\rm FF0,IC}(x,Q^2)= (|V_{cd}|^2+|V_{cs}|^2)\overline{c}(x)
\end{equation}
\begin{equation}
F_2^{\nu,\rm FF0,IC}(x,Q^2)= 2(|V_{cd}|^2+|V_{cs}|^2)x\overline{c}(x)
\end{equation}
\begin{equation}
xF_3^{\nu,\rm FF0,IC}(x,Q^2)= 2(|V_{cd}|^2+|V_{cs}|^2)x\overline{c}(x)
\end{equation}
\begin{equation}
F_L^{\nu,\rm FF0,IC}(x,Q^2)= 0
\end{equation}
\end{subequations}
and:
\begin{subequations}\label{CCnb}
\begin{equation}
F_1^{\overline{\nu},\rm FF0,IC}(x,Q^2)= (|V_{cd}|^2+|V_{cs}|^2)c(x)
\end{equation}
\begin{equation}
F_2^{\overline{\nu},\rm FF0,IC}(x,Q^2)= 2(|V_{cd}|^2+|V_{cs}|^2)xc(x)
\end{equation}
\begin{equation}
xF_3^{\overline{\nu},\rm FF0,IC}(x,Q^2)= 2(|V_{cd}|^2+|V_{cs}|^2)xc(x)
\end{equation}
\begin{equation}
F_L^{\overline{\nu},\rm FF0,IC}(x,Q^2)= 0
\end{equation}
\end{subequations}

\subsection{Order $\alpha_s$ Contributions}

We can now turn to describe the NLO contributions to the IC component
of the DIS structure functions. The explicit expressions can be found
in Appendix C of Ref.~\cite{Kretzer:1998ju}. The main difficulty here
is the fact that are particularly involved and it is not possible to
identify the singular terms from the regular ones. As a consequence,
in this case we have to adopt a different strategy.

The kind of expressions we have to deal with have this apparently
simple form:
\begin{equation}\label{CoeffFuncs}
\widetilde{C}(y,m_1,m_2) = \frac{R(y,m_1,m_2)}{(1-y)_+}+ L(m_1,m_2)\delta(1-y)\,,
\end{equation}
where $R$ is a regular function of $x$, that is:
\begin{equation}
R(1,m_1,m_2) =  \lim_{y\rightarrow 1} R(y,m_1,m_2) = K(m_1,m_2)\,,
\end{equation}
$K$ being a finite function of $m_1$ and $m_2$.

As we know, when convoluting coefficient functions like that in
eq.~(\ref{CoeffFuncs}) with a PDF, say, $f(y)$ interpolated over an
$x$-space grid to obtaing the structure function $F(x,m_1,m_2)$ the
resulting expression is(\footnote{Note that the factor $x_\beta$ in
  front of the intergral in eq.~(\ref{ConvolutionGrid}) is not always
  included in the definition of the structure functions. In
  particular, while $F_2$ and $F_L$ include it, $F_1$ and and $F_3$ do
  not so we need to keep it in mind in what follows.}):
\begin{equation}\label{ConvolutionGrid}
F(x_\beta,m_1,m_2) =x_\beta
\int_{x_\beta}^1\frac{dy}{y}\widetilde{C}\left(\frac{x_\beta}{y},m_1,m_2\right)f(y)
= \sum^{N_{x}}_{\alpha=0} \Gamma_{\beta\alpha}(m_1,m_2)x_\alpha f(x_{\alpha},t)\,,
\end{equation}
with:
\begin{equation}\label{GridKernel}
\Gamma_{\beta\alpha}(m_1,m_2) = \int_c^d dy\,\widetilde{C}\left(y,m_1,m_2\right)
  w_{\alpha}^{(k)}\left(\frac{x_\beta}{y}\right)\,,
\end{equation}
where $w_{\alpha}^{(k)}$ is the order $k$ interpolation functions in
the grid node $\alpha$ and:
\begin{equation}
  c =
  \mbox{max}(x_\beta,x_\beta/x_{\alpha+1}) \quad\mbox{and}\quad d =
  \mbox{min}(1,x_\beta/x_{\alpha-k}) \,.
\end{equation}

As it usually happens when mass effects are taken into account, the
phase-space available to the process gets reduced and this is
reflected by the fact that the convolution between coeffient functions
and PDFs needed to obtain the structure functions takes the following form:
\begin{equation}\label{MassiveIntegral0}
  F(x)=x\int_\chi^1\frac{dz}{z}
  C(z)f\left(\frac{\chi}{z}\right)=x\int_\chi^1\frac{dz}{z}
  C\left(\frac{\chi}{z}\right)f(z)\,,
\end{equation}
where $\chi=x/\eta$ with $\eta\leq 1$ given in eq.~(\ref{Rescaledx})
and usually expression for the coefficient functions are given in this
form. However, in order to write this integral in the form given in
eq.~(\ref{ConvolutionGrid}), that is with lower bound of the integral
is not the rescaled variable $\chi$ but the physical Bjorken $x$, one
needs to perform the change of variable $y=\eta z$, so that:
\begin{equation}\label{MassiveIntegral}
  F(x)=x\int_x^\eta\frac{dy}{y}
  C\left(\frac{y}{\eta}\right)f\left(\frac{x}{y}\right)=x\int_x^1 \frac{dy}{y}\,
  \widetilde{C}(y)f\left(\frac{x}{y}\right)\,,
\end{equation}
with:
\begin{equation}\label{RescaledIntegrand}
\widetilde{C}(y) = \theta\left(1-\frac{y}{\eta}\right)C\left(\frac{y}{\eta}\right)\,.
\end{equation}
Now, plugging eq.~(\ref{RescaledIntegrand}) into
eq.~(\ref{GridKernel}), taking into account eq.~(\ref{CoeffFuncs}) and
dropping all the mass dependencies we have that:
\begin{equation}
\begin{array}{rcl}
\Gamma_{\beta\alpha} &=&\displaystyle \int_c^d dy\,\theta\left(1-\frac{y}{\eta}\right) C\left(\frac{y}{\eta}\right) w_{\alpha}^{(k)}\left(\frac{x_\beta}{y}\right)=\int_c^{\bar{d}} dy\,C\left(\frac{y}{\eta}\right) w_{\alpha}^{(k)}\left(\frac{x_\beta}{y}\right)\\
\\
 &=&\displaystyle 
\int_c^{\bar{d}} dy\,\left[\frac{1}{(1-y/\eta)_+}R\left(\frac{y}{\eta}\right)+\delta\left(1-\frac{y}{\eta}\right)L\right] w_{\alpha}^{(k)}\left(\frac{x_\beta}{y}\right)\\
\\
&=& \displaystyle \int_c^{\bar{d}} dy\frac{1}{1-y/\eta}\left[R\left(\frac{y}{\eta}\right)w_{\alpha}^{(k)}\left(\frac{x_\beta}{y}\right)-R(1)w_{\alpha}^{(k)}\left(\frac{x_\beta}{\eta}\right)\theta(\bar{d}-\eta)\right]\\
\\
&+&\displaystyle \left[R(1)\ln\left(1-\frac{c}{\eta}\right)\theta(\bar{d}-\eta)+L \right]\eta w_{\alpha}^{(k)}\left(\frac{x_\beta}{\eta}\right)\,.
\end{array}
\end{equation}
where we have redefined:
\begin{equation}
\bar{d} = \mbox{min}(\eta,x_\beta/x_{\alpha-k})\,.
\end{equation}
Finally, changing the integration variable in $z=y/\eta$, we have:
\begin{equation}
\begin{array}{rcl}
\Gamma_{\beta\alpha} &=& \eta\displaystyle \int_{c/\eta}^{\bar{d}/\eta} dz\frac{1}{1-z}\left[R\left(z\right)w_{\alpha}^{(k)}\left(\frac{x_\beta}{\eta z}\right)-R(1)w_{\alpha}^{(k)}\left(\frac{x_\beta}{\eta}\right)\theta(\bar{d}-\eta)\right]\\
\\
&+&\displaystyle \eta\left[R(1)\ln\left(1-\frac{c}{\eta}\right)\theta(\bar{d}-\eta)+L \right]w_{\alpha}^{(k)}\left(\frac{x_\beta}{\eta}\right)\,.
\end{array}
\end{equation}

Explicit expressions for the functions $R$ and $L$ for the structure
functions $xF_1$, $F_2$ and $xF_3$ can be extracted Appendix C of
Ref.~\cite{Kretzer:1998ju}. The relative expressions for $F_L$ can be
constructed using the usual relation:
\begin{equation}
F_L(x,Q) = F_2(x,Q) - 2xF_1(x,Q)\,.
\end{equation}

Using the same notation of Ref.~\cite{Kretzer:1998ju} and dropping an
overall factor $\alpha_s/4\pi$, we can write the
$\mathcal{O}(\alpha_s)$ contribution to the ``reduced'' structure
functions as:
\begin{equation}
x\hat{\mathcal{F}}_i^{QS^{(1)}} = x
\int_\chi^1\frac{dz}{z}
2\hat{H}_i^{q}(z)c\left(\frac{\chi}{z}\right)\quad\mbox{with}\quad i=1,2,3\,,
\end{equation}
and:
\begin{equation}\label{KretzerCFs}
2\hat{H}_i^{q}(z)= 2C_F\left[(S_i+V_i)\delta(1-z)+\frac{(1-z)}{(1-z)_+}\frac{\hat{s}-m_2^2}{8 \hat{s}}N_i^{-1}\hat{f}_i^{Q}(z)\right]\,.
\end{equation}
The standard structrure functions can be recovered just by applying
the correct kinematic factors. We will do this in the next section
where we will consider the NC and the CC sectors separately.

It is now easy to identify the function $C(z)$ in
eq.~(\ref{MassiveIntegral0}) with $2\hat{H}_i^{q}(z)$. As a
consequence, we also have that the function $\widetilde{C}(y)$ in
eq.~(\ref{MassiveIntegral}) is equal to:
\begin{equation}
\widetilde{C}(y) = \theta\left(1-\frac{y}{\eta}\right) 2\hat{H}_i^{q}\left(\frac{y}{\eta}\right)\,.
\end{equation}
Finally, comparing eq.~(\ref{KretzerCFs}) to eq.~(\ref{CoeffFuncs}),
we can easily make the following identifications:
\begin{equation}
\begin{array}{l}
\displaystyle R(z) = 2C_F (1-z)\frac{\hat{s}-m_2^2}{8
  \hat{s}}N_i^{-1}\hat{f}_i^{Q}(z)\,,\\
\\
\displaystyle L = 2C_F (S_i+V_i)\,.
\end{array}
\end{equation}



In the following, we will treat the NC and the CC cases separately,
showing how to implement the $\mathcal{O}(\alpha_s)$ corrections to
the relative structure functions keeping into account all the relevant
kinematic factors.

%It can be shown that:
%\begin{equation}
%\left[g(x)f(x)\right]_+= \left[g(x) \right]_+ f(x) -
%\delta(1-x)\int_0^1 dy \left[g(y) \right]_+ f(y)
%\end{equation}








\section{The FONLL Structure Functions}\label{FONNLSF}

Once the inclusion of the IC into the massive sectors has been
established, one can construct the FONLL structure functions using the
usual recipe but now including the additional contributions. Calling
$F_i^{\rm FONLL}$ the usual FONLL structure functions without IC and
$F_i^{\rm FONLL,IC}$ the structure function with IC, the relation is:
\begin{equation}\label{FONLL_mIC}
\begin{array}{rcl}
\displaystyle F_i^{\rm FONLL,IC} & = & \displaystyle F_i^{\rm FF} +
F_i^{\rm FF,IC} + D(Q^2) \left[ F_i^{\rm ZM} - F_i^{\rm FF0} - F_i^{\rm FF0,IC}
\right]\\
\\
                                              & = & \displaystyle
                                              F_i^{\rm FONLL} +
                                              \left[F_i^{\rm FF,IC} - D(Q^2) F_i^{\rm FF0,IC}\right] = F_i^{\rm FONLL} + \Delta F_i^{\rm FONLL,IC}
\end{array}
\end{equation}
where $D(Q^2)$ is a damping factor needed to quench undesired possibly
large subleading terms at small energies. In the rest of theses notes
I will concentrate on the implementation of the $\Delta F_i^{\rm
  FONLL,IC}$ in {\tt APFEL}.


\section{The Implementation}

At $\mathcal{O}(\alpha_s^0)$ there is no convolution between PDFs and
coefficient functions and the charm PDFs appear directly in the
expressions. According to whether one considers CC or NC
heavy-quark-initiated processes, PDFs enter either as $xc(x)$, where
$x$ is the measured Bjorken variable, or as $\chi c(\chi)$, where
$\chi$ is the rescaled variable defined in eq.~(\ref{RescaledX}). Now,
in order to achive a proper implementation of the FONLL scheme in {\tt
  APFEL}, I need to know all the component of the struncture functions
(massive and massless) on the same $x$-space interpolation grid,
defined as $\{x_{\alpha}\},\,,\alpha\in=0,\dots,N_x$.  At LO, this
essentially means knowing both $xc(x)$ and $\chi c(\chi)$ on the same
grid. But choosing to tabulate $xc(x)$, such that in the CC case:
\begin{equation}
F^{\rm CC}(x_\alpha)\propto x_\alpha c(x_\alpha)=\tilde{c}(x_\alpha)
=\sum_{\beta=0}^{N_x} \delta_{\alpha\beta} \tilde{c}(x_{\beta})\,,
\end{equation}
in the NC case, using the usual interpolation formula, the structure
function can be expanded as:
\begin{equation}\label{interpolationF}
F^{\rm NC}(x_\alpha)\propto \chi(x_\alpha) c(\chi(x_\alpha)) = \tilde{c}(\chi(x_\alpha)) = \sum_{\beta=0}^{N_x} w_{\beta}^{(k)}(\chi(x_\alpha)) 
\tilde{c}(x_{\beta}) = \sum_{\beta=0}^{N_x} w_{\beta}^{(k)}\left(\frac{x_\alpha}{\eta}\right) 
\tilde{c}(x_{\beta})\,.
\end{equation}
As a consequence, in the {\tt APFEL} framework, quantities to store
are $\delta_{\alpha\beta}$ and $w_{\beta}^{(k)}(x_\alpha/\eta)$ to be
combined in a proper way to the other coefficient functions. First of
all, let us compute case by case the quantity $\Delta F_i^{\rm
  FONLL,IC}$. In the NC case one has:
\begin{subequations}\label{LONCgrid}
\begin{equation}
\begin{array}{rcl}
\displaystyle \Delta F_2^{\rm  FONLL,IC}(x_\alpha) &=&\displaystyle 
\sum_{\beta=0}^{N_x}B_c\left[(2-\eta)
  w_{\beta}^{(k)}\left(\frac{x_\alpha}{\eta}\right) - D(Q^2)\delta_{\alpha\beta}\right]\tilde{c}^+(x_{\beta})
\end{array}
\end{equation}
\begin{equation}
\begin{array}{rcl}
\displaystyle x_\alpha \Delta F_3^{\rm  FONLL,IC}(x_\alpha) & = &\displaystyle 
\sum_{\beta=0}^{N_x}D_c\left[2\eta
  w_{\beta}^{(k)}\left(\frac{x_\alpha}{\eta}\right) - D(Q^2)2\delta_{\alpha\beta}\right]\tilde{c}^-(x_{\beta})
\end{array}
\end{equation}
\begin{equation}
\begin{array}{rcl}
\displaystyle \Delta F_L^{\rm  FONLL,IC}(x_\alpha) &=& \displaystyle\sum_{\beta=0}^{N_x} B_c\left[1-\frac{1}{2}\left(1-\frac{\tilde{B}_c}{B_c}\right)\right]
4\frac{1-\eta}{2-\eta} w_{\beta}^{(k)}\left(\frac{x_\alpha}{\eta}\right)\tilde{c}^+(x_{\beta})
\end{array}
\end{equation}
\end{subequations}

Finally, the CC case is slightly simpler:
\begin{subequations}\label{LOCCgrid}
\begin{equation}
\Delta F_2^{\nu,\rm FONLL,IC}(x_\alpha )=
\sum_{\beta=0}^{N_x} 2(|V_{cd}|^2+|V_{cs}|^2) \left[\left(1+\lambda\right) - D(Q^2)\right]
\delta_{\alpha\beta}\tilde{\overline{c}}(x_\beta )
\end{equation}
\begin{equation}
x_\alpha\Delta F_3^{\nu,\rm FONLL,IC}(x_\alpha )=
\sum_{\beta=0}^{N_x} 2(|V_{cd}|^2+|V_{cs}|^2) \left[1 -D(Q^2)\right] \delta_{\alpha\beta}\tilde{\overline{c}}(x_\beta )
\end{equation}
\begin{equation}
\Delta F_L^{\nu,\rm FONLL,IC}(x_\alpha ) = \sum_{\beta=0}^{N_x} 2(|V_{cd}|^2+|V_{cs}|^2)\lambda
\delta_{\alpha\beta}\tilde{\overline{c}}(x_\beta )
\end{equation}
\end{subequations}
and:
\begin{subequations}\label{CCnb}
\begin{equation}
\Delta F_2^{\overline{\nu},\rm FONLL,IC}(x_\alpha )= \sum_{\beta=0}^{N_x} 2(|V_{cd}|^2+|V_{cs}|^2) \left[\left(1+\lambda\right)-D(Q^2)\right]\delta_{\alpha\beta}\tilde{c}(x_\beta )
\end{equation}
\begin{equation}
x_\alpha\Delta F_3^{\overline{\nu},\rm FONLL,IC}(x_\alpha )= \sum_{\beta=0}^{N_x} 2(|V_{cd}|^2+|V_{cs}|^2) \left[1-D(Q^2)\right]\delta_{\alpha\beta}\tilde{c}(x_\beta )
\end{equation}
\begin{equation}
\Delta F_L^{\overline{\nu},\rm FONLL,IC}(x_\alpha )=
\sum_{\beta=0}^{N_x} 2(|V_{cd}|^2+|V_{cs}|^2) \lambda \delta_{\alpha\beta}\tilde{c}(x_\beta )
\end{equation}
\end{subequations}

Now, since structure functions in {\tt APFEL} are expressed in the
so-called evolution basis $\{\Sigma,g,V,V_3,\dots\}$, we only need to
re-express the charm PDFs in terms of the distributions in the
evolution basis. In particular, it is easy to show that:
\begin{equation}
c^+=\frac{1}{6}\Sigma - \frac{1}{4}T_{15}+\frac{1}{20}T_{24}+\frac{1}{30}T_{35}\,,
\end{equation}
and:
\begin{equation}
c^-=\frac{1}{6}V - \frac{1}{4}V_{15}+\frac{1}{20}V_{24}+\frac{1}{30}V_{35}\,.
\end{equation}
In addition:
\begin{equation}\label{SingleCharms}
c = \frac{1}{2}(c^+ + c^-)\quad\mbox{and}\quad \overline{c} = \frac{1}{2}(c^+ - c^-)\,.
\end{equation}

It should finally be noticed that the LO coefficient functions on the
grid as written in eqs.~(\ref{LONCgrid}) and~(\ref{LOCCgrid}) are
non-singlet like and as such should be treated.

In the NC sector, the IC contributions to the massive stucture
functions represent the only possible non-singlet
contribution(\footnote{This is essentially due to the fact that,
  requiring that the incoming photon (or $Z$) only couples to the
  charm, the photon vertex is never directly connected with the light
  initial state.}). As a consequence, the IC fills the non-singlet
``slot'' and does not interfere with the non-IC part making the
implementation in {\tt APFEL} easier. Unfortunately this is not the
case in the CC sector where instead the IC contribution overlaps with
the non-IC one in a non-trivial way. In fact, even at LO the CC IC
diagrams have a different kinematics as compared to the non-IC ones
and thus their combination on the interpolation grid is not
strightforward. What we need to do in {\tt APFEL} is creating a new
\textit{ad hoc} slot for the IC contrbutions in such a way that it
does not interfere with the other contributions(\footnote{In order not
  to burden {\tt APFEL} with any additional big array, we exploit the
  fact that the massive CC coefficient functions are presently known
  up to $\mathcal{O}(\alpha_s)$ and up to this order no pure-singlet
  contribution is present. We then ``artificially'' place the IC
  contribution in the pure-singlet slot bearing in mind that if ever
  the $\mathcal{O}(\alpha_s^2)$ corrections to this process, which
  contain a pure-singlet contribution, will be computed, we will need
  to move the IC contribution in a real new slot.}).

To conclude the section on LO implementation, we mention that, since
the contributions reported above are to be included to the $N_F=3$
massive scheme (be it full or asymptotic), heavy quark PDFs different
from charm, \textit{i.e.} bottom and top, do not contribute. As a
consequence, it turns out that:
\begin{equation}
\begin{array}{l}
T_{24} = T_{35} = \Sigma\,,\\
V_{24} = V_{35} = V\,,
\end{array}
\end{equation}
and thus eq.~(\ref{SingleCharms}) can be written as:
\begin{equation}
c = \frac{1}{8}(\Sigma - T_{15} + V - V_{15})\quad\mbox{and}\quad \overline{c} = \frac{1}{8}(\Sigma - T_{15} - V + V_{15})\,.
\end{equation}





\begin{thebibliography}{alp}

%\cite{Kretzer:1998ju}
\bibitem{Kretzer:1998ju}
  S.~Kretzer and I.~Schienbein,
  %``Heavy quark initiated contributions to deep inelastic structure functions,''
  Phys.\ Rev.\ D {\bf 58} (1998) 094035
  [hep-ph/9805233].
  %%CITATION = HEP-PH/9805233;%%

\end{thebibliography}


\end{document}
