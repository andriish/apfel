%% LyX 2.0.3 created this file.  For more info, see http://www.lyx.org/.
%% Do not edit unless you really know what you are doing.
\documentclass[twoside,english]{paper}
\usepackage{lmodern}
\renewcommand{\ttdefault}{lmodern}
\usepackage[T1]{fontenc}
\usepackage[latin9]{inputenc}
\usepackage[a4paper]{geometry}
\geometry{verbose,tmargin=3cm,bmargin=2.5cm,lmargin=2cm,rmargin=2cm}
\usepackage{color}
\usepackage{babel}
\usepackage{float}
\usepackage{bm}
\usepackage{amsthm}
\usepackage{amsmath}
\usepackage{amssymb}
\usepackage{graphicx}
\usepackage{esint}
\usepackage[unicode=true,pdfusetitle,
 bookmarks=true,bookmarksnumbered=false,bookmarksopen=false,
 breaklinks=false,pdfborder={0 0 0},backref=false,colorlinks=false]
 {hyperref}
\usepackage{breakurl}

\makeatletter

%%%%%%%%%%%%%%%%%%%%%%%%%%%%%% LyX specific LaTeX commands.
%% Because html converters don't know tabularnewline
\providecommand{\tabularnewline}{\\}

%%%%%%%%%%%%%%%%%%%%%%%%%%%%%% Textclass specific LaTeX commands.
\numberwithin{equation}{section}
\numberwithin{figure}{section}

%%%%%%%%%%%%%%%%%%%%%%%%%%%%%% User specified LaTeX commands.
\usepackage{babel}

\@ifundefined{showcaptionsetup}{}{%
 \PassOptionsToPackage{caption=false}{subfig}}
\usepackage{subfig}
\makeatother

\usepackage{listings}

\makeindex

\begin{document}

\title{Intrinsic Charm Implementation}

\author{Valerio Bertone$^{a}$}

\institution{$^{a}$PH Department, TH Unit, CERN, CH-1211 Geneva 23, Switzerland}
\maketitle

\begin{abstract}
In these set of notes I will describe the strategy to include the
intrinsic charm (IC) contribution to the FONLL structure functions as
implemented in {\tt APFEL}. I will first consider the massive sector
(and its massless limit), where the IC implies the presence of the
charm in the initial state with the consequence of additional diagrams
to be include in the computation. I will the consider the massless
sector where the presence of an IC implies a retratment of the PDF
matching conditions at the charm threshold.
\end{abstract}

\tableofcontents{}

\section{Intrinsic Charm Contribution to the Massive Structure Functions}

\subsection{Order $\alpha_s^0$ Contributions}

Assuming the presence of IC in the proton, the massive structure
functions with $N_f=3$ light flavours acquire a further contribution
coming from the presence of a massive charm in the initial state. As a
consequence, the massive structure functions get a term that is
proportional to a \textit{static} charm PDF, $i.e.$ a PDF that, being
massive, does not evolve according to the DGLAP equantion. Such a
contribution starts already at order $\alpha_s^0$ and has the novel
effect to ``allign'' the massive scheme to the massless scheme in
terms of power counting because, contrary to what happens without IC,
the two sectors start at $\alpha_s^0$.

In order to write explicitly the form of such LO contributions to the
DIS structure functions, I consider eq.~(2) of \cite{Kretzer:1998ju}
where the function $Q_1$ schould be indentified with the charm PDF. It
should be noticed that in the $N_f=3$ scheme, such PDF does not obey
the DGLAP equation because, due to the presence of the mass of the
charm $m_c$, no large collinear logarithms appear in the calculation
and thus there is no need to resum them.

From eq.~(2) of \cite{Kretzer:1998ju} one reads that the
$\mathcal{O}(\alpha_s^0)$ IC contributions to the massive structrure
functions are given by:
\begin{subequations}\label{GeneralSF}
\begin{equation}
F_1^{\rm FF,IC}(x,Q^2) = \frac{S_+\Sigma_{++}-2m_1m_2S_-}{2\Delta}c(\chi)
\end{equation}
\begin{equation}
F_2^{\rm FF,IC}(x,Q^2) = \frac{S_+\Delta}{2Q^2}2xc(\chi)
\end{equation}
\begin{equation}
xF_3^{\rm FF,IC}(x,Q^2) = 2Rxc(\chi)
\end{equation}
\end{subequations}
where $m_1$ and $m_2$ are the masses of the incoming and outcoming
quarks, respectively, while $\Delta\equiv\Delta(m_1^2,m_2^2,-Q^2)$
with the function $\Delta$ defined as:
\begin{equation}
  \Delta(a,b,c)=\sqrt{a^2+b^2+c^2 -2(ab+ac+bc)}
\end{equation}
and:
\begin{equation}
\Sigma_{\pm\pm}=Q^2\pm m_2^2\pm m_1^2
\end{equation}
\begin{equation}\label{Rescaledx}
\chi=\frac{x}{2Q^2}(\Sigma_{+-}+\Delta)
\end{equation}
The quantities $S_\pm$ and $R_\pm$, instead, are linked to the EW
couplings and depend on the vector boson that strikes the heavy quark
with mass $m_1$ in the initial state. Notice that in
eq.~(\ref{GeneralSF}) the PDF $c$ does not depend on any factorization
scale and, as mentioned before, the reason is that it is a static
distribution of non-perturbative origin that does not evolve according
to the DGLAP equation.

In practice, assuming the presence of IC in the proton, the massive
(FF) structure functions become:
\begin{equation}
F_i^{\rm FF}(x,Q^2)\longrightarrow F_i^{\rm FF}(x,Q^2)  + F_i^{\rm
  FF,IC}(x,Q^2) \quad\mbox{with}\quad i =1,2,3
\end{equation}

Now, for a purely \textit{electromagnetic} process, where only a
$\gamma$ strikes the charm, one has:
\begin{equation}
S_+=S_-=e_c^2\quad\mbox{and}\quad R=0
\end{equation}
Moreover, in this case both the incoming and the outcoming quarks are
of the same flavour (charm) therefore we have $m_1=m_2=m_c$. Under
this conditions one finds:
\begin{subequations}
\begin{equation}
F_1^{\rm FF,IC}(x,Q^2)= \frac{1}{2\sqrt{1+4\lambda}}e_c^2c(\chi)
\end{equation}
\begin{equation}
F_2^{\rm FF,IC}(x,Q^2)=\left(\sqrt{1+4\lambda}\right)e_c^2xc(\chi)
\end{equation}
\begin{equation}
xF_3^{\rm FF,IC}(x,Q^2)= 0
\end{equation}
\end{subequations}
with:
\begin{equation}\label{RescaledX}
\chi = \frac{x}{2}\left(1+\sqrt{1+4\lambda}\right) = \frac{x}{\eta}\,,
\end{equation}
where I have defined:
\begin{equation}\label{Rescaledx}
\eta=\frac{2Q^2}{\Sigma_{+-}+\Delta} = 2\left(1+\sqrt{1+4\lambda}\right)^{-1}=2\left(1+\sqrt{1+4\lambda}\right)^{-1}\,,
\end{equation}
with $\lambda=\frac{m_c^2}{Q^2}$.

For a \textit{neutral current} process, where all the $\gamma$, the
$Z$ and the interference $\gamma Z$ contributions are considered, one
has:
\begin{equation}
S_+=S_-=B_c=e_c^2-2e_cV_eV_cP_Z+(V_e^2+A_e^2)(V_c^2+A_c^2)P_Z^2\quad\mbox{and}\quad R=D_c=-2e_c A_cA_eP_Z+4V_cA_cV_eA_eP_Z^2
\end{equation}
with:
\begin{equation}
V_c = \frac12-\frac43\sin^2\theta_W\quad\mbox{and}\quad A_c=\frac12
\end{equation}
and
\begin{equation}
V_e = -\frac12+2\sin^2\theta_W\quad\mbox{and}\quad A_e=-\frac12
\end{equation}
the vector and the axial coupling of charm and electron to the $Z$ and
where:
\begin{equation}
P_Z=\frac1{4\sin^2\theta_W(1-\sin^2\theta_W)}\frac{Q^2}{Q^2+M_Z^2}
\end{equation}
Here, exactly as in the electromagnetic case, $m_1=m_2=m_c$ so that
one ends up with:
\begin{subequations}
\begin{equation}
F_1^{\rm FF,IC}(x,Q^2)= \frac{1}{2\sqrt{1+4\lambda}}B_c c(\chi)\,,
\end{equation}
\begin{equation}
F_2^{\rm FF,IC}(x,Q^2)=\left(\sqrt{1+4\lambda}\right)B_cxc(\chi)\,,
\end{equation}
\begin{equation}
xF_3^{\rm FF,IC}(x,Q^2)= 2D_cxc(\chi)\,.
\end{equation}
\end{subequations}

Finally, for a \textit{charged current} process, where a charged boson
$W^\pm$ strikes the charm, one has:
\begin{equation}
S_+=S_-=2|V_{cs}|^2\quad\mbox{and}\quad R=|V_{cs}|^2
\end{equation}
if the outcoming quark is a strange or an anti-strange, and:
\begin{equation}
S_+=S_-=2|V_{cd}|^2\quad\mbox{and}\quad R=|V_{cd}|^2
\end{equation}
if the outcoming quark is a down or an anti-down. 

In this case $m_1=m_c$ but $m_2=0$ with the consequence that:
\begin{subequations}
\begin{equation}
F_1^{\rm FF,IC}(x,Q^2)= |V_{cj}|^2c(x)
\end{equation}
\begin{equation}
F_2^{\rm FF,IC}(x,Q^2)= 2\left(1+\lambda\right)|V_{cj}|^2xc(x)
\end{equation}
\begin{equation}
xF_3^{\rm FF,IC}(x,Q^2)= 2|V_{cj}|^2xc(x)
\end{equation}
\end{subequations}
with $j=d,s$. Note that in this case $\eta=1$ and thus $\chi=x$.

In order to take into account the possible contributions due to
intrinsic charm, one has to consider all diagrams contributing to a
given process. As far as the neutral current (electromagnetic) case is
concerned, one has to consider also the presence of $\overline{c}$ in
the proton which, summed to the contribution of the $c$, gives:
\begin{subequations}\label{NCEM}
\begin{equation}
F_1^{\rm FF,IC}(x,Q^2)= \frac{1}{2\sqrt{1+4\lambda}}B_c c^+(\chi)
\end{equation}
\begin{equation}
F_2^{\rm FF,IC}(x,Q^2)=\frac{2\sqrt{1+4\lambda}}{1+\sqrt{1+4\lambda}}B_c\chi c^+(\chi)
\end{equation}
\begin{equation}
xF_3^{\rm FF,IC}(x,Q^2)=\frac{4}{1+\sqrt{1+4\lambda}}D_c\chi c^-(\chi)
\end{equation}
\end{subequations}
where:
\begin{equation}
c^{\pm}=c\pm\overline{c}
\end{equation}
therefore:
\begin{equation}
F_L^{\rm FF,IC}(x,Q^2) = F_2^{\rm FF,IC}(x,Q^2) -2x F_1^{\rm
  FF,IC}(x,Q^2)  = \frac{8\lambda}{1+4\lambda+\sqrt{1+4\lambda}}B_c \chi c^+(\chi)
\end{equation}

In the charged current case, instead, one has to distinguish between
neutrino and anti-neutrino scattering. The neutrino scattering gives
as a result the following structure functions:
\begin{subequations}\label{CCnu}
\begin{equation}
F_1^{\nu,\rm FF,IC}(x,Q^2)= (|V_{cd}|^2+|V_{cs}|^2)\overline{c}(x)
\end{equation}
\begin{equation}
F_2^{\nu,\rm FF,IC}(x,Q^2)= 2\left(1+\lambda\right)(|V_{cd}|^2+|V_{cs}|^2)x\overline{c}(x)
\end{equation}
\begin{equation}
xF_3^{\nu,\rm FF,IC}(x,Q^2)= 2(|V_{cd}|^2+|V_{cs}|^2)x\overline{c}(x)
\end{equation}
\begin{equation}
F_L^{\nu,\rm FF,IC}(x,Q^2) = 2\lambda(|V_{cd}|^2+|V_{cs}|^2)x\overline{c}(x)
\end{equation}
\end{subequations}
The anti-neutrino scattering instead gives as a result the following
structure functions:
\begin{subequations}\label{CCnb}
\begin{equation}
F_1^{\overline{\nu},\rm FF,IC}(x,Q^2)= (|V_{cd}|^2+|V_{cs}|^2)c(x)
\end{equation}
\begin{equation}
F_2^{\overline{\nu},\rm FF,IC}(x,Q^2)= 2\left(1+\lambda\right)(|V_{cd}|^2+|V_{cs}|^2)xc(x)
\end{equation}
\begin{equation}
xF_3^{\overline{\nu},\rm FF,IC}(x,Q^2)= 2(|V_{cd}|^2+|V_{cs}|^2)xc(x)
\end{equation}
\begin{equation}
F_L^{\overline{\nu},\rm FF,IC}(x,Q^2)= 2\lambda(|V_{cd}|^2+|V_{cs}|^2)xc(x)
\end{equation}
\end{subequations}

It should be pointed out that since the charm quark belongs to the sea
it is symmetric under isospin symmetry and thus all the above
structure functions are the same for proton and neutron.

\subsection{Massless Limit}

When implementing the FONLL scheme, one also needs to consider the
massless limit of the massive structuture functions (FF0). To this
end, we just need to take the limit for $m_c\rightarrow 0$ of
eqs.~(\ref{NCEM}), (\ref{CCnu}) and~(\ref{CCnb}). Considering that:
\begin{equation}
\chi \mathop{\longrightarrow}_{m_c\rightarrow0} x\,,
\end{equation}
one finds :
\begin{subequations}\label{NCEM}
\begin{equation}
F_1^{\rm FF0,IC}(x,Q^2)= \frac{1}{2}B_c c^+(x)
\end{equation}
\begin{equation}
F_2^{\rm FF0,IC}(x,Q^2)=B_cxc^+(x)
\end{equation}
\begin{equation}
xF_3^{\rm FF0,IC}(x,Q^2)= 2D_cxc^-(x)
\end{equation}
\begin{equation}
F_L^{\rm FF0,IC}(x,Q^2)= 0
\end{equation}
\end{subequations}
and:
\begin{subequations}\label{CCnu}
\begin{equation}
F_1^{\nu,\rm FF0,IC}(x,Q^2)= (|V_{cd}|^2+|V_{cs}|^2)\overline{c}(x)
\end{equation}
\begin{equation}
F_2^{\nu,\rm FF0,IC}(x,Q^2)= 2(|V_{cd}|^2+|V_{cs}|^2)x\overline{c}(x)
\end{equation}
\begin{equation}
xF_3^{\nu,\rm FF0,IC}(x,Q^2)= 2(|V_{cd}|^2+|V_{cs}|^2)x\overline{c}(x)
\end{equation}
\begin{equation}
F_L^{\nu,\rm FF0,IC}(x,Q^2)= 0
\end{equation}
\end{subequations}
and:
\begin{subequations}\label{CCnb}
\begin{equation}
F_1^{\overline{\nu},\rm FF0,IC}(x,Q^2)= (|V_{cd}|^2+|V_{cs}|^2)c(x)
\end{equation}
\begin{equation}
F_2^{\overline{\nu},\rm FF0,IC}(x,Q^2)= 2(|V_{cd}|^2+|V_{cs}|^2)xc(x)
\end{equation}
\begin{equation}
xF_3^{\overline{\nu},\rm FF0,IC}(x,Q^2)= 2(|V_{cd}|^2+|V_{cs}|^2)xc(x)
\end{equation}
\begin{equation}
F_L^{\overline{\nu},\rm FF0,IC}(x,Q^2)= 0
\end{equation}
\end{subequations}

\section{The FONLL Structure Functions}\label{FONNLSF}

Once the inclusion of the IC into the massive sectors has been
established, one can construct the FONLL structure functions using the
usual recipe but now including the additional contributions. Calling
$F_i^{\rm FONLL}$ the usual FONLL structure functions without IC and
$F_i^{\rm FONLL,IC}$ the structure function with IC, the relation is:
\begin{equation}\label{FONLL_mIC}
\begin{array}{rcl}
\displaystyle F_i^{\rm FONLL,IC} & = & \displaystyle F_i^{\rm FF} +
F_i^{\rm FF,IC} + D(Q^2) \left[ F_i^{\rm ZM} - F_i^{\rm FF0} - F_i^{\rm FF0,IC}
\right]\\
\\
                                              & = & \displaystyle
                                              F_i^{\rm FONLL} +
                                              \left[F_i^{\rm FF,IC} - D(Q^2) F_i^{\rm FF0,IC}\right] = F_i^{\rm FONLL} + \Delta F_i^{\rm FONLL,IC}
\end{array}
\end{equation}
where $D(Q^2)$ is a damping factor needed to quench undesired possibly
large subleading terms at small energies. In the rest of theses notes
I will concentrate on the implementation of the $\Delta F_i^{\rm
  FONLL,IC}$ in {\tt APFEL}.


\section{The Implementation}

At $\mathcal{O}(\alpha_s^0)$ there is no convolution between PDFs and
coefficient functions and the charm PDFs appear directly in the
expressions. According to whether one considers CC or NC
heavy-quark-initiated processes, PDFs enter either as $xc(x)$, where
$x$ is the measured Bjorken variable, or as $\chi c(\chi)$, where
$\chi$ is the rescaled variable defined in eq.~(\ref{RescaledX}). Now,
in order to achive a proper implementation of the FONLL scheme in {\tt
  APFEL}, I need to know all the component of the struncture functions
(massive and massless) on the same $x$-space interpolation grid,
defined as $\{x_{\alpha}\},\,,\alpha\in=0,\dots,N_x$.  At LO, this
essentially means knowing both $xc(x)$ and $\chi c(\chi)$ on the same
grid. But choosing to tabulate $xc(x)$, such that in the CC case:
\begin{equation}
F^{\rm CC}(x_\alpha)\propto x_\alpha c(x_\alpha)=\tilde{c}(x_\alpha)
=\sum_{\beta=0}^{N_x} \delta_{\alpha\beta} \tilde{c}(x_{\beta})\,,
\end{equation}
in the NC case, using the usual interpolation formula, the structure
function can be expanded as:
\begin{equation}\label{interpolationF}
F^{\rm NC}(x_\alpha)\propto \chi(x_\alpha) c(\chi(x_\alpha)) = \tilde{c}(\chi(x_\alpha)) = \sum_{\beta=0}^{N_x} w_{\beta}^{(k)}(\chi(x_\alpha)) 
\tilde{c}(x_{\beta}) = \sum_{\beta=0}^{N_x} w_{\beta}^{(k)}\left(\frac{x_\alpha}{\eta}\right) 
\tilde{c}(x_{\beta})\,.
\end{equation}
As a consequence, in the {\tt APFEL} framework, quantities to store
are $\delta_{\alpha\beta}$ and $w_{\beta}^{(k)}(x_\alpha/\eta)$ to be
combined in a proper way to the other coefficient functions. First of
all, let us compute case by case the quantity $\Delta F_i^{\rm
  FONLL,IC}$. In the NC case one has:
\begin{subequations}
\begin{equation}
\begin{array}{rcl}
\displaystyle \Delta F_2^{\rm  FONLL,IC}(x_\alpha) &=&\displaystyle B_c
\sum_{\beta=0}^{N_x}\left[\frac{2\sqrt{1+4\lambda}}{1+\sqrt{1+4\lambda}}
  w_{\beta}^{(k)}\left(\frac{x_\alpha}{\eta}\right) - D(Q^2)\delta_{\alpha\beta}\right]\tilde{c}^+(x_{\beta})
\end{array}
\end{equation}
\begin{equation}
\begin{array}{rcl}
\displaystyle x_\alpha \Delta F_3^{\rm  FONLL,IC}(x_\alpha) & = &\displaystyle 2D_c
\sum_{\beta=0}^{N_x}\left[\frac{2}{1+\sqrt{1+4\lambda}}
  w_{\beta}^{(k)}\left(\frac{x_\alpha}{\eta}\right) - D(Q^2)\delta_{\alpha\beta}\right]\tilde{c}^-(x_{\beta})
\end{array}
\end{equation}
\begin{equation}
\begin{array}{rcl}
\displaystyle \Delta F_L^{\rm  FONLL,IC}(x_\alpha) &=& \displaystyle
                                                       B_c \frac{8\lambda}{1+4\lambda+\sqrt{1+4\lambda}}
\sum_{\beta=0}^{N_x} w_{\beta}^{(k)}\left(\frac{x_\alpha}{\eta}\right)\tilde{c}^+(x_{\beta})
\end{array}
\end{equation}
\end{subequations}

Finally, the CC case is slightly simpler:
\begin{subequations}
\begin{equation}
\Delta F_2^{\nu,\rm FONLL,IC}(x_\alpha )=
2(|V_{cd}|^2+|V_{cs}|^2) \left[\left(1+\lambda\right) - D(Q^2)\right]
\sum_{\beta=0}^{N_x}\delta_{\alpha\beta}\tilde{\overline{c}}(x_\beta )
\end{equation}
\begin{equation}
x_\alpha\Delta F_3^{\nu,\rm FONLL,IC}(x_\alpha )=
2(|V_{cd}|^2+|V_{cs}|^2) \left[1 -D(Q^2)\right] \sum_{\beta=0}^{N_x}\delta_{\alpha\beta}\tilde{\overline{c}}(x_\beta )
\end{equation}
\begin{equation}
\Delta F_L^{\nu,\rm FONLL,IC}(x_\alpha ) = 2(|V_{cd}|^2+|V_{cs}|^2)\lambda
\sum_{\beta=0}^{N_x}\delta_{\alpha\beta}\tilde{\overline{c}}(x_\beta )
\end{equation}
\end{subequations}
and:
\begin{subequations}\label{CCnb}
\begin{equation}
\Delta F_2^{\overline{\nu},\rm FONLL,IC}(x_\alpha )= 2(|V_{cd}|^2+|V_{cs}|^2) \left[\left(1+\lambda\right)-D(Q^2)\right]\sum_{\beta=0}^{N_x}\delta_{\alpha\beta}\tilde{c}(x_\beta )
\end{equation}
\begin{equation}
x_\alpha\Delta F_3^{\overline{\nu},\rm FONLL,IC}(x_\alpha )= 2(|V_{cd}|^2+|V_{cs}|^2) \left[1-D(Q^2)\right]\sum_{\beta=0}^{N_x}\delta_{\alpha\beta}\tilde{c}(x_\beta )
\end{equation}
\begin{equation}
\Delta F_L^{\overline{\nu},\rm FONLL,IC}(x_\alpha )=
2(|V_{cd}|^2+|V_{cs}|^2) \lambda \sum_{\beta=0}^{N_x}\delta_{\alpha\beta}\tilde{c}(x_\beta )
\end{equation}
\end{subequations}

Now, since structure functions in {\tt APFEL} are expressed in the
so-called evolution basis $\{\Sigma,g,V,V_3,\dots\}$, we only need to
re-express the charm PDFs in terms of the distributions in the
evolution basis. In particular, it is easy to show that:
\begin{equation}
c^+=\frac{1}{6}\Sigma - \frac{1}{4}T_{15}+\frac{1}{20}T_{24}+\frac{1}{30}T_{35}\,,
\end{equation}
and:
\begin{equation}
c^-=\frac{1}{6}V - \frac{1}{4}V_{15}+\frac{1}{20}V_{24}+\frac{1}{30}V_{35}\,.
\end{equation}
In addition:
\begin{equation}
c = \frac{1}{2}(c^+ + c^-)\quad\mbox{and}\quad \overline{c} = \frac{1}{2}(c^+ - c^-)\,.
\end{equation}









\begin{thebibliography}{alp}

%\cite{Kretzer:1998ju}
\bibitem{Kretzer:1998ju}
  S.~Kretzer and I.~Schienbein,
  %``Heavy quark initiated contributions to deep inelastic structure functions,''
  Phys.\ Rev.\ D {\bf 58} (1998) 094035
  [hep-ph/9805233].
  %%CITATION = HEP-PH/9805233;%%

\end{thebibliography}


\end{document}
