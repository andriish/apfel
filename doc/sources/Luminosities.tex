%% LyX 2.0.2 created this file.  For more info, see http://www.lyx.org/.
%% Do not edit unless you really know what you are doing.
\documentclass[twoside,english]{article}
\usepackage{lmodern}
\usepackage[T1]{fontenc}
\usepackage[latin9]{inputenc}
\usepackage[a4paper]{geometry}
\geometry{verbose,tmargin=3cm,bmargin=2.5cm,lmargin=2cm,rmargin=2cm}
\usepackage{babel}
\usepackage{float}
\usepackage{bm}
\usepackage{amsmath}
\usepackage{esint}
\usepackage[unicode=true,pdfusetitle,
 bookmarks=true,bookmarksnumbered=false,bookmarksopen=false,
 breaklinks=false,pdfborder={0 0 0},backref=false,colorlinks=false]
 {hyperref}

\makeatletter

%%%%%%%%%%%%%%%%%%%%%%%%%%%%%% LyX specific LaTeX commands.
%% Because html converters don't know tabularnewline
\providecommand{\tabularnewline}{\\}

%%%%%%%%%%%%%%%%%%%%%%%%%%%%%% Textclass specific LaTeX commands.
%\numberwithin{equation}{section}

\makeatother

\begin{document}

\title{Notes on the Parton Luminosities}

\author{Valerio Bertone}
\maketitle
\begin{abstract}
In this document I describe the definition and the implementation of
the differential parton-luminosity functions.
\end{abstract}

\section{Definition}

The definition of the parton-luminosity functions is given by the
expression of the total hadronic cross section $\sigma$ in terms of
the parton ditribution functions $f_{i(j)}(x_{1,(2)},\mu)$ and the
partonic cross sections $\hat{\sigma}_{ij}$:
\begin{equation}\label{XsecX1X2}
\sigma = \sum_{ij}\int_0^1dx_1 \int_0^1dx_2 f_{i}(x_{1},\mu) f_{j}(x_{2},\mu) \hat{\sigma}_{ij}\,.
\end{equation}
Now, defining the kinematic variable $y$ (rapidity) and $M_X$
(invariant mass of the partonic final state, usually normalized to the
total center of mass squared energy energy $s$) as:
\begin{equation}
\begin{array}{l}
\displaystyle \tau = \frac{M_X^2}{s} = x_1x_2\\
\\
\displaystyle y = \frac12\ln\left(\frac{x_1}{x_2}\right)
\end{array}\,,
\end{equation}
one can express the integral in eq.~(\ref{XsecX1X2}) in terms of $y$
and $\tau$. In fact, by means of a change of variables and taking into
account the fact that the Jacobian $\partial
(y,\tau)/\partial(x_1,x_2)=1$, we find that:
\begin{equation}\label{XsecYM}
\sigma = \sum_{ij}\left(\int_{-\infty}^{0}dy \int_0^{e^{2y}} d\tau+\int_{0}^{+\infty}dy \int_0^{e^{-2y}} d\tau \right)
\frac{d^2\mathcal{L}_{ij}}{dy d\tau} \hat{\sigma}_{ij} = \sum_{ij}\int_0^{1} d\tau\left(\int_{0}^{-\frac12\ln\tau}dy +\int_{\frac12\ln\tau}^{0}dy \right)
\frac{d^2\mathcal{L}_{ij}}{dy d\tau} \hat{\sigma}_{ij}\,,
\end{equation}
where we have defined fully differential parton-luminosity functions as:
\begin{equation}
\frac{d^2\mathcal{L}_{ij}}{dy d\tau} = f_{i}\left(\sqrt{\tau}e^{y},\sqrt{s\tau}\right) f_{j}\left(\sqrt{\tau}e^{-y},\sqrt{s\tau}\right)\,,
\end{equation}
and where we have also set $\mu=M_X=\sqrt{s\tau}$.
Considering that the the differential parton-luminosity functions and
the partonic cross sections must be symmetric under $y\leftrightarrow
-y$, eq.~(\ref{XsecYM}) can be written as:
\begin{equation}\label{XsecYM1}
\sigma = \sum_{ij}\int_{-\infty}^{+\infty}dy \int_0^{e^{-2|y|}} d\tau
\frac{d^2\mathcal{L}_{ij}}{dy d\tau} \hat{\sigma}_{ij} = \sum_{ij}\int_0^{1} d\tau\int_{\frac12\ln\tau}^{-\frac12\ln\tau}dy
\frac{d^2\mathcal{L}_{ij}}{dy d\tau} \hat{\sigma}_{ij}\,,
\end{equation}
that can be written in a unique way as:
\begin{equation}\label{XsecYM2}
\sigma = \sum_{ij}\int_{-\infty}^{+\infty}dy \int_0^1 d\tau\,\theta\left(-y-\frac12\ln\tau\right) \theta\left(y-\frac12\ln\tau\right)
\frac{d^2\mathcal{L}_{ij}}{dy d\tau} \hat{\sigma}_{ij} \,.
\end{equation}
More in general, we can write the following equivalence at the level of
phase space as:
\begin{equation}\label{PhaseSpace}
\int_0^1dx_1 \int_0^1dx_2 = \int_{-\infty}^{+\infty}dy \int_0^1 d\tau  
\,\theta\left(-y-\frac12\ln\tau\right)
\theta\left(y-\frac12\ln\tau\right) =\int_0^1
d\tau \int_{-\infty}^{+\infty}dy \,\theta\left(e^{-2|y|}-\tau\right)\,.
\end{equation}

Now, let us compute the single differential parton-luminosity
functions integrating out one of the kinematic variables $y$ and
$\tau$. Let us start with $y$. In this case we have:
\begin{equation}
\begin{array}{rcl}
\displaystyle \frac{d\mathcal{L}_{ij}}{d\tau} &=&\displaystyle \sum_{ij}\int_{-\infty}^{+\infty}dy \int_0^1 d\tau\,\theta\left(-y-\frac12\ln\tau\right) \theta\left(y-\frac12\ln\tau\right)
\frac{d^2\mathcal{L}_{ij}}{dy d\tau} \delta(\tau-\tau')\\
\\
&=&\displaystyle \int_{\frac12\ln\tau}^{-\frac12\ln\tau}dy f_i\left(\sqrt{\tau'}e^{y},\sqrt{s\tau'}\right) f_j\left(\sqrt{\tau'}e^{-y},\sqrt{s\tau'}\right)\,,
\end{array}
\end{equation}
that, performing the change of variable $x=\sqrt{\tau'}e^{y}$ and
taking into account that $s\tau'=M_X^2$, becomes:
\begin{equation}
\Phi_{ij} = \frac{d\mathcal{L}_{ij}}{dM_X^2} =\frac1{s}\int_{\tau'}^1
\frac{dx}{x} f_{i}(x,M_X) f_{j}\left(\frac{\tau'}{x},M_X\right)\,.
\end{equation}

Now we integrate the same for $\tau$:
\begin{equation}\label{YLumiTau}
\begin{array}{rcl}
\displaystyle \Psi_{ij}=\frac{d\mathcal{L}_{ij}}{dy} &=&\displaystyle \sum_{ij} \int_0^1 d\tau\int_{-\infty}^{+\infty}dy\,\theta\left(e^{-2|y|}-\tau\right)
\frac{d^2\mathcal{L}_{ij}}{dy d\tau} \delta(y-y')\\
\\
&=&\displaystyle \int_{0}^{e^{-2|y'|}}d\tau f_i\left(\sqrt{\tau}e^{y'},\sqrt{s\tau}\right) f_j\left(\sqrt{\tau}e^{-y'},\sqrt{s\tau}\right)\,,
\end{array}
\end{equation}
that, defining $x=\sqrt{\tau}e^{y'}$ and assuming $y'\geq0$, becomes:
\begin{equation}\label{YLumi}
\Psi_{ij} = \frac{d\mathcal{L}_{ij}}{dy}  = 2e^{-2y'}\int_{0}^{e^{-y'}} dx\,x  f_{i}(x,\sqrt{s}xe^{-y'}) f_{j}(xe^{-2y'},\sqrt{s}xe^{-y'})\,.
\end{equation}

Eq.~(\ref{YLumi}) has the drawback that the integration range in $x$
goes down to zero. In addition, also the scale in which PDFs are
evaluate depends on $x$ linarly. This implies that performing the
integral requires the evaluation of PDFs in $x=\mu=0$ and this is
clearly impossible. However, for converge reasons, the invariant mass
of the final state is alway taken to be bigger than a give
threshold. At the LHC 14 TeV this threshold is typically of the order of a
few tens of GeV. Here, for explicative purposes, we take $M_{X,\rm cut}=10$ GeV
and this results in the cutoff $\tau_{\rm cut}\simeq 6 \cdot 10^{-7}$, so that
eq.~(\ref{YLumiTau}) becomes:
\begin{equation}\label{YLumiTau1}
\begin{array}{rcl}
\displaystyle \Psi_{ij}=\frac{d\mathcal{L}_{ij}}{dy} &=&\displaystyle
                                                         \int_{\tau_{\rm
                                                         cut}}^{e^{-2y'}}d\tau f_i\left(\sqrt{\tau}e^{y'},\sqrt{s\tau}\right) f_j\left(\sqrt{\tau}e^{-y'},\sqrt{s\tau}\right)\,,
\end{array}
\end{equation}
and thus:
\begin{equation}\label{YLumi2}
\Psi_{ij} = \frac{d\mathcal{L}_{ij}}{dy}  =
2e^{-2y'}\int_{\sqrt{\tau_{\rm cut}}e^{y'}}^{e^{-y'}} dx\,x  f_{i}(x,\sqrt{s}xe^{-y'}) f_{j}(xe^{-2y'},\sqrt{s}xe^{-y'})\,.
\end{equation}
Eq.~(\ref{YLumi2}) should allow for a practical implementation as the
minimum of the integration range is now of the order of $7\cdot 10^{-4}$ and the
minimum energy in which PDFs are avaluated is equal to
$M_{X,\rm cut}$. In addition, in order for the upper limit of the
integral in eq.~(\ref{YLumi2}) to be smaller than the lower one, one
should also require that:
\begin{equation}
y'\leq-\frac12\ln\frac{M_{X,\rm cut}}{\sqrt{s}}\simeq 3.6\,.
\end{equation}

\end{document}
