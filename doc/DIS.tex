%% LyX 2.0.3 created this file.  For more info, see http://www.lyx.org/.
%% Do not edit unless you really know what you are doing.
\documentclass[twoside,english]{paper}
\usepackage{lmodern}
\renewcommand{\ttdefault}{lmodern}
\usepackage[T1]{fontenc}
\usepackage[latin9]{inputenc}
\usepackage[a4paper]{geometry}
\geometry{verbose,tmargin=3cm,bmargin=2.5cm,lmargin=2cm,rmargin=2cm}
\usepackage{color}
\usepackage{babel}
\usepackage{float}
\usepackage{bm}
\usepackage{amsthm}
\usepackage{amsmath}
\usepackage{amssymb}
\usepackage{graphicx}
\usepackage{esint}
\usepackage[unicode=true,pdfusetitle,
 bookmarks=true,bookmarksnumbered=false,bookmarksopen=false,
 breaklinks=false,pdfborder={0 0 0},backref=false,colorlinks=false]
 {hyperref}
\usepackage{breakurl}

\makeatletter

%%%%%%%%%%%%%%%%%%%%%%%%%%%%%% LyX specific LaTeX commands.
%% Because html converters don't know tabularnewline
\providecommand{\tabularnewline}{\\}

%%%%%%%%%%%%%%%%%%%%%%%%%%%%%% Textclass specific LaTeX commands.
\numberwithin{equation}{section}
\numberwithin{figure}{section}

%%%%%%%%%%%%%%%%%%%%%%%%%%%%%% User specified LaTeX commands.
\usepackage{babel}

\@ifundefined{showcaptionsetup}{}{%
 \PassOptionsToPackage{caption=false}{subfig}}
\usepackage{subfig}
\makeatother


\usepackage{listings}

\begin{document}

\title{The APFEL DIS Module}

\author{Valerio Bertone$^{a}$}

\institution{$^{a}$PH Department, TH Unit, CERN, CH-1211 Geneva 23, Switzerland}
\maketitle

\begin{abstract}
In this document I will descrive the old and the new DIS module
embedded in APFEL.
\end{abstract}
\tableofcontents{}

\newpage{}

\section{Usage of the DIS code in APFEL}

Starting from release 2.0.0, APFEL implements a new module that
computes DIS neutral and charged current structure functions and 
cross sections up to order $\alpha_s^2$ (when possible) in the FONLL,
FFN and ZM-VFN schemes. In this short document we provide a short user
manual to use the new DIS module in APFEL.

The new DIS module can be used either in conjunction with the PDF
evolution provided by APFEL iteself or directly interfaced to LHAPDF.
To obtain any DIS observables it is enough to call one single
function, $i.e.$ {\tt DIS\_xsec}, providing a set of input parameters
that are needed to specify the computation to be performed.

Here follows the list of input/output parameters taken by the function {\tt DIS\_xsec}:
\begin{lstlisting}
APFEL::DIS_xsec(x,q0,q,y,proc,scheme,pto,pdfset,irep,target,proj,
                          F2,F3,FL,sigma);
\end{lstlisting}
where the input parameters are:
\begin{itemize}
\item the real variable {\tt x} that is the value of the Bjorken variable where the DIS
 obsevables are computed.
\item the real variable {\tt q0} that is the value of the initial scale (in GeV) used in the
 PDF evolution. This entry is used only if the APFEL internal
 evolution is used.
\item the real variable {\tt q} that is the value of the scale (in GeV) where the DIS
 observable is computed.
\item the real variable {\tt y} that is the value of the inelasticity where the DIS
observables are computed.
\item The string variable {\tt proc} that can take the values {\tt "EM"} for the
  electro-magnetic DIS computation (photon only exchange), {\tt "NC"}
  for the neutral current computation and {\tt CC"} for a charged
  current computation.
\item The string variable {\tt scheme} that can take the values {\tt "FONLL"} for FONLL, {\tt "FFNS"}
  for the FFN scheme and {\tt ZMVN"} for the ZM-VFN scheme. There is a
  further possibility, that is {\tt FFN0}, which refers to the
  zero-mass limit of the FFN scheme.
\item The integer variable {\tt pto} that can take the values 0, 1 or
  2. The user should be aware of the fact that when chosing {\tt
    pto=1} with {\tt scheme="FONLL"} the code automatically uses the
  FONLL-A scheme while when chosing {\tt
    pto=2} with {\tt scheme="FONLL"} the code automatically uses the
  FONLL-C scheme. As for the FONLL-B scheme, we
  plan to implement it in one of the next releases.
\item The string variable {\tt pdfset} that can take any of the PDF set
  names included in LHAPDF (including the {\tt .LHgrid} extension,
  $e.g$ {\tt pdfset ="NNPDF23\_nnlo\_as\_0118.LHgrid"}). This way APFEL will use
  the selected PDF set to compute the DIS observables using the evolution
  given in the grid itself. As an alternative, the user can choose {\tt
    pdfset="APFEL"}. This way the DIS observables will be computed
  using the evolution provided by APFEL bewteen the scales {\tt q0} and {\tt
    q}. For the setting of the PDF evolution, the user can refer to
  the APFEL manual.
\item The integer value {\tt irep} that specifies the replica of the PDF
  set to be used in the computation.
\item The string variable {\tt target} that takes the value {\tt
    "PROTON"} in case the target is a proton, {\tt "NEUTRON"} in case
  the target is a netron (assuming the isospin symmetry) or {\tt
    "ISOSCALAR"} if the target is an isoscalar, $e.g.$ a
  deuteron (also this option assumes isospin asymmetry). There is a
  further option which is {\tt "IRON"} which in addition uses the cross-section
  normalization used in the NeTeV experiment which is on iron nuclei.
\item The string variable {\tt proj} that takes the values {\tt
    "ELECTRON"} and {\tt "POSITRON"} if {\tt proc="EM"} or {\tt "NC"}
  in case the projectile is either an electron or a positron. If
  instead {\tt proc="CC"} the variable {\tt proj} can also take the
  values {\tt "NEUTRINO"} or {\tt "ANTINEUTRINO"} with obviuos meaning.
\end{itemize}

Once all these input parameters have been provided, the output
array variables are {\tt F2, F3, FL} and {\tt sigma}. Each of them has
5 entries corresponding to light, charm, bottom, top and total
components of the corresponding quantity.
The user should be careful because in the FORTRAN interface the arrays
are numerated from 3 to 7 ($e.g.$ {\tt F2(3)} = $F_2^l$, {\tt F2(4)} =
$F_2^c$, {\tt F2(5)} = $F_2^b$, {\tt F2(6)} = $F_2^t$, {\tt F2(7)} =
$F_2^p$) while in the C++ version they are numerated from 0 to 4 ($e.g.$ {\tt F2(0)} = $F_2^l$, {\tt F2(1)} =
$F_2^c$, {\tt F2(2)} = $F_2^b$, {\tt F2(3)} = $F_2^t$, {\tt F2(4)} =
$F_2^p$)


\section{Computing DIS Structure Functions on a Grid: the New DIS Module}

In order to speed up and optimize the compution of the DIS structure
functions in {\tt APFEL} we decided to use the same technology used
for the PDF evolution. In fact, up to version 2.0.0, the computation
of such observables in {\tt APFEL} was perfomed by directly
convoluting PDFs with the coefficient functions by mean of a
numerical integration.

Now the aim is that of precomputing on a grid the convolution of the
coefficient functions with a set of interpolating polynomials. This
way, the time consuming task of precomputing the coefficient functions
of the grid needs to be done only once and the numerical convolution
with any PDF set is instead very fast. In addition, as we will see
below, this approach provides a very natural framework to combine
precomputed coefficient functions and evolution operators, so that any
prediction of structure functions at any scale $Q$ can be obtained
very quickly by convolution with PDFs at some initial scale $Q_0$. 
Ultimately, this is particularly suitable for PDF fits.


\subsection{Zero Mass Structure Functions}

A structure function in the Zero-Mass (ZM) scheme is given by the
following convolution:
\begin{equation}
F(x,Q) = \sum_{i=g,q}x\int_x^1\frac{dy}y C_i\left(\frac{x}{y},\alpha_s(Q)\right)q_i(y,Q)\,,
\end{equation}
Now, defining $t = \ln (Q^2)$, $\widetilde{C}_i^{(n)}(y,t) = yC_i^{(n)}(y,\alpha_s(Q))$
and $\widetilde{q}_i(y,t) = y q_i(y,Q)$, the integral above can be
written as:
\begin{equation}\label{exact}
F(x,t) = \sum_{i=g,q}\int_x^1\frac{dy}y \widetilde{C}_i\left(\frac{x}{y},t\right)\widetilde{q}_i(y,t)\,.
\end{equation}
But, using a suitable interpolation basis, we can write:
\begin{equation}
\tilde{q}_i(y,t)=\sum^{N_{x}}_{\alpha=0}w_{\alpha}^{(k)}(y)\tilde{q}_i(x_{\alpha},t)\,,
\end{equation}
so that eq. (\ref{exact}) becomes:
\begin{equation}
F(x,t) = \sum_{i=g,q}\sum^{N_{x}}_{\alpha=0}\left[\int_x^1\frac{dy}y \widetilde{C}_i\left(\frac{x}{y},t\right) w_{\alpha}^{(k)}(y)\right]\tilde{q}_i(x_{\alpha},t)\,.
\end{equation}
Now let's assume that $x$ is on the grid, so that $x = x_\beta$. This
way we have:
\begin{equation}
F(x_\beta,t) = \sum_{i=g,q}\sum^{N_{x}}_{\alpha=0}\underbrace{\left[\int_{x_\beta}^1\frac{dy}y \widetilde{C}_i\left(\frac{x_\beta}{y},t\right) w_{\alpha}^{(k)}(y)\right]}_{\Gamma_{i,\beta\alpha}(t)}\tilde{q}(x_{\alpha},t)\,.
\end{equation}

Using the same arguments presented in the evolution code notes, we
have that:
\begin{equation}
\Gamma_{i,\beta\alpha}(t) \neq 0 \quad\mbox{for}\quad\beta\leq\alpha\,,
\end{equation}
and:
\begin{equation}
\Gamma_{i,\beta\alpha}(t) = \int_{c}^d\frac{dy}y \widetilde{C}_i\left(y,t\right) w_{\alpha}^{(k)}\left(\frac{x_\beta}{y}\right)
\end{equation}
with:
\begin{equation}\label{bounds2}
c =  \mbox{max}(x_\beta,x_\beta/x_{\alpha+1}) \quad\mbox{and}\quad d = \mbox{min}(1,x_\beta/x_{\alpha-k}) \,.
\end{equation}
The same symmetries holding for the splitting function case hold also here.

\subsubsection{Coefficient Functions Treatment}

The struncture of the DIS coefficient functions is very similar to
that of splitting functions with only one small complication, that is
the presence of a more divergent singular term. In practice the
structure of the DIS coefficient functions is the following:
\begin{equation}\label{CFstructure}
\widetilde{C}_{i}(x,t) = xC_{i}^{R}(x,t) +
xC_{i}^{S1}(t)\left[\frac{1}{1-x}\right]_+ +
xC_{i}^{S2}(t)\left[\frac{\ln(1-x)}{1-x}\right]_+ + xC_{i}^{L}(t)\delta(1-x)\,.
\end{equation}
The term proportional to $C_{i}^{S2}$ can be treated, considering that:
\begin{equation}
\begin{array}{rcl}
\displaystyle \int_c^ddy\left[\frac{\ln(1-y)}{1-y}\right]_+f(y)
&=&\displaystyle \int_c^ddy\frac{\ln(1-y)}{1-y}\left[f(y)-f(1)\theta(d-1)\right]\\
\\
&+&\displaystyle \frac12 f(1) \ln^2(1-c)\theta(d-1)
\end{array}
\end{equation}
On the same line of splitting functions, we know that the coefficient
functions ha the following perturbative expansion:
\begin{equation}
C_{i}^{J}(x,t) = \sum_{n=0}^{N}a_s^{n}(t)C_{i}^{J,(n)}(x)\qquad\mbox{with}\qquad J=R,S1,S2,L
\end{equation}

Therefore one has that:
\begin{equation}\label{Kernels}
\begin{array}{c}
\displaystyle \Gamma_{j,\beta\alpha}(t) = \\
\\
\displaystyle \sum_{n=0}^{N} a_s^{n}(t) \bigg\{\int^{d}_{c}dy\left[{C}_{i}^{R,(n)}(y)w_{\alpha}\left(\frac{x_\beta}{y}\right)+\frac{{C}_{i}^{S1,(n)}+{C}_{i}^{S2,(n)}\ln(1-y)}{1-y}\left(w_{\alpha}\left(\frac{x_\beta}{y}\right)-\delta_{\beta\alpha}\theta(d-1)\right)\right]\\
\\
\displaystyle +\left[{C}_{i}^{S1,(n)}\ln(1-c)\theta(d-1)+\frac12{C}_{i}^{S2,(n)}\ln^2(1-c)\theta(d-1)+{C}_{i}^{L,(n)}\right]\delta_{\beta\alpha}\bigg\}\,.
\end{array}
\end{equation}

Calling:
\begin{equation}
\begin{array}{c}
\displaystyle \Gamma_{i,\beta\alpha}^{(n)}(t) = \\
\\
\displaystyle \int^{d}_{c}dy\left[{C}_{i}^{R,(n)}(y)w_{\alpha}\left(\frac{x_\beta}{y}\right)+\frac{{C}_{i}^{S1,(n)}+{C}_{i}^{S2,(n)}\ln(1-y)}{1-y}\left(w_{\alpha}\left(\frac{x_\beta}{y}\right)-\delta_{\beta\alpha}\theta(d-1)\right)\right]\\
\\
\displaystyle +\left[{C}_{i}^{S1,(n)}\ln(1-c)\theta(d-1)+\frac12{C}_{i}^{S2,(n)}\ln^2(1-c)\theta(d-1)+{C}_{i}^{L,(n)}\right]\delta_{\beta\alpha}\,,
\end{array}
\end{equation}
we have that:
\begin{equation}\label{splittingexp}
\Gamma_{i,\beta\alpha}(t) = \sum_{n=0}^{N} a_s^{n}(t) \Gamma_{i,\beta\alpha}^{(n)}\,,
\end{equation}
and the integrals $\Gamma_{i,\beta\alpha}^{(n)}$ do not depend on the energy therefore, once the grid (and the number of active flavours) has been fixed, they can be evaluate once and for all at the beginning and used for the convolution at any scale.

Now, assuming to have computed the evolution operator
$M_{ij,\alpha\beta}(t,t_0)$ between the
scales $t=\ln(Q^2)$ and $t_0=\ln(Q_0^2)$ on the same grid where we
have computed the operator $\Gamma_{i,\beta\alpha}(t)$, one can esily
combine the two obtaining the prediction for the structure function
$F$ on the grid in terms of PDFs at the initial scale $Q_0$ just by performing the
following convolution:
\begin{equation}
F(x_\alpha,t) = \Gamma_{i,\alpha\beta}(t) M_{ij,\beta\gamma}(t,t_0) \tilde{q}_j(x_{\gamma},t_0)
\end{equation}
where a sum of the repeated indeces is understood.

Before proceeding to treatment of the massive coefficient functions,
we stress that in the massless scheme, for obviuos kinematical
reasons, there is no need to distinguish between charged- and
neutral-current coefficient functions. The difference between the two
cases appears only at the level of structure functions where the
coefficient functions are comvoluted with different combinations of
PDFs and combined according to the structure of the couplings to
quarks of the $Z/\gamma^*$ vector bosons in the neutral-current case
and the $W^\pm$ in the charged-current case.

It is opportune at this point to mention that, when considering
charged-current observables at NLO in the massive scheme, there is a
further contribution to be added to eq.~(\ref{CFstructure}) that has the form:
\begin{equation}
C_{i}^{SL}(t)\frac{d}{dx}\delta(1-x)\,.
\end{equation}
Starting from the relation:
\begin{equation}
x\frac{d}{dx}\delta(x) = -\delta(x)\,,
\end{equation}
one can easily show that:
\begin{equation}\label{KeyIdentity}
\frac{d}{dx}\delta(1-x) = \left[\frac{\delta(1-x)}{1-x}\right]_+\,.
\end{equation}
To make sure that this identity is correct, we try to convolute both
the r.h.s. and the l.h.s. of eq.~(\ref{KeyIdentity}) with the test
function $f(x)$, such that $f(1) = 0$, to see what is the result and
whether the results are equal. Using the l.h.s. we have:
\begin{equation}
\int_x^1dy\,f(y)\,\frac{d}{dx}\delta(1-y) =
\underbrace{f(y)\delta(1-y)\Big{|}_{x}^1}_{=0} -
\int_x^1dy\frac{df(y)}{dy}\delta(1-y) = -\frac{df(y)}{dy}\bigg{|}_{y=1}\,,
\end{equation}
while using the r.h.s.(\footnote{Since the delta function selects the
  point $y=1$ in the following integral, the ``incomplete'' integral of
  plus-prescripted function does not give rise to any residual
  logarithm of the form $\ln(1-x)$.}):
\begin{equation}
\begin{array}{c}
\displaystyle \int_x^1dy\,f(y)\,\left[\frac{\delta(1-y)}{1-y}\right]_+ =
\int_x^1dy\frac{f(y)-f(1)}{1-y}\delta(1-y) = \lim_{\epsilon\rightarrow
  0^+} \int_x^1dy\frac{f(y)-f(1)}{1-y}\delta(1-\epsilon-y) =\\
\\
\displaystyle - \lim_{\epsilon\rightarrow
  0^+} \frac{f(1)-f(1-\epsilon)}{\epsilon} =
-\frac{df(y)}{dy}\bigg{|}_{y=1}\,.
\end{array}
\end{equation}
So the results are equal and the distributions in
eq.~(\ref{KeyIdentity}) when convoluted with a test function extract
its derivative in $y=1$, up to a minus sign.

At the end of the day one has to include inside the curly brackets of eq.~(\ref{Kernels}) the term:
\begin{equation}
- C_{i}^{SL,(n)}\frac{dw_\alpha^{(k)}(x_\beta)}{dx}
\end{equation}

In addition, when using a Lagrange interpolation, one can show that
the first derivative of the Lagrange polynomials have the form:
\begin{equation}\label{LagrangFirstDerivative}
\frac{d w_\alpha^{(k)}(x_\rho)}{dx} =
\left\{
\begin{array}{ll}
\displaystyle \sum_{\sigma=0\atop \sigma\neq\alpha}^k\frac{1}{x_\alpha-x_\sigma}  & \quad \rho = \alpha \\
\\
\displaystyle \frac{1}{x_\alpha-x_\rho}
\prod_{\sigma=0\atop\sigma\neq\alpha,\rho}^k\frac{x_\rho - x_\sigma}{x_\alpha-x_\sigma} & \quad \rho \neq \alpha
\end{array}
\right.
\end{equation}
The relation in eq.~(\ref{LagrangFirstDerivative}) is proved in the
``Lagrange\_derivative.pdf'' notes.


\subsection{Massive Structure Functions}

Now we can proceed considering the massive structure functions. When
computing structure functions in the massive scheme, there is a
further complication that complicates a fast precomputation of the
cefficient functions on the $x$-space grid and it is the fact that the
coefficients of the perturbative expansion of the massive coeffient
functions carry an intrinsic dependence on the scale of the process.
This prevents a scale independent pre-tabulation of the coefficient
functions on on $x$-space grid.

One possible way out is to pre-tabulate the coefficient functions, not
only on an $x$-space grid, but also on a $Q$-space grid, where $Q$ is
the scale at which the structure functions are evaluated. Actually the
most efficient way of precomputing the massive coefficient functions
is on $\xi$-space grid, where $\xi$ is defined as:
\begin{equation}
\xi = \frac{Q^2}{m_H^2}\,.
\end{equation}
where $m_H$ is the mass of the heavy quark under consideration.  In
fact, for dimentional reasons, massive coefficient functions depend on
the scale $Q$ through $\xi$. Neglecting for the moment the dependence
on the renormalization and on the factorization scales, the massive
coefficient functions allow for the following expansion:
\begin{equation}
C_i(x,Q,m_H) = \sum_n a_s^n(t)C_i^{(n)}(x,\xi)\,.
\end{equation}
Given this relation, the massive analogous of the
eq.~(\ref{splittingexp}) is:
\begin{equation}\label{splittingexp}
\Gamma_{i,\beta\alpha}(Q,m_H) = \sum_n a_s^{n}(t) \Gamma_{i,\beta\alpha}^{(n)}(\xi)\,.
\end{equation}
In order not to recompute the operator
$\Gamma_{i,\beta\alpha}^{(n)}(\xi)$ any time that $\xi$ changes, we
can tabulate the on a grid in $\xi$,
$\{\xi_1,\dots,\xi_\tau,\dots,\xi_{N_\xi}\}$, definining:
\begin{equation}
\Gamma_{i,\beta\alpha,\tau}^{(n)} = \Gamma_{i,\beta\alpha}^{(n)}(\xi_\tau),.
\end{equation}
and then interpolate to obtain the operator for a generic value of
$\xi$. We have chosen to use a linear interpolation so that:
\begin{equation}
\Gamma_{i,\beta\alpha}^{(n)}(\xi) =
c^{(0)}(\xi)\Gamma_{i,\beta\alpha,\tau}^{(n)} + c^{(1)}(\xi)\Gamma_{i,\beta\alpha,\tau+1}^{(n)}\,,
\end{equation}
with:
\begin{equation}
c^{(0)}(\xi) = \frac{\ln\xi_{\tau+1} - \ln\xi}{\ln\xi_{\tau+1} -
  \ln\xi_\tau}\quad\mbox{and}\quad c^{(1)}(\xi) =
\frac{\ln\xi - \ln\xi_\tau}{\ln\xi_{\tau+1} - \ln\xi_\tau} \,,
\end{equation}
provided that $\xi_\tau \leq \xi < \xi_{\tau+1}$.

To conclude, once the operators $\Gamma_{i,\beta\alpha,\tau}^{(n)}$
have been precomputed, the operator for a generic value of $\xi$ can
be quickly computed by interpolation.

\subsubsection{Neutral Current Coefficient Functions}

As far as the neutral current coefficient functions are concerned,
beyond LO(\footnote{We remind that, in the neutral current case, the
  LO is order $\alpha_s$.}), a close analytical form is not available
and only a semi-analitical form~\cite{Laenen:1992xs} which is not
suitable for a fast numerical implementation. The authors of
Ref.~\cite{Alekhin:2003ev} have used a simple parametrization to fit
the exact coefficient functions. Such parametrization is actually
meant to be used in Mellin space, however it can equally be used in
$x$ space providing a fast and accurate enough alternative to the
original implemetation. The parametrization of
Ref.~\cite{Alekhin:2003ev} has the form:
\begin{equation}
C(x,\xi) = \theta(\rho-x)(\rho-x)^{-\kappa}\sum_{k=0}^K
a_k(\rho)x^k\quad\mbox{with}\quad\rho = \frac{\xi}{\xi+4}\,,
\end{equation}
and the authors provide the numerical values of $\kappa$, $K$ and
$a_k(\rho)$ for all the relevant coefficient functions at LO
($\mathcal{O}(\alpha_s)$) and NLO ($\mathcal{O}(\alpha_s^2)$) 
tabulated on a $\xi$-space grid for large enough range in $\xi$.
Note the presence of the $\theta$-function that has the scope of
reducing the phase scace available for the process due the production
of two heavy quarks in the final state.

In {\tt APFEL} we make use of the parametrization above only for the
NLO coefficient functions as the exact form of the LO ones is
available in Ref.~\cite{Forte:2010ta} and compact enough for an
efficient implementation. In addition, also for the pure singlet NLO
coefficient functions (sometimes called gluon-radiation terms) we
employ the analytical expressions given in Appendix A of
Ref.~\cite{Buza:1995ie}.

We finally remark that massive coeffient functions for the neutral
current structure functions are presently known only for $F_2$ and
$F_L$. For the parity-violating structure function $F_3$ we thus use
the massless coefficient functions.

As far as the massless limit of the massive (massive-zero) coefficient
functions is concerned, exact expressions up to
$\mathcal{O}(\alpha_s^2)$ have been evaluated in
Ref.~\cite{Buza:1995ie} and reported in Appendix D. Such expressions
are implemented in {\tt APFEL}

As in the massive case, massive-zero coefficient functions are know
only for $F_2$ and $F_L$ and again for the $F_3$ structure function we
use the massless coefficient functions.

\subsubsection{Charged Current Coefficient Functions}

We can now consider the charged-current sector. In this case, massive
coeffincient functions are know only up to $\mathcal{O}(\alpha_s)$
(NLO), therefore a proper computation of charged-current structure
functions the NNLO version of the FONLL scheme (called FONLL-C) is
impossible. However, the best we can do when computing charged-current
structure functions in the FONLL-C scheme is to set the NNLO contributions to
zero in the massive sectors but keeping those in the massless sector, as
well as using NNLO evolution for PDFs and $\alpha_s$.

The charged-current massive structure functions, like the other
structure functions, are given by the convolution of PDFs with
coefficient functions. Considering the heavy-quark $H$ structure functions in the
approximation of diagonal CKM matrix, the definitions are:
\begin{equation}\label{F1}
F_1^H(x,Q,m_H)=\frac12\int_{\chi}^{1}\frac{dy}{y}\left[C_{1,q}(y,\xi)s\left(\frac{\chi}{y},Q\right)+C_{1,g}(y,\xi)g\left(\frac{\chi}{y},Q\right)\right]
\end{equation}
\begin{equation}\label{F2}
F_2^H(x,Q,m_H)=\chi\int_{\chi}^{1}\frac{dy}{y}\left[C_{2,q}(y,\xi)s\left(\frac{\chi}{y},Q\right)+C_{2,g}(y,\xi)g\left(\frac{\chi}{y},Q\right)\right]
\end{equation}
\begin{equation}\label{F3}
F_3^H(x,Q,m_H)=\int_{\chi}^{1}\frac{dy}{y}\left[C_{3,q}(y,\xi)s\left(\frac{\chi}{y},Q\right)+C_{3,g}(y,\xi)g\left(\frac{\chi}{y},Q\right)\right]
\end{equation}
with:
\begin{equation}
\chi = x\left(1+\frac{m_H^2}{Q^2}\right) = \frac{x}{\lambda}\,,
\end{equation}
where:
\begin{equation}
\lambda = \frac{Q^2}{Q^2+m_H^2} = \frac{\xi}{1+\xi}\,.
\end{equation}
Now, defining:
\begin{equation}
F_L^H(x,Q,m_H) = F_2^H(x,Q,m_H) - 2xF_1^H(x,Q,m_H)\,,
\end{equation}
we have that:
\begin{equation}\label{FL}
F_L^H(x,Q,m_H)=\chi\int_{\chi}^{1}\frac{dy}{y}\left[C_{L,q}(y,\xi)s\left(\frac{\chi}{y},Q\right)+C_{L,g}(y,\xi)g\left(\frac{\chi}{y},Q\right)\right]\,,
\end{equation}
where we have defined:
\begin{equation}\label{clll}
C_{L,q(g)}(y,\xi) = C_{2,q(g)}(y,\xi)-\lambda C_{1,q(g)}(y,\xi)
\end{equation}

All the coefficient functions entering the structure functions above
admit a perturbative expansion that at N$^N$LO reads:
\begin{equation}
C_{k,q(g)}(y,\xi) = \sum_{n=0}^N a_s^n(Q)
C_{k,q(g)}^{(n)}(y,\xi)\,,\quad k = 1,2,3,L\,.
\end{equation}
In the following we will truncate the expansion at NLO.

After the definitions above we can write down, first the LO
coefficient functions. While at LO the gluon coefficient functions are
all zero ($C_{k,g}^{(0)}(y,\xi)=0$), the quark coefficient functions are:
\begin{equation}
\begin{array}{l}
\displaystyle C^{(0)}_{1,q}(x,\xi) = \delta(1-x)\,,\\
\\
\displaystyle C^{(0)}_{2,q}(x,\xi) = \delta(1-x) \,,\\
\\
\displaystyle C^{(0)}_{3,q}(x,\xi) = \delta(1-x) \,,\\
\\
\displaystyle C^{(0)}_{L,q}(x,\xi) = (1-\lambda)\delta(1-x) \,.
\end{array}
\end{equation}

The NLO charged-current massive coefficient have been computed and
reported in Appendix A of Ref.~\cite{Gluck:1996ve}. However, before
being implemented in {\tt APFEL} they need some manipulation. We start
defining:
\begin{equation}
K_A=\frac{1}{\lambda}(1-\lambda)\ln(1-\lambda)\,.
\end{equation}
In addition, in order to consider factorization scale variations, we also need to
consider the splitting function:
\begin{equation}
P_{qq}^{(0)}(z) = C_F\left[\frac{1+z^2}{(1-z)_+}+\frac32\delta(1-z)\right]=C_F\left[\frac{2}{(1-z)_+}-(1+z)+\frac32\delta(1-z)\right]\,,
\end{equation}
and we also define:
\begin{equation}
K_F^2 =\frac{Q^2}{\mu_F^2}\,.
\end{equation}

The explicit expressions of the NLO quark coefficient functions read:
\begin{equation}
\begin{array}{rcl}
C^{(1)}_{1,q}&=&\displaystyle 2C_F \bigg\{
\bigg(-4-\frac{1}{2\lambda}-2\zeta_2-\frac{1+3\lambda}{2\lambda}K_A+\frac32
\ln\frac{K_F^2}{\lambda}\bigg)\delta(1-z)\\
\\
&-&\displaystyle \frac{(1+z^2)\ln z}{1-z} +
\left(-\ln\frac{K_F^2}{\lambda}-2\ln(1-z)+\ln(1-\lambda
  z)\right)(1+z)+(3-z)+\frac{1}{\lambda^2}+\frac{z-1}{\lambda}\\
\\
&+&\displaystyle 2
\left[\frac{2\ln(1-z)-\ln(1-\lambda z)}{1-z}\right]_++
2\left(-1+\ln\frac{K_F^2}{\lambda}\right)\left[\frac{1}{1-z}\right]_+\\
\\
&+& \displaystyle \frac{\lambda-1}{\lambda^2}\left[\frac{1}{1-\lambda z}\right]_+ +\frac{1}{2}\left[\frac{1-z}{(1-\lambda
      z)^2}\right]_+\bigg\}
\end{array}\,,
\end{equation}

\begin{equation}
\begin{array}{rcl}
C^{(1)}_{2,q}&=&\displaystyle 2C_F \bigg\{
\bigg(-4-\frac{1}{2\lambda}-2\zeta_2-\frac{1+\lambda}{2\lambda}K_A+\frac32
\ln\frac{K_F^2}{\lambda}\bigg)\delta(1-z)\\
\\
&-&\displaystyle \frac{(1+z^2)\ln z}{1-z} +
\left(2-\ln\frac{K_F^2}{\lambda}-2\ln(1-z)+\ln(1-\lambda
  z)\right)(1+z)+\frac{1}{\lambda}\\
\\
&+&\displaystyle 2
\left[\frac{2\ln(1-z)-\ln(1-\lambda z)}{1-z}\right]_++
2\left(-1+\ln\frac{K_F^2}{\lambda}\right)\left[\frac{1}{1-z}\right]_+\\
\\
&+& \displaystyle \frac{2\lambda^2-\lambda-1}{\lambda}\left[\frac{1}{1-\lambda z}\right]_+ +\frac{1}{2}\left[\frac{1-z}{(1-\lambda
      z)^2}\right]_+\bigg\}
\end{array}\,,
\end{equation}

\begin{equation}
\begin{array}{rcl}
C^{(1)}_{3,q}&=&\displaystyle 2C_F \bigg\{
\bigg(-4-\frac{1}{2\lambda}-2\zeta_2-\frac{1+3\lambda}{2\lambda}K_A+\frac32
\ln\frac{K_F^2}{\lambda}\bigg)\delta(1-z)\\
\\
&-&\displaystyle \frac{(1+z^2)\ln z}{1-z} +
\left(1-\ln\frac{K_F^2}{\lambda}-2\ln(1-z)+\ln(1-\lambda
  z)\right)(1+z)+\frac{1}{\lambda}\\
\\
&+&\displaystyle 2
\left[\frac{2\ln(1-z)-\ln(1-\lambda z)}{1-z}\right]_++
2\left(-1+\ln\frac{K_F^2}{\lambda}\right)\left[\frac{1}{1-z}\right]_+\\
\\
&+& \displaystyle \frac{\lambda-1}{\lambda}\left[\frac{1}{1-\lambda z}\right]_+ +\frac{1}{2}\left[\frac{1-z}{(1-\lambda
      z)^2}\right]_+\bigg\}
\end{array}\,,
\end{equation}

\begin{equation}
\begin{array}{rcl}
C^{(1)}_{L,q}
&=&\displaystyle 2C_F (1-\lambda)\bigg\{
\bigg(-4-\frac{1}{2\lambda}-2\zeta_2-\frac{1+\lambda}{2\lambda}K_A+\frac32
\ln\frac{K_F^2}{\lambda}\bigg)\delta(1-z)\\
\\
&-&\displaystyle \frac{(1+z^2)\ln z}{1-z} +
\left(-\ln\frac{K_F^2}{\lambda}-2\ln(1-z)+\ln(1-\lambda
  z)\right)(1+z)+3\\
\\
&+&\displaystyle 2 
\left[\frac{2\ln(1-z)-\ln(1-\lambda z)}{1-z}\right]_++
2 \left(-1+\ln\frac{K_F^2}{\lambda}\right)\left[\frac{1}{1-z}\right]_+\\
\\
&-& \displaystyle 2\left[\frac{1}{1-\lambda z}\right]_+ +\frac{1}{2}\left[\frac{1-z}{(1-\lambda
      z)^2}\right]_+\bigg\} + 2C_F \left[\lambda K_A \delta(1-z) + (1+\lambda)z\right]
\end{array}\,.
\end{equation}

In order to proceed with our manipulations we need to define the
generalized or incomplete +-prescription:
\begin{equation}
\begin{array}{c}
\displaystyle \int_x^1 dz\left[f(z)\right]_+g(z) = \int_x^1
dz\,f(z)\left[g(z)-g(1)\right]-g(1)\underbrace{\int_0^xdz\,f(z)}_{-R_f(x)}=\\
\\
\displaystyle \int_x^1 dz\left\{\left[f(z)\right]_++R_f(x)\delta(1-z)\right\}g(z)\,.
\end{array}
\end{equation}
where the +-prescription in the r.h.s of the equation above should be
understood in the usual way independently of the integration bounds.

Often the residual $R_f(x)$ function can be evaluated analytically by
performing the integral, however sometimes it need to be evaluated
numerically performing the integral in a numerical way. In particular
the $+$-prescripted functions that enter the expressions above give
rise to the following residual functions that can be computed analytically:
\begin{equation}
-\int_0^x\frac{dz}{1-z} = \ln(1-x)\,,
\end{equation}
\begin{equation}
-\int_0^xdz\frac{\ln(1-z)}{1-z} = \frac12 \ln^2(1-x)\,,
\end{equation}
\begin{equation}
-\int_0^x\frac{dz}{1-\lambda z} = \frac{1}{\lambda}\ln(1-\lambda x)\,,
\end{equation}
\begin{equation}
-\int_0^xdz\frac{1-z}{(1-\lambda z)^2} =
\frac{1}{\lambda^2}\ln(1-\lambda
x)+\frac{1-\lambda}{\lambda}\frac{x}{1-\lambda x}\,,
\end{equation}
while we do not know how to solve analytically the integral:
\begin{equation}
R(x)=-\int_0^xdz\frac{\ln(1-\lambda z)}{1-z}
\end{equation}
therefore we will compute it numerically.

As a consequence, when convoluting the coefficient functions above
with PDFs in the point $x$, we can treat the $+$-prescripted functions
using the standard definition at the price of adding to the local
terms the following functions:
\begin{equation}
\begin{array}{rcl}
C^{(1)}_{1,q}&\rightarrow& \displaystyle C^{(1)}_{1,q} +
2C_F\bigg[2\ln^2(1-x)-2R(x)+2\left(-1+\ln\frac{K_F^2}{\lambda}\right)\ln(1-x)\\
\\
&&\displaystyle +\frac{\lambda-1}{\lambda^3}\ln(1-\lambda
  x)+\frac{1}{2\lambda^2}\ln(1-\lambda
x)+\frac{1-\lambda}{2\lambda}\frac{x}{1-\lambda x}\bigg]\delta(1-z)
\end{array}\,,
\end{equation}
\begin{equation}
\begin{array}{rcl}
C^{(1)}_{2,q}&\rightarrow& \displaystyle C^{(1)}_{2,q} +
2C_F\bigg[2\ln^2(1-x)-2R(x)+2\left(-1+\ln\frac{K_F^2}{\lambda}\right)\ln(1-x)\\
\\
&&\displaystyle +\frac{2\lambda^2-\lambda-1}{\lambda^2}\ln(1-\lambda
  x)+\frac{1}{2\lambda^2}\ln(1-\lambda
x)+\frac{1-\lambda}{2\lambda}\frac{x}{1-\lambda x}\bigg]\delta(1-z)
\end{array}\,,
\end{equation}
\begin{equation}
\begin{array}{rcl}
C^{(1)}_{3,q}&\rightarrow& \displaystyle C^{(1)}_{3,q} +
2C_F\bigg[2\ln^2(1-x)-2R(x)+2\left(-1+\ln\frac{K_F^2}{\lambda}\right)\ln(1-x)\\
\\
&&\displaystyle +\frac{\lambda-1}{\lambda^2}\ln(1-\lambda
  x)+\frac{1}{2\lambda^2}\ln(1-\lambda
x)+\frac{1-\lambda}{2\lambda}\frac{x}{1-\lambda x}\bigg]\delta(1-z)
\end{array}\,,
\end{equation}
\begin{equation}
\begin{array}{rcl}
C^{(1)}_{L,q}&\rightarrow& \displaystyle C^{(1)}_{L,q} +
2C_F(1-\lambda)\bigg[2\ln^2(1-x)-2R(x)+2\left(-1+\ln\frac{K_F^2}{\lambda}\right)\ln(1-x)\\
\\
&&\displaystyle -\frac{2}{\lambda}\ln(1-\lambda
  x)+\frac{1}{2\lambda^2}\ln(1-\lambda
x)+\frac{1-\lambda}{2\lambda}\frac{x}{1-\lambda x}\bigg]\delta(1-z)
\end{array}\,.
\end{equation}

Now let us consider the gluon coefficient functions, that read:
\begin{equation}
\begin{array}{rcl}
C^{(1)}_{1,g}&=&\displaystyle 2T_R\bigg\{[z^2+(1-z)^2]\left[\ln\left(\frac{1-z}{z}\right)
    -\frac12\ln(1-\lambda) +\frac12\ln\frac{K_F^2}{\lambda}
\right]+\\
\\
&&\displaystyle 4z(1-z) - 1+\\
 \\
&&\displaystyle (1-\lambda)\left[-4z(1-z) + \frac{z}{1-\lambda z}
  +2z(1-2\lambda z)\ln\frac{1-\lambda
      z}{(1-\lambda)z}\right]\bigg\}\\
\end{array}\,,
\end{equation}
\begin{equation}
\begin{array}{rcl}
C^{(1)}_{2,g}&=&\displaystyle 2T_R \bigg\{[z^2+(1-z)^2]\left[\ln\left(\frac{1-z}{z}\right)
    -\frac12\ln(1-\lambda) +\frac12\ln\frac{K_F^2}{\lambda}
\right]+\\
\\
&&\displaystyle 8z(1-z)- 1+\\
\\
&&\displaystyle (1-\lambda)\left[-6(1+2\lambda) z(1-z)+\frac{1}{1-\lambda z} + 6\lambda z(1-2\lambda z)\ln\frac{1-\lambda
      z}{(1-\lambda)z}\right]\bigg\}\\
\end{array}\,,
\end{equation}
\begin{equation}
\begin{array}{rcl}
C^{(1)}_{3,g}&=&\displaystyle 2T_R \bigg\{[z^2+(1-z)^2]\left[
  2\ln\left(\frac{1-z}{1-\lambda z}\right)+\frac12\ln(1-\lambda) +\frac12\ln\frac{K_F^2}{\lambda}\right]+\\
\\
&&\displaystyle (1-\lambda)\left[2 z(1-z) - 2z[1-(1+\lambda
  )z]\ln\frac{1-\lambda z}{(1-\lambda)z}\right]\bigg\}\,,
\end{array}
\end{equation}
\begin{equation}
\begin{array}{rcl}
C^{(1)}_{L,g}&=&\displaystyle 2T_R \bigg\{(1-\lambda)[z^2+(1-z)^2]\left[\ln\left(\frac{1-z}{z}\right)
    -\frac12\ln(1-\lambda) +\frac12\ln\frac{K_F^2}{\lambda}
\right]+\\
\\
&&\displaystyle 4(2-\lambda)z(1-z)+\\
\\
&&\displaystyle (1-\lambda)\left[-2(3+4\lambda) z(1-z)+ 4\lambda z(1-2\lambda z)\ln\frac{1-\lambda
      z}{(1-\lambda)z}\right]\bigg\}\,.
\end{array}
\end{equation}
Since these functions do not contain any $+$-prescripted functions,
they can be implemented as they are.

We now consider the massless limit of the above massive coefficient
functions.  We start considering that:
\begin{equation}
\begin{array}{l}
\lambda \rightarrow 1\\
K_A \rightarrow 0
\end{array}\,,
\end{equation}
as consequence we find that the quarl coefficient functions tend to:
\begin{equation}
\begin{array}{rcl}
C^{(1)}_{1,q} \displaystyle\mathop{\longrightarrow}_{m_H\rightarrow 0}
C^{0,(1)}_{1,q} &=&\displaystyle 2C_F \bigg\{
-\left(\frac{9}{2}+2\zeta_2-\frac32
\ln K_F^2\right)\delta(1-z)\\
\\
&-&\displaystyle \frac{(1+z^2)\ln z}{1-z}
-\left(\ln(1-z)+\ln K_F^2\right)(1+z)+3\\
\\
&+&\displaystyle 2
\left[\frac{\ln(1-z)}{1-z}\right]_+
-\left(\frac{3}{2}-2\ln K_F^2\right)\left[\frac{1}{1-z}\right]_+\bigg\}
\end{array}
\end{equation}
\begin{equation}
\begin{array}{rcl}
C^{(1)}_{2,q}\displaystyle\mathop{\longrightarrow}_{m_H\rightarrow 0}C^{0,(1)}_{2,q}&=&\displaystyle 2C_F \bigg\{
-\left(\frac{9}{2}+2\zeta_2-\frac32
\ln K_F^2\right)\delta(1-z)\\
\\
&-&\displaystyle \frac{(1+z^2)\ln z}{1-z}
-\left(\ln(1-z)+\ln K_F^2\right)(1+z)+2z+3\\
\\
&+&\displaystyle 2
\left[\frac{\ln(1-z)}{1-z}\right]_+
-\left(\frac{3}{2}-2\ln K_F^2\right)\left[\frac{1}{1-z}\right]_+\bigg\}
\end{array}
\end{equation}
\begin{equation}
\begin{array}{rcl}
C^{(1)}_{3,q}\displaystyle\mathop{\longrightarrow}_{m_H\rightarrow 0 C^{0,(1)}_{3,q}}&=&\displaystyle 2C_F \bigg\{
-\left(\frac{9}{2}+2\zeta_2-\frac32
\ln K_F^2\right)\delta(1-z)\\
\\
&-&\displaystyle \frac{(1+z^2)\ln z}{1-z}
-\left(\ln(1-z)+\ln K_F^2\right)(1+z)+z+2\\
\\
&+&\displaystyle 2
\left[\frac{\ln(1-z)}{1-z}\right]_+
-\left(\frac{3}{2}-2\ln K_F^2\right)\left[\frac{1}{1-z}\right]_+\bigg\}
\end{array}
\end{equation}
\begin{equation}
C^{(1)}_{L,q}\mathop{\longrightarrow}_{m_H\rightarrow 0}
C^{0,(1)}_{L,q} = 4C_F z
\end{equation}

The local term to be added to the quark coefficient functions,
considering that:
\begin{equation}
R(x) \mathop{\longrightarrow}_{m_H\rightarrow 0} \frac12\ln(1-x)^2\,,
\end{equation}
are:
\begin{equation}
C^{0,(1)}_{1,q}\rightarrow C^{0, (1)}_{1,q} +
2C_F\left[\ln^2(1-x)-\left(\frac32-2\ln K_F^2\right)\ln(1-x)\right]\delta(1-z)\,,
\end{equation}
\begin{equation}
C^{0, (1)}_{2,q}\rightarrow C^{0, (1)}_{2,q} +
2C_F\left[\ln^2(1-x)-\left(\frac32-2\ln K_F^2\right)\ln(1-x)\right]\delta(1-z)\,,
\end{equation}
\begin{equation}
C^{0, (1)}_{3,q}\rightarrow C^{0, (1)}_{3,q} +
2C_F\left[\ln^2(1-x)-\left(\frac32-2\ln K_F^2\right)\ln(1-x)\right]\delta(1-z)\,,
\end{equation}
while no local term needs to be added to $C^{0, (1)}_{L,q}$.

Now we turn to the gluon coefficient functions where we need to
know that:
\begin{equation}
\ln(1-\lambda) \mathop{\longrightarrow}_{m_H\rightarrow 0} -\ln\left(\frac{Q^2}{m_H^2}\right)
\end{equation}
so that:
\begin{equation}
C^{(1)}_{1,g}\mathop{\longrightarrow}_{m_H\rightarrow 0} C^{0, (1)}_{1,g} = 2T_R\left\{[z^2+(1-z)^2]\left[\ln\left(\frac{1-z}{z}\right)
    +\frac12 \ln\left(\frac{Q^2}{m_H^2}\right) +\frac12\ln K_F^2\right]+ 4z(1-z) - 1\right\}\,,
\end{equation}
\begin{equation}
C^{(1)}_{2,g}\mathop{\longrightarrow}_{m_H\rightarrow 0} C^{0, (1)}_{2,g} = 2T_R\left\{[z^2+(1-z)^2]\left[\ln\left(\frac{1-z}{z}\right)
    +\frac12 \ln\left(\frac{Q^2}{m_H^2}\right) +\frac12\ln K_F^2\right]+ 8z(1-z) - 1\right\}\,,
\end{equation}
\begin{equation}
C^{(1)}_{3,g}\mathop{\longrightarrow}_{m_H\rightarrow 0} C^{0, (1)}_{3,g} = 2T_R[z^2+(1-z)^2]\left[-\frac12 \ln\left(\frac{Q^2}{m_H^2}\right) +\frac12\ln K_F^2\right]\,,
\end{equation}
\begin{equation}
C^{(1)}_{L,g}\mathop{\longrightarrow}_{m_H\rightarrow 0} C^{0,(1)}_{L,g} = 2T_R\left[4z(1-z)\right]\,.
\end{equation}
We also note that in the limit $m_H\rightarrow 0$, the covolution
integrals in eqs.~(\ref{F1}), (\ref{F2}), (\ref{F3}) and~(\ref{FL})
will extend from $x$ to 1 rather than from $\chi$ to 1.

As clear from the definitions in eqs.~(\ref{F1}), (\ref{F2}),
(\ref{F3}) and~(\ref{FL}), in order to compute a give structure
functions for some given value of $x$, one needs to convolute the
coefficient functions that we have written above with PDFs in the
rescaled point $\chi=x/\lambda > x$, so in particular the convolution
integral extends from $\chi$ to 1. This is a kinematical consequence
of the mass of the heavy quark involved that reduces
the phase space available for the process.

From the point of view of the implementation of the FONLL scheme in
{\tt APFEL}, given that the massive scheme needs to be combined with
the massless and the massive-zero schemes whose convolution integrals
extend from $x$ to 1, it would be convinient to rewrite eqs.~(\ref{F1}), (\ref{F2}),
(\ref{F3}) and~(\ref{FL}) in such a way that the lower interagration bound
is $x$ rather than $\chi$. To this end, let us consider the integral:
\begin{equation}
I=\int_\chi^1\frac{dy}{y} C(y)f\left(\frac{\chi}{y}\right)\,,
\end{equation}
where $\chi=x/\lambda$. By the change of integration variable $z =
\lambda y$, we can rewrite the integral above as:
\begin{equation}
I=\int_x^\lambda\frac{dz}{z} C\left(\frac{z}{\lambda}\right)f\left(\frac{x}{y}\right) = \int_x^1\frac{dz}{z} \widetilde{C}(z,\lambda)f\left(\frac{x}{y}\right)\,,
\end{equation}
where:
\begin{equation}
\widetilde{C}(z,\lambda)=\theta(\lambda-z)C\left(\frac{z}{\lambda}\right)\,.
\end{equation}
In this way we have achived the goal of expressing the ``reduced''
convolution in eqs.~(\ref{F1}), (\ref{F2}), (\ref{F3}) and~(\ref{FL})
as a ``standard'' convolution between $x$ and 1. The price to pay is
to consider the massive coefficient functions during the integration
as explicit functions of the variable $z/\lambda$ and to cut off the
region $z>\lambda$ by means of the heaviside $\theta$-function.  Of
course, this does not neet to be done in the massive-zero case as the
convolution already extends between $x$ and $1$.

\newpage

\begin{thebibliography}{alp}

%\cite{Alekhin:2003ev}
\bibitem{Alekhin:2003ev}
  S.~I.~Alekhin and J.~Blumlein,
  %``Mellin representation for the heavy flavor contributions to deep inelastic structure functions,''
  Phys.\ Lett.\ B {\bf 594} (2004) 299
  [hep-ph/0404034].
  %%CITATION = HEP-PH/0404034;%%
  %63 citations counted in INSPIRE as of 12 Feb 2015

%\cite{Laenen:1992xs}
\bibitem{Laenen:1992xs}
  E.~Laenen, S.~Riemersma, J.~Smith and W.~L.~van Neerven,
  %``O(alpha-s) corrections to heavy flavor inclusive distributions in electroproduction,''
  Nucl.\ Phys.\ B {\bf 392} (1993) 229.
  %%CITATION = NUPHA,B392,229;%%
  %154 citations counted in INSPIRE as of 12 Feb 2015

%\cite{Forte:2010ta}
\bibitem{Forte:2010ta}
  S.~Forte, E.~Laenen, P.~Nason and J.~Rojo,
  %``Heavy quarks in deep-inelastic scattering,''
  Nucl.\ Phys.\ B {\bf 834} (2010) 116
  [arXiv:1001.2312 [hep-ph]].
  %%CITATION = ARXIV:1001.2312;%%
  %98 citations counted in INSPIRE as of 12 Feb 2015

%\cite{Buza:1995ie}
\bibitem{Buza:1995ie}
  M.~Buza, Y.~Matiounine, J.~Smith, R.~Migneron and W.~L.~van Neerven,
  %``Heavy quark coefficient functions at asymptotic values Q**2 >> m**2,''
  Nucl.\ Phys.\ B {\bf 472} (1996) 611
  [hep-ph/9601302].
  %%CITATION = HEP-PH/9601302;%%
  %164 citations counted in INSPIRE as of 12 Feb 2015

%\cite{Gluck:1996ve}
\bibitem{Gluck:1996ve}
  M.~Gluck, S.~Kretzer and E.~Reya,
  %``The Strange sea density and charm production in deep inelastic charged current processes,''
  Phys.\ Lett.\ B {\bf 380} (1996) 171
   [Erratum-ibid.\ B {\bf 405} (1997) 391]
  [hep-ph/9603304].
  %%CITATION = HEP-PH/9603304;%%
  %88 citations counted in INSPIRE as of 12 Feb 2015

\end{thebibliography}

\end{document}