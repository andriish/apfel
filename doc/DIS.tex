%% LyX 2.0.3 created this file.  For more info, see http://www.lyx.org/.
%% Do not edit unless you really know what you are doing.
\documentclass[twoside,english]{paper}
\usepackage{lmodern}
\renewcommand{\ttdefault}{lmodern}
\usepackage[T1]{fontenc}
\usepackage[latin9]{inputenc}
\usepackage[a4paper]{geometry}
\geometry{verbose,tmargin=3cm,bmargin=2.5cm,lmargin=2cm,rmargin=2cm}
\usepackage{color}
\usepackage{babel}
\usepackage{float}
\usepackage{bm}
\usepackage{amsthm}
\usepackage{amsmath}
\usepackage{amssymb}
\usepackage{graphicx}
\usepackage{esint}
\usepackage[unicode=true,pdfusetitle,
 bookmarks=true,bookmarksnumbered=false,bookmarksopen=false,
 breaklinks=false,pdfborder={0 0 0},backref=false,colorlinks=false]
 {hyperref}
\usepackage{breakurl}

\makeatletter

%%%%%%%%%%%%%%%%%%%%%%%%%%%%%% LyX specific LaTeX commands.
%% Because html converters don't know tabularnewline
\providecommand{\tabularnewline}{\\}

%%%%%%%%%%%%%%%%%%%%%%%%%%%%%% Textclass specific LaTeX commands.
\numberwithin{equation}{section}
\numberwithin{figure}{section}

%%%%%%%%%%%%%%%%%%%%%%%%%%%%%% User specified LaTeX commands.
\usepackage{babel}

\@ifundefined{showcaptionsetup}{}{%
 \PassOptionsToPackage{caption=false}{subfig}}
\usepackage{subfig}
\makeatother


\usepackage{listings}

\begin{document}

\title{The APFEL DIS Module}

\author{Valerio Bertone$^{a}$}

\institution{$^{a}$PH Department, TH Unit, CERN, CH-1211 Geneva 23, Switzerland}
\maketitle

\begin{abstract}
In this document I will descrive the old and the new DIS module
embedded in APFEL.
\end{abstract}
\tableofcontents{}

\newpage{}

\section{Usage of the DIS code in APFEL}

Starting from release 2.0.0, APFEL implements a new module that
computes DIS neutral and charged current structure functions and 
cross sections up to order $\alpha_s^2$ (when possible) in the FONLL,
FFN and ZM-VFN schemes. In this short document we provide a short user
manual to use the new DIS module in APFEL.

The new DIS module can be used either in conjunction with the PDF
evolution provided by APFEL iteself or directly interfaced to LHAPDF.
To obtain any DIS observables it is enough to call one single
function, $i.e.$ {\tt DIS\_xsec}, providing a set of input parameters
that are needed to specify the computation to be performed.

Here follows the list of input/output parameters taken by the function {\tt DIS\_xsec}:
\begin{lstlisting}
APFEL::DIS_xsec(x,q0,q,y,proc,scheme,pto,pdfset,irep,target,proj,
                          F2,F3,FL,sigma);
\end{lstlisting}
where the input parameters are:
\begin{itemize}
\item the real variable {\tt x} that is the value of the Bjorken variable where the DIS
 obsevables are computed.
\item the real variable {\tt q0} that is the value of the initial scale (in GeV) used in the
 PDF evolution. This entry is used only if the APFEL internal
 evolution is used.
\item the real variable {\tt q} that is the value of the scale (in GeV) where the DIS
 observable is computed.
\item the real variable {\tt y} that is the value of the inelasticity where the DIS
observables are computed.
\item The string variable {\tt proc} that can take the values {\tt "EM"} for the
  electro-magnetic DIS computation (photon only exchange), {\tt "NC"}
  for the neutral current computation and {\tt CC"} for a charged
  current computation.
\item The string variable {\tt scheme} that can take the values {\tt "FONLL"} for FONLL, {\tt "FFNS"}
  for the FFN scheme and {\tt ZMVN"} for the ZM-VFN scheme. There is a
  further possibility, that is {\tt FFN0}, which refers to the
  zero-mass limit of the FFN scheme.
\item The integer variable {\tt pto} that can take the values 0, 1 or
  2. The user should be aware of the fact that when chosing {\tt
    pto=1} with {\tt scheme="FONLL"} the code automatically uses the
  FONLL-A scheme while when chosing {\tt
    pto=2} with {\tt scheme="FONLL"} the code automatically uses the
  FONLL-C scheme. As for the FONLL-B scheme, we
  plan to implement it in one of the next releases.
\item The string variable {\tt pdfset} that can take any of the PDF set
  names included in LHAPDF (including the {\tt .LHgrid} extension,
  $e.g$ {\tt pdfset ="NNPDF23\_nnlo\_as\_0118.LHgrid"}). This way APFEL will use
  the selected PDF set to compute the DIS observables using the evolution
  given in the grid itself. As an alternative, the user can choose {\tt
    pdfset="APFEL"}. This way the DIS observables will be computed
  using the evolution provided by APFEL bewteen the scales {\tt q0} and {\tt
    q}. For the setting of the PDF evolution, the user can refer to
  the APFEL manual.
\item The integer value {\tt irep} that specifies the replica of the PDF
  set to be used in the computation.
\item The string variable {\tt target} that takes the value {\tt
    "PROTON"} in case the target is a proton, {\tt "NEUTRON"} in case
  the target is a netron (assuming the isospin symmetry) or {\tt
    "ISOSCALAR"} if the target is an isoscalar, $e.g.$ a
  deuteron (also this option assumes isospin asymmetry). There is a
  further option which is {\tt "IRON"} which in addition uses the cross-section
  normalization used in the NeTeV experiment which is on iron nuclei.
\item The string variable {\tt proj} that takes the values {\tt
    "ELECTRON"} and {\tt "POSITRON"} if {\tt proc="EM"} or {\tt "NC"}
  in case the projectile is either an electron or a positron. If
  instead {\tt proc="CC"} the variable {\tt proj} can also take the
  values {\tt "NEUTRINO"} or {\tt "ANTINEUTRINO"} with obviuos meaning.
\end{itemize}

Once all these input parameters have been provided, the output
array variables are {\tt F2, F3, FL} and {\tt sigma}. Each of them has
5 entries corresponding to light, charm, bottom, top and total
components of the corresponding quantity.
The user should be careful because in the FORTRAN interface the arrays
are numerated from 3 to 7 ($e.g.$ {\tt F2(3)} = $F_2^l$, {\tt F2(4)} =
$F_2^c$, {\tt F2(5)} = $F_2^b$, {\tt F2(6)} = $F_2^t$, {\tt F2(7)} =
$F_2^p$) while in the C++ version they are numerated from 0 to 4 ($e.g.$ {\tt F2(0)} = $F_2^l$, {\tt F2(1)} =
$F_2^c$, {\tt F2(2)} = $F_2^b$, {\tt F2(3)} = $F_2^t$, {\tt F2(4)} =
$F_2^p$)


\section{Computing DIS Structure Functions on a Grid}

In order to speed up and optimize the compution of the DIS structure
functions in {\tt APFEL} we decided to use the same technology used
for the PDF evolution. In fact, up to version 2.0.0, the computation
of such observables in {\tt APFEL} was perfomed by directly
convoluting PDFs with the coefficient functions by mean of a
numerical integration.

Now the aim is that of precomputing on a grid the convolution of the
coefficient functions with a set of interpolating polynomials. This
way, the time consuming task of precomputing the coefficient functions
of the grid needs to be done only once and the numerical convolution
with any PDF set is instead very fast. In addition, as we will see
below, this approach provides a very natural framework to combine
precomputed coefficient functions and evolution operators, so that any
prediction of structure functions at any scale $Q$ can be obtained
very quickly by convolution with PDFs at some initial scale $Q_0$. 
Ultimately, this is particularly suitable for PDF fits.

A structure function in the Zero-Mass (ZM) scheme is given by the
following convolution:
\begin{equation}
F(x,Q) = \sum_{i=g,q}x\int_x^1\frac{dy}y C_i\left(\frac{x}{y},\alpha_s(Q)\right)q_i(y,Q)\,,
\end{equation}
Now, defining $t = \ln (Q^2)$, $\widetilde{C}_i^{(n)}(y,t) = yC_i^{(n)}(y,\alpha_s(Q))$
and $\widetilde{q}_i(y,t) = y q_i(y,Q)$, the integral above can be
written as:
\begin{equation}\label{exact}
F(x,t) = \sum_{i=g,q}\int_x^1\frac{dy}y \widetilde{C}_i\left(\frac{x}{y},t\right)\widetilde{q}_i(y,t)\,.
\end{equation}
But, using a suitable interpolation basis, we can write:
\begin{equation}
\tilde{q}_i(y,t)=\sum^{N_{x}}_{\alpha=0}w_{\alpha}^{(k)}(y)\tilde{q}_i(x_{\alpha},t)\,,
\end{equation}
so that eq. (\ref{exact}) becomes:
\begin{equation}
F(x,t) = \sum_{i=g,q}\sum^{N_{x}}_{\alpha=0}\left[\int_x^1\frac{dy}y \widetilde{C}_i\left(\frac{x}{y},t\right) w_{\alpha}^{(k)}(y)\right]\tilde{q}_i(x_{\alpha},t)\,.
\end{equation}
Now let's assume that $x$ is on the grid, so that $x = x_\beta$. This
way we have:
\begin{equation}
F(x_\beta,t) = \sum_{i=g,q}\sum^{N_{x}}_{\alpha=0}\underbrace{\left[\int_{x_\beta}^1\frac{dy}y \widetilde{C}_i\left(\frac{x_\beta}{y},t\right) w_{\alpha}^{(k)}(y)\right]}_{\Gamma_{i,\beta\alpha}(t)}\tilde{q}(x_{\alpha},t)\,.
\end{equation}

Using the same arguments presented in the evolution code notes, we
have that:
\begin{equation}
\Gamma_{i,\beta\alpha}(t) \neq 0 \quad\mbox{for}\quad\beta\leq\alpha\,,
\end{equation}
and:
\begin{equation}
\Gamma_{i,\beta\alpha}(t) = \int_{c}^d\frac{dy}y \widetilde{C}_i\left(y,t\right) w_{\alpha}^{(k)}\left(\frac{x_\beta}{y}\right)
\end{equation}
with:
\begin{equation}\label{bounds2}
c =  \mbox{max}(x_\beta,x_\beta/x_{\alpha+1}) \quad\mbox{and}\quad d = \mbox{min}(1,x_\beta/x_{\alpha-k}) \,.
\end{equation}
The same symmetries holding for the splitting function case hold also here.

\section{Coefficient Functions Treatment}

The struncture of the DIS coefficient functions is very similar to
that of splitting functions with only one small complication, that is
the presence of a more divergent singular term. In practice the
structure of the DIS coefficient functions is the following:
\begin{equation}
\widetilde{C}_{i}(x,t) = xC_{i}^{R}(x,t) +
xC_{i}^{S1}(t)\left[\frac{1}{1-x}\right]_+ +
xC_{i}^{S2}(t)\left[\frac{\ln(1-x)}{1-x}\right]_+ + xC_{i}^{L}(t)\delta(1-x)\,.
\end{equation}
The term proportional to $C_{i}^{S2}$ can be treated, considering that:
\begin{equation}
\begin{array}{rcl}
\displaystyle \int_c^ddy\left[\frac{\ln(1-y)}{1-y}\right]_+f(y)
&=&\displaystyle \int_c^ddy\frac{\ln(1-y)}{1-y}\left[f(y)-f(1)\theta(d-1)\right]\\
\\
&+&\displaystyle \frac12 f(1) \ln^2(1-c)\theta(d-1)
\end{array}
\end{equation}
On the same line of splitting functions, we know that the coefficient
functions ha the following perturbative expansion:
\begin{equation}
C_{i}^{J}(x,t) = \sum_{n=0}^{N}a_s^{n}(t)C_{i}^{J,(n)}(x)\qquad\mbox{with}\qquad J=R,S1,S2,L
\end{equation}

Therefore one has that:
\begin{equation}
\begin{array}{c}
\displaystyle \Gamma_{j,\beta\alpha}(t) = \\
\\
\displaystyle \sum_{n=0}^{N} a_s^{n}(t) \bigg\{\int^{d}_{c}dy\left[{C}_{i}^{R,(n)}(y)w_{\alpha}\left(\frac{x_\beta}{y}\right)+\frac{{C}_{i}^{S1,(n)}+{C}_{i}^{S2,(n)}\ln(1-y)}{1-y}\left(w_{\alpha}\left(\frac{x_\beta}{y}\right)-\delta_{\beta\alpha}\theta(d-1)\right)\right]\\
\\
\displaystyle +\left[{C}_{i}^{S1,(n)}\ln(1-c)\theta(d-1)+\frac12{C}_{i}^{S2,(n)}\ln^2(1-c)\theta(d-1)+{C}_{i}^{L,(n)}\right]\delta_{\beta\alpha}\bigg\}\,.
\end{array}
\end{equation}

Calling:
\begin{equation}
\begin{array}{c}
\displaystyle \Gamma_{i,\beta\alpha}^{(n)}(t) = \\
\\
\displaystyle \int^{d}_{c}dy\left[{C}_{i}^{R,(n)}(y)w_{\alpha}\left(\frac{x_\beta}{y}\right)+\frac{{C}_{i}^{S1,(n)}+{C}_{i}^{S2,(n)}\ln(1-y)}{1-y}\left(w_{\alpha}\left(\frac{x_\beta}{y}\right)-\delta_{\beta\alpha}\theta(d-1)\right)\right]\\
\\
\displaystyle +\left[{C}_{i}^{S1,(n)}\ln(1-c)\theta(d-1)+\frac12{C}_{i}^{S2,(n)}\ln^2(1-c)\theta(d-1)+{C}_{i}^{L,(n)}\right]\delta_{\beta\alpha}\,,
\end{array}
\end{equation}
we have that:
\begin{equation}\label{splittingexp}
\Gamma_{i,\beta\alpha}(t) = \sum_{n=0}^{N} a_s^{n}(t) \Gamma_{i,\beta\alpha}^{(n)}\,,
\end{equation}
and the integrals $\Gamma_{i,\beta\alpha}^{(n)}$ do not depend on the energy therefore, once the grid (and the number of active flavours) has been fixed, they can be evaluate once and for all at the beginning and used for the convolution at any scale.

Now, assuming to have computed the evolution operator
$M_{ij,\alpha\beta}(t,t_0)$ between the
scales $t=\ln(Q^2)$ and $t_0=\ln(Q_0^2)$ on the same grid where we
have computed the operator $\Gamma_{i,\beta\alpha}(t)$, one can esily
combine the two obtaining the prediction for the structure function
$F$ on the grid in terms of PDFs at the initial scale $Q_0$ just by performing the
following convolution:
\begin{equation}
F(x_\alpha,t) = \Gamma_{i,\alpha\beta}(t) M_{ij,\beta\gamma}(t,t_0) \tilde{q}_j(x_{\gamma},t_0)
\end{equation}
where a sum of the repeated indeces is understood.

\end{document}